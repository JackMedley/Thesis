% !TEX root = thesis.tex

\chapter{$\zg$+Jets at the LHC}
\label{chap:Zs}

	Except where otherwise referenced the work in this chapter and the subsequent chapters refers to
	work undertaken by the author as part of the High Energy Jets collaboration.\\This work is the theoretical foundation for the
	full-flexible parton level Monte Carlo generator soon to be publicly available from

	\begin{center}
		\url{http://hej.web.cern.ch/HEJ/}
	\end{center}

	and is published in~\cite{ZPaper}.

\section{Introducing $\zg$+Jets at the LHC}

	The Large Hadron Collider (LHC) has opened up a new range of energies for hadronic
	collisions.  It has already been a resounding success with the discovery of the
	scalar Higgs boson completing the particle content of the Standard Model (SM).
	Hadronic colliders, by their very nature, lead to final states with large
	amounts of QCD radiation and being able to accurately describe this is
	essential.  Both the SM and many `Beyond the Standard Model' theories
	predict events with multiple hard jets in the final state and as seen in the
	previous chapter this can pose a serious problem for fixed-order descriptions.

	The current best approach for describing QCD radiation is through the use of
	a Monte Carlo (MC) generators using the principles outlined in section~\ref{sec:MC}.
	A wide range of such MC generators are available implementing everything from
	the fixed-order perturbative schemes discussed in chapter~\ref{chap:theory} to parton
	shower models which resum the logarithms arising from soft and collinear logarithms.
	The current state-of-the-art for fixed-order is the next-to-leading order calculation
	of $\zg$ plus 4 jets by the \texttt{BlackHat} collaboration~\cite{Ita:2011wn} while
	jet production in association with $\zg$ has only been computed for QCD merged with
	a parton shower for up to 2 jets~\cite{Re:2012zi,Campbell:2013vha}.
	This soft and collinear radiation is experimentally observed to cascade from outgoing
	high energy quarks and gluons in chaotic patterns we refer to as jets.  While parton
	showers do a good job of describing the composition of jets they cannot be expected to give the correct
	description of the large invariant mass region based solely on their resummation of soft and collinear logarithms.
	Merging parton showers with fixed order schemes helps give a better description of
	multi-jet states however the current state-of-the-art for the fixed order component is
	still limited to next-to-leading order in $\alpha_s$ (with the exception of the recent
	$\text{N}^3$LO result~\cite{Anastasiou:2015ema}).

	In the \hej framework we aim to resum the large logarithmic corrections
	arising from well separated (in rapidity), hard final state jets.  We capture these
	important terms by calculating those
	diagrams which contribute a `leading logarithm' in the High Energy limit
	at all orders in $\alpha_s$.

	In the remainder of this chapter we discuss how we can describe the production
	of di-leptons plus multiple hard jets through the emission of an electroweak $Z^0$
	boson and an off-shell photon, $\gamma^*$.  We do this by constructing a current
	describing $\zg$ emission from one of the incoming quark or anti-quark lines and
	then combine this with a `passive' quark or gluon current as was described in
	section~\ref{sec:currents}.  The effective vertex derived in section~\ref{sec:effectiveVertices}
	can then be used to generalise the resulting matrix element to give an approximate
	description of the $(\zg\to)e^+e^- + n$ jets final state valid in the High Energy
	limit of QCD discussed in chapter~\ref{chap:HEQCD}.  The interference
	present from the multiple possible emission sites of the $\zg$ is described exactly by a
	generalisation of the $t$-channel picture which allows for multiple `chains' of
	momenta flowing through the reggeised gluons in the $t$-channel.  This approach
	requires a new regularisation procedure to carefully render the resulting
	matrix element finite when we consider the cancellation of poles from the Lipatov
	propagator terms and the effective vertices.  We also treat the interference arising
	from the two distinct emissions ($Z^0$ and $\gamma^*$) exactly.

	The formal accuracy of the description given here is LO+LL.  The leading order
	accuracy is achieved by performing a multiplicative matching to exact matrix
	elements generated using \texttt{MadGraph\_aMC@NLO}.  As discussed in section~\ref{sec:HEJ}
	this required a completely new matching set-up and this too is described later in
	this chapter along with some of the other computational challenges encountered
	along the way.

	Finally we present a comparison of results from \hej $\zg$ plus jets to several recent
	experimental studies at the LHC; both from the ATLAS collaboration and from
	the CMS experiment.  We see that we describe the data well in these studies
	and, in particular, in the regions of phase-space with large rapidity gaps and high invariant
	mass \hej gives a better description of the data than the other fixed-order and
	fixed-order plus parton shower predictions included.

\section{Constructing $\zg$+jets}
	\label{sec:Zcurrents}

	We now consider the construction of a current and an all orders
	inclusive cross-section for the $\zg$.  We start by looking at just
	the $Z^0$ emission from a single fermion line.

	\subsection{A Current for $Z^0$+Jets}
		\label{sub:zCurrent}

		For any given initial state (excluding the case of gluon-gluon scattering which will
		not contribute an FKL configuration) there are two possible emission sites for the
		$Z^0$ per fermion i.e. two for $qg$ and $\bar qg$ scattering and four for $qq$,
		$\bar qq$ and $\bar q\bar q$ scattering. The emission sites on a single fermion line
		are illustrated in fig.~\eqref{fig:emissionsites}.  In the language of currents discussed
		previously we call the left hand side of fig.~\eqref{fig:emissionsites} $j_\mu^{Z}$.
		It is given by:

		\begin{align}
		  \begin{split}
		    j^Z_\mu = \frac{C_{Zq}C_{Ze}}{p_Z^2 - M^2_Z +
		      i\Gamma_ZM_Z}\Bigg(&\frac{\bk{1}{\gu{\sigma}(\slashed p_{out} + \slashed
		        p_{e^+} + \slashed p_{e^-})\gl{\mu}}{a}}{(p_{out} + p_Z)^2} + \\
		    &\frac{\bk{1}{\gu{\mu}(\slashed p_{in} - \slashed p_{e^+} - \slashed
		        p_{e^-})\gl{\sigma}}{a}}{(p_{in} - p_Z)^2}\Bigg)\bk{e^+}{\gl{\sigma}}{e^-},
		    \label{eqn:Zcurrent}
		  \end{split}
		\end{align}

		\begin{figure}[bt]
			\includegraphics[width=0.98\linewidth]{figures/EmissionSites.pdf}
			\caption{The possible emission sites for a neutral weak boson.}
			\label{fig:emissionsites}
		\end{figure}

		where $M_Z$ is the mass of the $Z^0$ boson, $\Gamma_Z$ is its width, $C_{Zx}$ is
		the coupling of the $Z^0$ to $x$, $x=e,q,\nu_e,\ldots$ and $\mu$ is the Lorentz
		index for the $t$-channel gluon propagator.

		We can expand the quark and lepton momenta using their completeness relations which,
		in terms of spinor-helicity brackets, is given by:

		\begin{equation}
			\slashed p_{i} = |i_+\rangle\langle i_+| + |i_-\rangle\langle i_-|.
		\end{equation}

		This fixes the helicity of the incoming quark, $h_{\text{in}}$, and the outgoing quark,
		$h_{out}$, to be identical, and we are left with a current which only has four
		possible helicity configurations depending on $h_q = h_{\text{in}} = h_{\text{out}}$ and the
		electron helicity, $h_e$:

		\begin{align}
		  \begin{split}
		    j^Z_\mu(h_q, h_e) = &\ C_{Zq}^{h_q}C_{Ze}^{h_e}\ \frac{\bk{e^+_{h_e}}{\gl{\sigma}}{e^-_{h_e}}}{p_Z^2 -
		      M^2_Z + i\Gamma_ZM_Z} \\ &\ \times\bigg(\frac{2p_1^\sigma\bk{1_{h_q}}{\gu{\mu}}{a_{h_q}} +
		      \bk{1_{h_q}}{\gu{\sigma}}{e^+_{h_q}} \bk{e^+_{h_q}}{\gu{\mu}}{a_{h_q}} +
		      \bk{1_{h_q}}{\gu{\sigma}}{e^-_{h_q}} \bk{e^-_{h_q}}{\gu{\mu}}{a_{h_q}}}
		    {(p_{out} + p_Z)^2}\\
		    &\qquad + \frac{ 2p_a^\sigma\bk{1_{h_q}}{\gu{\mu}}{a_{h_q}} -
		      \bk{1_{h_q}}{\gu{\mu}}{e^+_{h_q}} \bk{e^+_{h_q}}{\gu{\sigma}}{a_{h_q}} -
		      \bk{1_{h_q}}{\gu{\mu}}{e^-_{h_q}} \bk{e^-_{h_q}}{\gu{\sigma}}{a_{h_q}}}
		    {(p_{in} - p_Z)^2}\bigg).
		    \label{eq:Zcurrent}
		  \end{split}
		\end{align}

		We can then express amplitudes for $Z^0$ plus jets in terms of contractions of
		a $Z^0$ emitting current with either a quark or gluon current discussed previously.  Taking
		the concrete example of $qg\to Zqg$ we can write the matrix element as follows:

		\begin{align}
		\begin{split}
			{|\bar{\mathcal{M}}_{qg\to Zqg}^{t}|}^2 =& \ \frac{g_s^2}{8}\
			\frac{1}{(p_a-p_1-p_{e^+}-p_{e^-})^2 (p_b-p_n)^2}  \sum_{h_q,h_e,h_g}|
			j^{Z}_\mu(h_q,h_e) j^{g\mu}(h_g)|^2.
		  	\label{eq:qgamp}
		\end{split}
		\end{align}

		We will investigate the behaviour of eqn.~\eqref{eq:qgamp} for a `slice' through the final state phase-space
		where each particles momenta is parametrised by:

		\begin{align*}
		\begin{split}
			p_i = &k_{i\perp}\Big(\cosh (y_i); \cos (\phi_i), \sin (\phi_i), \sinh (y_i)\Big) \\
		\end{split}
		\end{align*}

		and,

		\begin{align}
		\begin{split}
			k_{1\perp} = k_{e^+\perp} = 40\mbox{GeV}& \hspace{0.5cm} k_{e^-\perp} =
			\frac{m_Z^2}{2k_{e^+\perp}\left(\cosh(y_{e^+} - y_{e^-}) -
			    \cos(\varphi_{e^+} - \varphi_{e^-}))\right)}, \\
			\varphi_{1} =& \pi \hspace{0.5cm} \varphi_{e^+} = \pi + 0.2 \hspace{0.5cm}
			\varphi_{e^-} = -(\pi + 0.2), \\
			y_1=\Delta& \hspace{0.5cm} y_2=-\Delta \hspace{0.5cm} y_{e^+} = \Delta
			\hspace{0.5cm} y_{e^-} = \Delta - 1.5,
			\label{eq:momenta}
		\end{split}
		\end{align}

		So as $\Delta$ increases we pull the two jets apart in rapidity.  In this phase space slice
		the lepton pair are emitted in the forward region and so the physical picture is that the
		incoming quark with $p_a\sim p_1=p_+$ emits a $Z^0$ and then a $t$-channel gluon (or a
		$t$-channel gluon and \emph{then} a $Z^0$).

		We then observe the behaviour shown in fig.~\eqref{fig:ZatLO}: as we pull the
		jets apart in rapidity (i.e. as we go to large $\Delta$) we see that the matrix element
		approaches a constant; this is the result which would be obtained by using the BFKL
		formalism in which all jets are taken be infinitely well separated in rapidity.\\As we
		expect for the case of $2\to 2$ scattering we see exact agreement between our expression
		and the leading order result obtained from \texttt{MadGraph\_aMC@NLO}~\cite{Alwall:2014hca}.
		It is clear that the BFKL limit is not reached until relatively large values for $\Delta$,
		therefore it would be a poor approximation were we to just take the infinite rapidity limit
		of this as our expression for the $Z^0$ matrix element.

		\begin{figure}[hbt]
			\centering
			\includegraphics[width=1.0\linewidth]{Z_NoI_qg.pdf}
			\caption{$\overline{|\mathcal{M}_{qg\to Zqg}^{t}|}^2$ shown for a slice through the
			final state phase-space defined by eqn.~\eqref{eq:momenta}.  We compare to the
			leading order result obtained from \texttt{MadGraph\_aMC@NLO}.}
			\label{fig:ZatLO}
		\end{figure}

		The picture becomes more complicated when considering the $qQ\to ZqQ$ scattering since
		there are now four possible places where the (anti-)quark may emit the $Z^0$.  In
		previous work by the High Energy Jets collaboration the case of $W^\pm$ plus multiple
		hard jets was treated by attaching the boson to a single external quark line
		probabilistically~\cite{Andersen:2012gk}.  In the case of $W^\pm$ the interference
		terms arising from multiple boson emission sites is small and can be neglected, the reason for this is two
		fold. Firstly, the emission of the $W^\pm$ changes the flavour of the emitting quark line and so the
		final state will differ in almost all diagrams depending on where you emit (and hence there is no
		interference allowed), the exceptions to this are processes such as $uu\to(W^\pm\to)e+\nu_eud$ where
		either line could have been the emitter and so these are PDF suppressed. Secondly, in order to have
		interference the $W$ boson must be able to be emitted from multiple legs \emph{and} have the same charge
		wherever you attach it - again this is because the $W^\pm$ will decay to a distinct final state and no
		interference can occur. With these constraints in mind there are far fewer diagrams which contribute to the
		interference effects in $W^\pm$-plus-jets.  However as we will see later in the case of $\zg$ this leads to an approximate matrix
		element which differs significantly from the leading-order result and so here we will include not only the
		contributions to the matrix element arising from the $Z^0$ being emitted from both (anti-)quark
		legs separately but also the resulting interference term.

	\subsection{$Z^0$ Emission Interference}

		Our high-energy
		description of the matrix elements relies on the correct description of the
		$t$-channel momenta, and this obviously depends on which of the quark lines
		the $Z^0$ was emitted from.  We therefore need to modify the simple
		framework outlined above.  We will use the subscript $a$ ($b$) to label the
		current at the lowest (highest) end of the rapidity chain.  We then define $t_a$
		($t_b$) to be the $t$-channel momentum exchanged when the bosons are emitted at
		the lowest (highest) end of the rapidity chain.  Then the amplitude squared
		for $qQ\to qQ(Z^0\to) e^+ e^-$ is given by:

		\begin{align}
		\begin{split}
			{|\bar{\mathcal{M}}_{qQ\to ZqQ}^{t}|}^2 &= g_s^2 \frac{C_F}{8N_c}
			\Big|\frac{j^{Z^0}_a\cdot j_b}{t_a} + \frac{j_a\cdot
			j^{Z^0}_b}{t_b}\Big|^2\\
			&= g_s^2 \frac{C_F}{8N_c} \left( \Big|\frac{j^{Z^0}_a\cdot j_b}{t_a}\Big|^2 + \Big|\frac{j_a\cdot
			j^{Z^0}_b}{t_b}\Big|^2 + 2\Re{\Big\{\Big(\frac{j^{Z^0}_a\cdot
			j_b}{t_a}\Big)\Big(\frac{j_a\cdot j^{Z^0}_b}{t_b}\Big)^*\Big\}} \right),
			\label{eqn:interference}
		\end{split}
		\end{align}

		where $j_{a,b}$ are the pure quark currents defined previously.  The
		coupling constants of the $Z^0$ to the relevant quarks and leptons are contained
		within $j^{Z^0}(h_q,h_e)$, as in eqn.~\eqref{eqn:Zcurrent}.

		Fig.~\eqref{fig:twojets} shows the value of this matrix element squared scaled by the squared partonic
		centre-of-mass energy for increasing rapidity separation of the two jets. Once again the
		result is compared with that obtained from the full, tree-level matrix elements from
		\texttt{MadGraph\_aMC@NLO}.  The slice through phase space here is the same as that used in the
		previous section given by eqn.~\eqref{eq:momenta}.  Fig.~\eqref{fig:twojets} also
		shows the separate contributions to the total matrix element squared coming from the
		$\zg$ emission from the forward moving quark line (black, dashed) and emission
		from the backward moving quark line (green, dotted).  In this phase space slice,
		the leptons also have an increasing positive rapidity and so the forward emission
		matrix element describes the full matrix element most closely, with the contribution
		from backward-emission falling at large values of $\Delta y$.  The sum of the forward
		and backward emission matrix elements neglecting interference (magenta, dotted)
		significantly overestimates the final result.  Once the destructive interference
		effects have been taken into account, the full sum (red, solid) correctly reproduces
		the LO matrix element (blue, thick solid).  It is therefore clear that at low
		rapidities the inclusion of the interference effect plays an important r\^ole in
		the accuracy of the matrix element.

		Fig.~\eqref{fig:twojets} shows that in the region of very high rapidity separation
		the full matrix element squared (scaled by $s^2$ and an irrelevant phase space factor)
		approaches a constant.  We could have predicted this behaviour by considering
		eqn.~\eqref{eqn:HEfail2}; in the strict High Energy limit all the absolute rapidity
		information becomes lost and we only have dependence on $s$ and the transverse
		momenta of the outgoing partons (we still have information about the rapidity gap through
		$s$ using eqn.~\eqref{eqn:mandel2}).  In this case we also have the $Z^0$ propagator
		and its couplings to the partons and leptons to consider but the kinematics of the
		$t$-channel gluon still dominates.  The limit approached can be easily evaluated
		by applying the high energy limit, since in the $\Delta y\to\infty$ limit only
		the forward emission term contributes we have:

		\begin{align}
		\begin{split}
			\frac{{|\bar{\mathcal{M}}_{qQ\to ZqQ}^{HE}|}^2}{256\pi^5s^2} &= \frac{\alpha_s^2C_F^2}{64\pi^3(N_c^2-1)}
			\sum_{h_a, h_e}\frac{|C_{Za}^{h_a}C_{Ze}^{h_e}|^2}{|p_{1\perp}|^2|p_{2\perp}|^2}.
			\label{eqn:zLimit}
		\end{split}
		\end{align}

		Once we have scaled out the invariant mass squared divergence (as well as the usual phase
		space factor) out we have the limiting behaviour shown in fig~\eqref{fig:twojets}.

		\begin{figure}[hbt]
		  \begin{center}
		    \includegraphics[width=1.0\linewidth]{slice.pdf}
		    \caption{The matrix-element squared divided by the square of the partonic
		      centre-of-mass energy for $qQ\to ZqQ$ with the $Z^0$ decaying to an
		      electron-positron pair for the phase space slice described in
		      eqn.~\eqref{eq:momenta}.  Increasing values of $\Delta$ represent
		      increasing rapidity separation between the jets.  The different lines show the contributions from
		      different terms in the calculation: only emission from the forward or the
		      backward quark line (black, dashed and green, dotted), their sum without
		      the interference term (magenta, dotted) and their sum including
		      interference (red, solid) which is seen to agree exactly with the LO result
		      (blue, thick solid).}
		    \label{fig:twojets}
		  \end{center}
		\end{figure}

	\subsection{Photonic Interference}

		Since any virtual $Z^0$ decaying to an $e^+e^-$ pair could also have proceeded via an off-shell virtual photon,
		$\gamma^*$, we must also include these processes and the resulting interference
		between the $Z^0$ and $\gamma^*$.

		The $\gamma^*$ emission matrix element is similar to that of the $Z^0$-only matrix
		element shown in eqn.~\eqref{eq:qgamp} and the same story applies with the
		possible emission sites causing interference.  All that needs to be changed is the
		propagator term and the couplings of the boson to the emitting (anti-)quark and
		the decay products.  The full current for $\zg$ emission is then obtained by summing
		the two separate currents as follows:

		\begin{align}
			\label{eq:jsum}
			j^{\zg}_\mu(h_q,h_e) = j^Z_\mu(h_q,h_e) + j^\gamma_\mu(h_q,h_e).
		\end{align}

		Then upon squaring eqn.~\eqref{eq:jsum} we will automatically include the interference
		terms from the cross-terms. The inclusion of the virtual photon terms is particularly
		important when studying a combined lepton invariant mass, $(p_{e^+} + p_{e^-})^2$,
		far from the $Z^0$ Breit-Wigner mass peak. This can be seen in fig.~\eqref{fig:DileptonMass},
		where slices through phase space are shown similarly to fig.~\eqref{fig:twojets}, but
		now for an (a) lower and (b) higher value of the di-lepton mass.  In both cases, the
		contribution of the virtual photon processes is above 25\%.

		\begin{figure}[hbtp]
		        \centering
		        \begin{subfigure}[b]{0.78\textwidth}
		                \includegraphics[width=\textwidth]{ZLowMass}
		                \caption{}
		                \label{fig:LowDileptonMass}
		        \end{subfigure}
		        ~
		        \begin{subfigure}[b]{0.78\textwidth}
		                \includegraphics[width=\textwidth]{ZHighMass}
		                \caption{}
		                \label{fig:HighDileptonMass}
		        \end{subfigure}
		        \caption{The matrix-element squared divided by the square of the
		          partonic centre-of-mass energy for $qQ\to \zg qQ$ with the $\zg$ decaying
		          to an electron-positron pair.  The
		          $\mathcal{O}(\alpha_s^2 \alpha_W)$ tree-level contribution
		          as described in HEJ (red, dashed) exactly matches that of
		          \texttt{MadGraph\_aMC@NLO} (blue, solid).  The terms corresponding to the production
		          of a $Z^0$ boson only (green, dotted) significantly undershoots the
		          full result.  The
		          virtual photon terms are, therefore, clearly an important
		          contribution to the matrix element away from the $Z^0$ Breit-Wigner
		          peak.}
		        \label{fig:DileptonMass}
		\end{figure}

	\subsection{The $2\rightarrow n$ Matrix Element}

		Armed with eqn.~\eqref{eq:jsum} we can extend eqn.~\eqref{eq:qgamp} in the obvious way
		to form a complete matrix element for the emission of a $\zg$ boson.  We can also
		describe the various possibilities ($qq$, $qg$ and $gq$) simply by substituting
		the currents which apply in any given situation.  Of course, in practice we use
		the importance sampling techniques discussed in chapter~\ref{chap:HEQCD} to
		randomly sample the possible incoming parton types and so all combinations of
		currents are included.

		Following in the vein of the previous chapter we now look to extend our description
		to higher multiplicity final states.  Since our expression for the effective vertices are
		independent of the currents at either end of the $t$-chain we can write the squared matrix element for
		$qQ\to (\zg\to ) e^+e^- q (n-2)g Q$ as:

		\begin{align}
		\begin{split}
		    &|\mathcal{M}^{t}_{qQ\to \zg q(n-2)gQ}|^2 = \\&\frac{g_s^{2n}C_F}{8N_c}\
		    \ \times \Bigg( \frac{| j_a^{\zg}\cdot j_b|^2}{t_{a1}t_{a(n-1)}} \prod^{n-2}_{i=1} \frac{-\Ca V^2(q_{ai},
		      q_{a(i+1)})}{t_{ai} t_{a(i+1)}}  \\ & + \ \frac{|j_a \cdot j_b^{\zg} |^2}{t_{b1}t_{b(n-1)}}
		    \prod^{n-2}_{i=1}\frac{-\Ca V^2(q_{bi}, q_{b(i+1)})}{t_{bi} t_{b(i+1)}}  \\
		    &-\ \frac{2\Re\{(j_a^{\zg}\cdot j_b)({j_a \cdot
		        j_b^{\zg}})^*\}}{\sqrt{t_{a1}t_{b1}}\sqrt{t_{a(n-1)} t_{b(n-1)}}}
		    \prod^{n-2}_{i=1}\frac{\Ca V(q_{ai}, q_{a(i+1)})\cdot V(q_{bi},
		      q_{b(i+1)})}{\sqrt{t_{ai}t_{bi}} \sqrt{t_{a(i+1)}t_{b(i+1)}}}\Bigg).
			\label{eq:allorderreal}
		\end{split}
		\end{align}

		In the case of $n=2$, this reduces back to eqn.~\eqref{eqn:interference}.  If
		either $a$ or $b$ is an incoming gluon then there are only two possible emission sites
		for the $\zg$ once again and therefore we only need calculate one $t$-channel momenta chain
		and one can set the relevant $j_a^{\zg}$ or $j_b^{\zg}$ to
		zero in eqn.~\eqref{eqn:interference}.

		Fig.s~\eqref{fig:3jslice} and~\eqref{fig:4jslice} show the phase space
		slices for $qQ\to(\zg\to)e^+e^-qgQ$ and $qQ\to(\zg\to)e^+e^-qggQ$
		respectively, we see the same behaviour as observed for the 2 jet final state described in
		fig.~\eqref{fig:ZatLO}.  For these higher multiplicity final states we only
		approximate the leading order matrix element however we see that at large rapidity separations
		our approximation converges to the exact result as indeed it must in the High Energy limit.

		Eqn.~\eqref{eq:allorderreal} gives us the all-orders real corrections to $\zg$ plus jets which we wish to sum for all $n$.
		However, before proceeding to calculate the cross-section we must carefully render eqn.~\eqref{eq:allorderreal}
		finite by including the corresponding virtual corrections whose divergences will cancel the pathologies
		in the effective vertices.

		\begin{figure}[hbt]
		  \begin{center}
		    \includegraphics[width=1.0\linewidth]{slice3j.pdf}
		    \caption{A slice through phase space for the $\zg$+ 3 jet final state.  The slice defined is
		    akin to that described for the 2 jet case in fig.~\eqref{fig:ZatLO} where as $\Delta$ increases
		    we pull apart all three jets and the leptonic decay products are emitted increasingly far into
		    the forward direction.}
		    \label{fig:3jslice}
		  \end{center}
		\end{figure}

		\begin{figure}[hbt]
		  \begin{center}
		    \includegraphics[width=1.0\linewidth]{slice4j.pdf}
		    \caption{A slice through phase space for the $\zg$+ 3 jet final state.  The slice defined is
		    akin to that described for the 2 and 3 jet case shown in fig.~\eqref{fig:ZatLO} and
		    fig.~\eqref{fig:3jslice} respectively.}
		    \label{fig:4jslice}
		  \end{center}
		\end{figure}

\section{Regularising the $\zg$+Jets Matrix Element}
	\label{sec:regularising}

	\subsection{Real Soft Emissions}
		\label{sub:softEmissions}

		To calculate useful quantities such as cross sections etc. we must integrate equation
		eqn.~\eqref{eq:allorderreal} over all of phase space.  However, as discussed in chapter~\ref{chap:theory}
		problems arise when we attempt to integrate over the soft regions of phase space.  It is well understood
		that the divergences coming from soft real emissions cancel with those coming from
		virtual emissions and so we must explicitly show this cancellation and calculate the remaining
		finite contribution multiplying the $(n-1)$-final state parton matrix element.

		In the previous work on $W^\pm$ emission the finite remainder from this cancellation was found
		to be~\cite{Andersen:2009nu, Andersen:2008gc}:

		\begin{equation}
			\frac{\alpha_s C_a \Delta_{j-1, j+1}}{\pi}\ln{\frac{\lambda_{cut}^2}{|\vec{q}_{j\perp}|^2}},
		\end{equation}

		where $\Delta_{i-1, i+1}$ is the rapidity span of the final state partons either side of our
		soft emission and $|\vec{q}_{j\perp}|^2$ is the sum of squares of the transverse components of
		the $j^{th}$ $t$-channel gluon momenta.  $\lambda_{cut}$ is a parameter we choose which \emph{defines}
		the soft region.  That is, any real emission satisfying $p^2 \geq \lambda_{cut}^2$ we consider a hard perturbative
		emission while any emission with $p^2 < \lambda_{cut}^2$ we consider too soft to be resolved.\\
		Here we investigate
		the cancellation of these divergences for $Z^0$ emission and most importantly whether the finite term
		is of the same form for the interference term which was previously excluded.\\We start by looking
		at a $2\rightarrow n$ process and take the limit of one final state parton momentum, $p_i$, becoming
		small.  Because of the form of eqn.~\eqref{eq:allorderreal} this amounts to looking at the
		effect of an external gluon becoming soft on our expression for the effective vertex for real emissions.

		We can immediately see that for $p_i$ going soft the gluon `chain' momenta going into,
		and coming out of, the $j^{th}$ emission site will coincide: $q_{j+1}\sim q_j$, therefore:

		\begin{equation}
			V^\rho(q_j, q_{j+1}) = -2q_j^\rho - 2\left(\frac{s_{aj}}{s_{ab}} -
				\frac{q^2_{j}}{s_{bj}}\right)p_b^\rho + 2\left(\frac{s_{bj}}{s_{ab}} +
				\frac{q_j^2}{s_{aj}}\right)p_a^\rho
				\label{eqn:vertexlimit}
		\end{equation}

		Furthermore we can see that the two final terms from the bracketed expressions will dominate
		as $p_j\to0$ and so we can approximate the full vertex by:

		\begin{equation}
			V^\rho(q_j, q_{j+1}) \sim -2\frac{q^2_{j}}{s_{bj}}p_b^\rho + 2\frac{q_j^2}{s_{aj}}p_a^\rho.
				\label{eqn:vertexlimit2}
		\end{equation}

		In eqn.~\eqref{eq:allorderreal} we have three terms involving the effective vertex;
		quadratic terms like $V^2(q_{tj}, q_{t(j+1)})$ and $V^2(q_{bj}, q_{b(j+1)})$ and interference terms
		like $V(q_{tj}, q_{t(j+1)})\cdot V(q_{bj}, q_{b(j+1)})$.  The procedure for the $V^2$ terms follows
		similarly for both the quadratic top-line emission and bottom-line emission terms and so only
		the calculation for top-line emission is shown here.

		\subsubsection{$V^2(q_{tj}, q_{t(j+1)})$ Terms}

			Upon squaring eqn.~\eqref{eqn:vertexlimit2} and imposing the on-shell conditions for
			$p_a$ and $p_b$ we have:

			\begin{equation}
				V^2(q_{ti}, q_{ti}) = - \frac{4s_{ab}}{s_{bi}s_{ai}}q^4_{ti}
				\label{eqn:temp}
			\end{equation}

			We must now explicitly calculate the invariant mass terms.  Since we are in the
			High Energy regime we have that $p_a^+\gg p_a^-, p_{a\perp}$ and $p_b^+\gg p_b^-, p_{b\perp}$
			therefore we may take:

			\begin{align}
				s_{ab}= 2p_a\cdot p_b \sim 2p^+_ap^-_b,\\
				s_{bi}= 2p_b\cdot p_i \sim 2p^-_bp^+_i,\\
				s_{ai}= 2p_a\cdot p_i \sim 2p^+_ap^-_i.
			\end{align}

			Using this we can write eqn.~\eqref{eqn:temp} as:

			\begin{equation}
				V^2(q_{ti}, q_{ti}) = - \frac{4}{|\vec{p}_{1\perp}|^2}q^4_{ti},
				\label{eqn:temp2}
			\end{equation}

			Now looking back to eqn.~\eqref{eq:allorderreal} we see that each vertex is associated with factors of
			$q^{-2}_{ti}q^{-2}_{t(i+1)}$ but since the emission is soft $q_{ti}=q_{t(i+1)}$ and this becomes $q^{-4}_{ti}$.
			That factor conspires to cancel with the corresponding factor of $q^{4}_{ti}$ in eqn.~\eqref{eqn:temp2}.
			Including the additional factors of $C_A$ and $g_s$ the finite factor remaining is given by:

			\begin{equation}
				\frac{4C_Ag_s^2}{|\vec{p}_\perp|^2},
				\label{eqn:finalsoft}
			\end{equation}

		\subsubsection{$V(q_{ti}, q_{t(i+1)})\cdot V(q_{bi}, q_{b(i+1)})$ Terms}

			Taking the mixed dot-product of the two vertex terms we have:

			\begin{equation}
				V(q_{ti}, q_{ti})\cdot V(q_{bi}, q_{bi}) = -\frac{s_{ab}}{s_{ai}s_{bi}}t_{ti}t_{bi},
			\end{equation}

			having simplified the expression using $p_a^2=0$ and $p_b^2=0$ once again.  The invariant mass terms
			here are identical to those we saw in the $V^2$ terms and the products of $t_{ti}t_{bi}$ also appear
			in the denominator of the interference term in eqn.~\eqref{eq:allorderreal}.
			After this cancellation we are left with exactly what we had before in eqn.~\eqref{eqn:finalsoft}.
			Since the same factor comes from all three terms at the amplitude squared level we factor
			them out and express the amplitude squared for an $n$-parton final state with one soft emission in
			terms of an $(n-1)$-parton final state amplitude squared multiplied by the common factor:

			\begin{equation}
				\lim_{p_i\rightarrow0} |\mathcal{A}_{Z/\gamma}^{2\rightarrow n}|^2 = \left(\frac{4C_Ag_s^2}{|\vec{p}_{i\perp}|^2}\right)
					|\mathcal{A}_{Z/\gamma}^{2\rightarrow (n-1)}|^2
				\label{eqn:apparent}
			\end{equation}

		\subsubsection{Integration of Soft Divergences}

			As discussed above the divergences contained in eqn.~\eqref{eqn:apparent} only become
			apparent after we have attempted to integrate over phase space.  The Lorentz
			invariant phase space integral associated with $p_i$ is given by:

			\begin{equation}
				\int\frac{d^3\vec{p_i}}{(2\pi)^32E_i}\frac{4C_Ag_s^2}{|\vec{p}_\perp|^2}.
			\end{equation}

			It is convenient to exchange the integral over the $z$-component of momentum with one over rapidity,
			$y_2$.  Rapidity and transverse momentum are related through the definition of rapidity given
			in eqn.~\eqref{eqn:rap} and the Jacobian of this transformation is given by:

			\begin{align*}
				\frac{dy}{dp_z} = \frac{1}{E}.
			\end{align*}

			Therefore upon performing the change of variables the phase space integral reads:

			\begin{equation}
				\int_{\text{soft}}\frac{d^{2+2\epsilon}\vec{p}_{\perp}}{(2\pi)^{2+2\epsilon}}\int_{y_{i-1}}^{y_{i+1}}\frac{dy}{4\pi}\frac{4C_Ag_s^2}
					{|\vec{p}_\perp|^2}\mu^{-2\epsilon} = \frac{4C_Ag_s^2\mu^{-2\epsilon}}{(2\pi)^{2+2\epsilon}4\pi}
					\Delta_{i-1, i+1}\int_0^{\lambda_{\text{cut}}}\frac{d^{2+2\epsilon}\vec{p}_{\perp}}{|\vec{p}_\perp|^2},
			\end{equation}

			where we have analytically continued the integral to $2+2\epsilon$ dimensions to regulate the anticipated
			divergences and introduced the parameter $\mu$ to keep the coupling dimensionless in the process.
			We have also introduced an upper bound on the transverse momentum integration of $\lambda_{\text{cut}}$ -
			this parameter will be discussed in more detail later.  Converting to polar coordinates and using
			the result for the volume of a unit hypersphere gives the integrated soft contribution:

			\begin{equation}
				\frac{4C_Ag_s^2}{(2\pi)^{2+2\epsilon}4\pi}\Delta_{i-1, i+1}
				\frac{1}{\epsilon}\frac{\pi^{1+\epsilon}}
				{\Gamma(\epsilon+1)}\left(\frac{\lambda_{cut}^2}{\mu^2}\right)^\epsilon.
				\label{eqn:soft}
			\end{equation}

			As promised eqn.~\eqref{eqn:soft} is clearly divergent in the limit where $\epsilon\to0$.

	\subsection{Virtual Emissions}

		As discussed in chapter~\ref{chap:HEQCD} the virtual emission diagrams are included
		using the Lipatov ansatz for the gluon propagator:

		\begin{equation}
			\frac{1}{q_i^2}\longrightarrow\frac{1}{q_i^2}e^{\hat{\alpha}(q_i)\Delta_{i,i-1}},
		\end{equation}

		where:

		\begin{equation}
			\hat{\alpha}(q_i) = \alpha_sC_Aq_i^2\int \frac{d^{2+2\epsilon}k_{\perp}}{(2\pi)^{2+2\epsilon}}
			\frac{1}{k^2_\perp(k_\perp - q_{i\perp})^2}\mu^{-2\epsilon}.
			\label{eqn:feynmanPs}
		\end{equation}

		To see the cancellation of the infrared $\epsilon$ poles we must perform the integral
		explicitly using dimensional regularisation. Using Feynman parameters to re-express
		eqn.~\eqref{eqn:feynmanPs}:

		\begin{align}
			\hat{\alpha}(q_i) &= \alpha_sC_Aq_i^2\int \frac{d^{2+2\epsilon}\hat{k}_{\perp}}{(2\pi)^{2+2\epsilon}}\int_0^1
				\frac{dx}{[\hat{k}^2 _\perp + q_{i\perp}^2(1-x)]^2}\mu^{-2\epsilon},
		\end{align}

		where we have performed a change of variables to $\hat{k}_\perp = k_\perp - xq_{i\perp}$
		Changing the order of integration we can perform the $\hat{k}_\perp$ integral
		using the following result:

		\begin{equation}
			\int \frac{d^dk}{(2\pi)^d}\frac{1}{(k^2 - C)^\alpha} = \frac{1}{(4\pi)^{\frac{d}{2}}}
				\frac{\Gamma(\alpha - \frac{d}{2})}{\Gamma(\alpha)}
				\frac{(-1)^\alpha}{C^{\alpha - \frac{d}{2}}},
		\end{equation}

		to give:

		\begin{align}
			\hat{\alpha}(q_i) &= -\frac{2g_s^2C_A}{(4\pi)^{2+\epsilon}}\frac{\Gamma(1-\epsilon)}{\epsilon}
			\left(\frac{q_{i\perp}^2}{\mu^2}\right)^\epsilon,
		\end{align}

		having completed the $x$ integral and used the definition $\alpha_s=\frac{g_s^2}{4\pi}$.

	\subsection{Cancellation of Infrared Divergences}
		\label{sub:cancellation}

		We now have all the necessary ingredients to show how the infrared contributions from
		soft real emissions and virtual emissions cancel leaving our integrated matrix element
		finite.  The only subtlety being that we must sum two diagrams with different multiplicity
		final states to see the cancellation.  This is because they are experimentally indistinguishable;
		the $2\rightarrow (n-1)$ virtual diagram has $(n-1)$ resolvable partons in the final state
		and we only `see' $(n-1)$ of the final states partons from $2\rightarrow n$ process because
		we consider one of the emissions too soft to resolve.\\
		The matrix element squared for the real emission diagram with one soft parton
		will look like:

		\begin{align}
		\begin{split}
			|\mathcal{A}_{Z/\gamma}^{2\rightarrow n}|^2 = \left(\frac{4g_s^2C_a}{|p_{i\perp}|^2}\right)
				\Bigg[&\left|\mathcal{K}_a j_1^{Z/\gamma}\cdot j_2\right|^2
				\frac{\prod^{n-2}_{i\neq j}V^2(q_{ti},
				q_{t(i+1)})}{\prod^{n-1}_{i\neq j}q^2_{ti}} + \\
				&\left|\mathcal{K}_b j_2^{Z/\gamma}\cdot j_1\right|^2
				\frac{\prod^{n-2}_{i\neq j}V^2(q_{bi}, q_{b(i+1)})}{\prod^{n-1}_{i\neq j}q^2_{bi}} + \\
				&2\Re\{\mathcal{K}_a\overline{\mathcal{K}_b} \times
				(j_1^{Z/\gamma}\cdot j_2)(\overline{j_2^{Z/\gamma}\cdot j_1})\}\\
				&\times\frac{\prod^{n-2}_{i\neq j}V(q_{ti}, q_{t(i+1)})\cdot V(q_{bi}, q_{b(i+1)}))}
				{\prod^{n-1}_{i\neq j}q_{ti}q_{bi}}\Bigg],
		\end{split}
		\end{align}

		where we have taken the $i^{th}$ gluon to be soft.  After including the virtual corrections
		via the insertion of the Lipatov ansatz the $2\rightarrow (n-1)$ matrix element squared is:

		\begin{align}
		\begin{split}
			|\mathcal{A}_{Z/\gamma}^{2\rightarrow (n-1)}|^2 = &\left|\mathcal{K}_a j_1^{Z/\gamma}\cdot j_2\right|^2
				\frac{\prod^{n-3}_{i}V^2(q_{ti}, q_{t(i+1)})}{\prod^{n-2}_{i}q^2_{ti}}e^{2\hat{\alpha}(q_{ti})\Delta_{i-1,i+1}} + \\
				&\left|\mathcal{K}_b j_2^{Z/\gamma}\cdot j_1\right|^2 \frac{\prod^{n-3}_{i}V^2(q_{bi}, q_{b(i+1)})}
				{\prod^{n-2}_{i}q^2_{bi}}e^{2\hat{\alpha}(q_{bi})\Delta_{i-1,i+1}} + \\
				&2\Re\{\mathcal{K}_a\overline{\mathcal{K}_b} \times (j_1^{Z/\gamma}\cdot j_2)(\overline{j_2^{Z/\gamma}\cdot j_1})\}\\
				&\times\frac{\prod^{n-3}_{i}V(q_{ti}, q_{t(i+1)})\cdot V(q_{bi}, q_{b(i+1)}))}{\prod^{n-2}_{i}q_{ti}q_{bi}}
				e^{(\hat{\alpha}(q_{bi}) + \hat{\alpha}(q_{bi}))\Delta_{i-1,i+1}},
		\end{split}
		\end{align}

		We can now go through term-by-term to show the divergences cancel and find the finite contribution to
		the matrix element squared.  As when we calculated the soft terms the arguments for the pure
		top- and bottom-line emissions follow similarly and so here we will only state the procedure for
		the top emission.\\For the top line emission we identify the following terms that will appear in the
		sum of the $2\rightarrow (n-1)$ virtual and $2\rightarrow n$ real matrix elements.  The finite
		contribution, $\mathcal{F}_{\text{top}}$, to the matrix element is given by:

		\begin{equation}
			\mathcal{F}_{\text{top}} = \frac{4C_Ag_s^2}{(2\pi)^{2+2\epsilon}4\pi}\Delta_{i-1, i+1}
			\frac{1}{\epsilon}\frac{\pi^{1+\epsilon}}
			{\Gamma(\epsilon+1)}\left(\frac{\lambda_{cut}^2}{\mu^2}\right)^\epsilon +
			e^{2\hat{\alpha}(q_{ti})\Delta_{i-1,i+1}}.
		\end{equation}

		Extracting the relevant power of the strong coupling from the
		exponential and substituting for $\hat{\alpha}(q_i)$ gives:

		\begin{align}
		\begin{split}
			\mathcal{F}_{\text{top}} &= \frac{4C_Ag_s^2}{(2\pi)^{2+2\epsilon}4\pi}\Delta_{i-1, i+1}\frac{1}{\epsilon}\frac{\pi^{1+\epsilon}}
			{\Gamma(\epsilon+1)}\left(\frac{\lambda_{cut}^2}{\mu^2}\right)^\epsilon - -\frac{2g_s^2C_A}{(4\pi)^{2+\epsilon}}
			\frac{\Gamma(1-\epsilon)}{\epsilon}\left(\frac{q_{ti\perp}^2}{\mu^2}\right)^\epsilon, \\
			&= \frac{g_s^2C_A}{4^{1+\epsilon}\pi^{2+\epsilon}}\Delta_{i-1, i+1}\left(\frac{1}{\epsilon\Gamma(1+\epsilon)}
			\left(\frac{\lambda_{cut}^2}{\mu^2}\right)^\epsilon - \frac{\Gamma(1-\epsilon)}{\epsilon}
			\left(\frac{q_{ti\perp}^2}{\mu^2}\right)^\epsilon\right).
		\end{split}
		\end{align}

		Expanding the terms involving the regularisation parameter for small values,
		$\epsilon\to0$, yields:

		\begin{equation}
			\mathcal{F}_{\text{top}} = \frac{\alpha_sC_A\Delta_{i-1, i+1}}{\pi}
			\ln\left(\frac{\lambda_{cut}^2}{q_{ti\perp}^2}\right) + \mathcal{O}(\epsilon),
		\end{equation}

		where we have used:

		\begin{align}
		\begin{split}
			\frac{1}{\Gamma(1+\epsilon)} &= 1 + \gamma_E\epsilon + \mathcal{O}(\epsilon^2),
			\Gamma(1-\epsilon) = 1 + \gamma_E\epsilon + \mathcal{O}(\epsilon^2),\\
			\intertext{and,}
			\left(\frac{x}{y}\right)^\epsilon &= 1 + \epsilon\ln\left(\frac{x}{y}\right) +
			\mathcal{O}(\epsilon^2).
		\end{split}
		\end{align}

		For the terms arising from the bottom quark-line emission we have:

		\begin{equation}
		\mathcal{F}_{\text{bottom}} = \frac{\alpha_sC_A\Delta_{i-1, i+1}}{\pi}\ln\left(\frac{\lambda_{cut}^2}{q_{bi\perp}^2}\right),
		\end{equation}

		Lastly, for the interference terms we expand the exponential with both top-line emission, $q_{ti}$,
		momenta and bottom-line emission, $q_{bi}$, momenta to get:

		\begin{align}
		\begin{split}
			\mathcal{F}_{\text{interf.}} &= \frac{g_s^2C_A\Delta_{i-1, i+1}}{4^{1+\epsilon}\pi^{2+\epsilon}}
			\Big(\Big(\frac{1}{\epsilon} + \gamma_E +  \ln\Big(\frac{\lambda_{cut}^2}{\mu^2}\Big) +
			\mathcal{O}(\epsilon)\Big) - \\
			&\hspace{0.75cm}\frac{1}{2}\Big[\frac{2}{\epsilon} + 2 \gamma_E +
			\ln\Big(\frac{q_{ti\perp}^2}{\mu^2}\Big)
			- \ln\Big(\frac{q_{bi\perp}^2}{\mu^2}\Big) +
			\mathcal{O}(\epsilon)\Big]\Big) \\
			&= \frac{\alpha_sC_A\Delta_{i-1, i+1}}{\pi}\ln\Bigg(\frac{\lambda_{cut}^2}{\sqrt{q_{ti\perp}^2
			q_{bi\perp}^2}}\Bigg).
			\label{eqn:result}
		\end{split}
		\end{align}

		Eqn.~\eqref{eqn:result} is a new result which allows the inclusion of the interference terms shown
		to be important in previous discussion.  We can now express the regulated $qQ\to \zg q(n-2)gQ$
		matrix element as follows:

		\begin{align}
		  \label{eq:allordereg}
		  \begin{split}
		    &|\mathcal{M}^{HEJ-{\rm reg}}_{qQ\to \zg q(n-2)gQ}|^2 =\ g_s^2 \frac{C_F}{8N_c}\ ( g_s^2
		    \Ca)^{n-2}\  \\  \times \Bigg(& \frac{| j_a^{\zg}\cdot
		      j_b|^2}{t_{a1}t_{a(n-1)}}
		    \exp(\omega^0(q_{a(n-1)\perp})\Delta y_{n-1}) \prod^{n-2}_{i=1} \frac{-V^2(q_{ai},
		      q_{a(i+1)})}{t_{ai} t_{a(i+1)}} \exp(\omega^0(q_{ai\perp})\Delta y_i)\\
		    +\ &\frac{|j_a \cdot j_b^{\zg} |^2}{t_{b1}t_{b(n-1)}} \exp(\omega^0(q_{b(n-1)\perp})\Delta y_{n-1})
		    \prod^{n-2}_{i=1}\frac{-V^2(q_{bi}, q_{b(i+1)})}{t_{bi} t_{b(i+1)}} \exp(\omega^0(q_{bi\perp})\Delta y_i) \\
		    -\ &\frac{2\Re\{ (j_a^{\zg}\cdot j_b)(\overline{j_a \cdot
		        j_b^{\zg}})\}}{\sqrt{t_{a1}t_{b1}}\sqrt{t_{a(n-1)} t_{b(n-1)}}} \exp(\omega^0(\sqrt{q_{a(n-1)\perp}q_{b(n-1)\perp}})\Delta y_{n-1})\\
		    & \; \prod^{n-2}_{i=1}\frac{V(q_{ai}, q_{a(i+1)})\cdot V(q_{bi},
		      q_{b(i+1)})}{\sqrt{t_{ai}t_{bi}} \sqrt{t_{a(i+1)}t_{b(i+1)}}}\exp(\omega^0(\sqrt{q_{ai\perp}q_{bi\perp}})\Delta y_{i})\Bigg),
		  \end{split}
		\end{align}

		where we have defined

		\begin{align}
		  \label{eq:omega0}
		  \omega^0(q_{\perp}^2) = - \frac{g_s^2 \Ca}{4\pi^2} \ln\left( \frac{q_\perp^2}{\lambda_{cut}^2}\right).
		\end{align}

		There is one final improvement we can make to eqn.~\eqref{eq:allordereg}.  The expressions
		we obtain upon taking the soft limit of the three effective vertex terms, $V^2_t$, $V^2_b$
		and $V_t\cdot V_b$, are not exact and there are sub-leading terms which we can account for.
		We therefore have to account for the difference between, e.g $-V^2(q_{i-1},q_{i})/(t_{i-1} t_i)$,
		and its strict limit of $4/p_{i\perp}^2$ for values of $p_{i\perp}$ below $\lambda_{cut}$.  In
		practice, we include this correction for $c_{cut}<|p_\perp|<\lambda_{cut}$ with
		$c_{cut} = 0.2$~GeV and find stable results around this value (see section~\ref{sec:indep-lambd}).

		We are, at last, in a position to move forward and form an expression for an all-orders
		gauge invariant \emph{finite} matrix element for $\zg$ plus jets.  Before pressing on
		we now discuss an example calculation showing explicitly the finite nature of
		matrix element expression.

	\subsection{An Explicit Check: $2\rightarrow3$ Scattering}

		\begin{figure}[bth]
			\centering
			\begin{subfigure}[b]{0.48\textwidth}
				\includegraphics[width=1.0\linewidth]{RealSoftEmissionZ.pdf}
				\caption{}
				\label{fig:real24}
			\end{subfigure}
			\begin{subfigure}[b]{0.48\textwidth}
				\centering
				\includegraphics[width=1.0\linewidth]{VirtualSoftEmissionZ.pdf}
				\caption{}
				\label{fig:virtual23}
			\end{subfigure}

			\caption{Examples of both real and virtual diagrams contributing to
			$2\rightarrow3$ scattering. In fig.~\eqref{fig:real24} the $p_2$ has been
			drawn with a dashed line to denote it is not resolvable.  In
			fig.~\eqref{fig:virtual23} the final state momenta have been labelled in a
			seemingly strange way - this was done to make clear the cancellation
			when working through the algebra.}
			\label{fig:2to}
		\end{figure}

		Consider the case of $2\rightarrow3$ scattering where on of the final state momenta, $p_2$, has become soft.  A contributing
		soft diagram is shown in fig.~\eqref{fig:real24} and one example of a contributing virtual
		diagram of the same order is shown in fig.~\eqref{fig:virtual23}. When $p_2$ goes soft we have
		the following form for the $2\rightarrow3$ integrated amplitude squared ({N.B.}: The
		integration is only schematic and doesn't represent the full Lorentz invariant phase space):

		\begin{align}
		\begin{split}
			\int dPS|\mathcal{A}^{2\rightarrow3}_{soft}|^2 = &\frac{4C_Ag_s^2\Delta_{1,3}}{(2\pi)^{2+2\epsilon}4\pi}
			\frac{\pi^{\epsilon+1}}{\epsilon\Gamma(\epsilon+1)}
			\left(\frac{\lambda_{cut}^2}{\mu^2}\right)^\epsilon\Bigg[|\mathcal{K}_aj_1^Z\cdot j_2|^2
			\frac{V^2(q_{t1}, q_{t3})}{q^2_{t1}q^2_{t3}} \\
			+ |\mathcal{K}_bj_1\cdot j_2^Z|^2
			\frac{V^2(q_{b1}, q_{b3})}{q^2_{b1}q^2_{b3}} +
			& 2\Re\left\{\mathcal{K}_a\overline{\mathcal{K}_b}
			(j_1^Z\cdot j_2)\overline{(j_1\cdot j_2^Z)}\right\} \frac{V(q_{t1}, q_{t3})
			\cdot V(q_{b1}, q_{b3})}{q_{t1}q_{t3}q_{b1}q_{b3}}\Bigg],
		\end{split}
		\end{align}

		and the virtual contributions for the $2\rightarrow3$ amplitude are:

		\begin{align}
		\begin{split}
			\int dPS&|\mathcal{A}^{2\rightarrow3}_{virtual}|^2 = |\mathcal{K}_bj_1\cdot j_2^Z|^2
			\frac{V^2(q_{t1}, q_{t3})}{q_{t1}^2}e^{2\hat{\alpha}(q_{t1})\Delta_{1,3}} + \\
			&|\mathcal{K}_tj_1^Z\cdot j_2|^2 \frac{V^2(q_{b1}, q_{b3})}{q_{b1}^2}e^{2\hat{\alpha}(q_{b1})\Delta_{1,3}} +  \\
			& 2\Re\left\{\mathcal{K}_a\overline{\mathcal{K}_b}  (j_1^Z\cdot j_2)\overline{(j_1\cdot j_2^Z)}\right\}
			\frac{V(q_{t1}, q_{t3})\cdot V(q_{b1}, q_{b3})}{q_{t1}q_{t3}q_{b1}q_{b3}}e^{(\hat{\alpha}(q_{t1}) +
			\hat{\alpha}(q_{b1}))\Delta_{1,3}}.
		\end{split}
		\end{align}

		Once we expand the exponential to the correct order in $g_s^2$, the sum of these
		matrix elements squared over the region of phase space when $p_2$ is soft is:

		\begin{align}
		\begin{split}
			\int dPS&\left(|\mathcal{A}^{2\rightarrow3}_{soft}|^2 + |\mathcal{A}^{2\rightarrow3}_{virtual}|^2\right) =\\
			&|\mathcal{K}_aj_1^Z\cdot j_2|^2 \frac{V^2(q_{t1}, q_{t3})}{q_{t1}^2}
			{\Bigg(\frac{4C_Ag_s^2\Delta_{1,3}}{(2\pi)^{2+2\epsilon}4\pi}\frac{\pi^{\epsilon+1}}
			{\epsilon\Gamma(\epsilon+1)} - 2\hat{\alpha}(q_{t1})\Delta_{1,3}\Bigg)} + \\
			& |\mathcal{K}_bj_1\cdot j_2^Z|^2 \frac{V^2(q_{b1}, q_{b3})}{q_{b1}^2}
			{\Bigg(\frac{4C_Ag_s^2\Delta_{1,3}}{(2\pi)^{2+2\epsilon}4\pi}\frac{\pi^{\epsilon+1}}
			{\epsilon\Gamma(\epsilon+1)} - 2\hat{\alpha}(q_{b1})\Delta_{1,3}\Bigg)}+ \\
			&2\Re\left\{\mathcal{K}_a\overline{\mathcal{K}_b}  (j_1^Z\cdot j_2)\overline{(j_1\cdot j_2^Z)}\right\}
			\frac{V(q_{t1}, q_{t3})\cdot V(q_{b1}, q_{b3})}{q_{t1}q_{t3}q_{b1}q_{b3}}\\
			&{\Bigg(\frac{4C_Ag_s^2\Delta_{1,3}}{(2\pi)^{2+2\epsilon}4\pi}\frac{\pi^{\epsilon+1}}{\epsilon\Gamma(\epsilon+1)} -
			(\hat{\alpha}(q_{t1}) + \hat{\alpha}(q_{b1}))\Delta_{1,3}\Bigg)} + \mathcal{O}(g_s^4),
			\label{eqn:sjhdfa}
		\end{split}
		\end{align}

		The bracketed terms in eqn.~\eqref{eqn:sjhdfa} are exactly the cancellations calculated in section 4 above and therefore:

		\begin{align}
		\begin{split}
			\int dPS\left(|\mathcal{A}^{2\rightarrow3}_{soft}|^2 + |\mathcal{A}^{2\rightarrow3}_{virtual}|^2\right) =&
			\frac{\alpha_sC_A\Delta_{1,3}}{\pi}\Bigg(|\mathcal{K}_aj_1^Z\cdot j_2|^2 \frac{V^2(q_{t1},
			q_{t3})}{q_{t1}^2}\ln\left(\frac{\lambda_{cut}^2}{|q_{1t\perp}|^2}\right)+ \\
			& |\mathcal{K}_bj_1\cdot j_2^Z|^2 \frac{V^2(q_{b1}, q_{b3})}{q_{b1}^2}\ln
			\left(\frac{\lambda_{cut}^2}{|q_{1b\perp}|^2}\right)+ \\
			&2\Re\left\{\mathcal{K}_a\overline{\mathcal{K}_b}  (j_1^Z\cdot j_2)\overline{(j_1\cdot j_2^Z)}\right\}
			\frac{V(q_{t1}, q_{t3})\cdot V(q_{b1}, q_{b3})}{q_{t1}q_{t3}q_{b1}q_{b3}}\\
			&\ln\left(\frac{\lambda_{cut}^2}{\sqrt{|q_{1t\perp}|^2|q_{1b\perp}|^2}}\right)\Bigg) + \mathcal{O}(\alpha_s^2),
		\end{split}
		\end{align}

		which is manifestly finite.

\section{Subtractions and the $\lambda_{cut}$ scale}
	\label{sec:indep-lambd}

	We now show the stability of the High Energy Jets predictions with respect to the
	$\lambda_{cut}$ scale described above.\\We
	increase our sensitivity to the parameter by showing results for FKL momentum
	configurations only.  The non-FKL samples which are included to give the total
	cross sections have no dependence on $\lambda_{cut}$ and would therefore dilute
	any dependence in the full sample.  We begin with tab.~\eqref{tab:lambdaxs} where we show the value of
	the cross section for different values of $\lambda_{cut}$ for exclusive 2-, 3-
	and 4-jet samples.  The cuts applied are the same as in section~\ref{sub:ATLASZsec}.
	It is clear that the cross section does not display a large dependence on the
	value of $\lambda_{cut}$.

	\begin{table}[hbt!]
		\begin{center}
		\begin{tabular}{| c | c | c | c |}
		\hline
		$\lambda_{cut}$ (GeV) & $\sigma(2j)$ ($pb$) & $\sigma(3j)$ ($pb$) & $\sigma(4j)$ ($pb$) \\ \hline
		0.2 & $5.03 \pm 0.02$ & $0.70 \pm 0.02$ & $0.13 \pm 0.03$ \\
		0.5 & $5.05 \pm 0.01$ & $0.70 \pm 0.01$ & $0.13 \pm 0.01$ \\
		1.0 & $5.09 \pm 0.01$ & $0.71 \pm 0.01$ & $0.13 \pm 0.01$ \\
		2.0 & $5.16 \pm 0.04$ & $0.72 \pm 0.01$ & $0.13 \pm 0.01$ \\ \hline
		\end{tabular}
		\caption{The FKL-only cross sections for the 2-, 3- and 4-jet exclusive rates
		with associated statistical errors shown for different values of the regularisation parameter
		$\lambda_{cut}$.  The scale choice was half the sum over all transverse scales in the event, $H_T/2$.}
		\label{tab:lambdaxs}
		\end{center}
	\end{table}

	Fig.~\eqref{fig:lambdadist} shows the effect of the same variation in $\lambda_{cut}$ on the
	differential distribution in both the rapidity gap between the two leading jets in $p_\perp$,
	$\Delta y_{j1, j2}$, (a)--(c), and the rapidity gap between the two extremal jets in
	rapidity, $\Delta y_{jf, jb}$, (d)--(f).  Results are shown for exclusive 2-, 3-
	and 4-jet samples in each case.  The distributions also show a very weak
	dependence on the choice of $\lambda_{cut}$.
	In practice, our default chosen value for $\lambda_{cut}$ is 0.2.

	\begin{figure}[H]
		\centering
		\includegraphics[width=.32\textwidth]{Z_11a_2j.pdf}\hfill
		\includegraphics[width=.32\textwidth]{Z_11a_3j.pdf}\hfill
		\includegraphics[width=.32\textwidth]{Z_11a_4j.pdf}
		\caption{The effect of varying $\lambda_{cut}$ on the differential distribution
		in the rapidity gap between the two leading jets in $p_\perp$, $\Delta y_{j1, j2}$,
		with the $N_{jet}=2,3,4$ exclusive selections shown from left to right.
		$\lambda_{cut}=0.2$ (red), 0.5 (blue), 1.0 (green), 2.0 (purple).
		The bands represent the scale variation described in the text.}
		\label{fig:lambdadist}
	\end{figure}

	\begin{figure}[H]
		\centering
		\includegraphics[width=.32\textwidth]{Z_11c_2j.pdf}\hfill
		\includegraphics[width=.32\textwidth]{Z_11c_3j.pdf}\hfill
		\includegraphics[width=.32\textwidth]{Z_11c_4j.pdf}
		\caption{The effect of varying $\lambda_{cut}$ on the differential distribution
		in the rapidity gap between the two extremal jets in rapidity, $\Delta y_{jf, jb}$,
		with the $N_{jet}=2,3,4$ exclusive selections shown from left to right.
		$\lambda_{cut}=0.2$ (red), 0.5 (blue), 1.0 (green), 2.0 (purple).
		The bands represent the scale variation described in the text.}
		\label{fig:lambdadistdy}
	\end{figure}

\section{The Differential ${Z/\gamma}$ Cross-Section}

	Starting from eqn.~\eqref{eq:allordereg} we can write down a total (differential)
	cross section obtained by summing over all values of the number of final state partons, $n$, and
	integrating over the full $n$-particle phase space using an efficient Monte
	Carlo sampling algorithm~\cite{Andersen:2008ue,Andersen:2008gc}:

	\begin{align}
	  \label{eq:sigma}
	  \begin{split}
	    \sigma =& \sum_{f_a,f_b} \sum_{n=2}^\infty \int \frac{d^3 p_a}{(2\pi)^3 2E_a} \int \frac{d^3
	      p_b}{(2\pi)^3 2E_b}  \left( \prod_{i=1}^n \int \frac{d^3 p_i}{(2\pi)^3
	        2E_i} \right) \int \frac{d^3 p_{e^-}}{(2\pi)^3 2E_{e^-}} \int \frac{d^3 p_{e^+}}{(2\pi)^3 2E_{e^+}} \\
	    & \ \times (2\pi)^4 \delta^{(4)}\left(p_a+p_b -\sum_i p_i - p_{e^-} -p_{e^+}\right) \\
	    & \ \times \ |\mathcal{M}^{HEJ-{\rm reg}}_{f_af_b\to \zg
	      f_a(n-2)gf_b}(p_a,p_b,\{p_i\})|^2 \ \frac{ x_a f_{f_a}(x_a, Q_a) x_b
	      f_{f_b}(x_b,Q_b)}{\hat s^2}\ \Theta_{\rm cut},
	  \end{split}
	\end{align}

	where $x_{a,b}$ are the momentum fractions of the incoming partons and
	$f_{f_k}(x_k,Q_k)$ are the corresponding parton density functions for beam,
	k, and flavour $f_k$.  The function $\Theta_{\rm cut}$ imposes the desired cuts on the final
	state.  The minimum requirement is that the final state momenta cluster into at
	least two jets for the desired jet clustering algorithm.

	In the regions of phase space where all final state particles are well separated
	in rapidity, this gives the dominant terms in QCD at all orders in $\alpha_s$  (the leading logarithmic
	terms in $s/t$).  However, in other areas of
	phase space, the differences due to the approximations used in
	$|\mathcal{M}^{HEJ-{\rm reg}}_{qQ\to \zg q(n-2)gQ}|^2$ will become more
	significant as we saw in figs.~\eqref{fig:ZatLO},~\eqref{fig:3jslice} and~\eqref{fig:4jslice}.
	We therefore further improve upon eqn.~\eqref{eq:sigma} by
	matching our results to fixed order results.  Here, we match to
	high-multiplicity tree-level
	results obtained from \texttt{MadGraph\_aMC@NLO}~\cite{Alwall:2014hca} in two different ways.
	This amounts to merging tree-level samples of different orders according to the
	logarithmic prescription of HEJ.

	\begin{enumerate}
		\item Matching for FKL configurations:

		  As described in chapter~\ref{chap:HEQCD}, these are the particle assignments
		  and momentum configurations which contain the dominant leading-logarithmic
		  terms in $s/t$.  The first step of the HEJ description was to develop an
		  approximation to the matrix element for these processes which was later
		  supplemented with the finite correction which remained after cancelling the
		  real and virtual divergences: $\overline{|\mathcal{M}_{qg\to Zqg}^{HE}|}^2$
		  (eqn.~(\ref{eq:qgamp})) or $\overline{|\mathcal{M}_{qQ\to ZqQ}^{HE}|}^2$
		  (eqn.~(\ref{eq:allorderreal})).  The approximation is necessary to allow us to
		  describe the matrix element for any (and in particular, large) $n$ and for
		  including both the leading real and virtual corrections.  However,
		  if the parton momenta cluster into four or fewer \emph{jets} (these
		  may have arisen from many more partons), the full tree-level matrix element remains
		  calculable.  In these cases, we perform the matching multiplicatively, so we
		  multiply the integrand of eqn.~(\ref{eq:sigma}) by the ratio:

		  \begin{align}
		    |\mathcal{M}^{\rm full}_{qQ\to \zg q(k-2)gQ}(p_a,p_b,\{
		    j_i'\})|^2/|\mathcal{M}^{HEJ}_{qQ\to \zg q(k-2)gQ}(p_a,p_b,\{j_i'\})|^2.
		  \end{align}

		  Here, $\{j_i'\}$ are the jet momenta after a small amount of
		  reshuffling.  This is necessary because the evaluation of the tree-level matrix elements
		  assumes that the jet momenta are both on-shell and have transverse momenta which
		  sum to zero, neither of which is true in general for our events due to the
		  presence of extra emissions.  Our reshuffling algorithm~\cite{Andersen:2011hs} redistributes this
		  extra transverse momentum in proportion to the size of the transverse
		  momentum of each jet.  The plus and minus light-cone components are then adjusted such
		  that the jet is put on-shell and the rapidity remains unaltered.  This last
		  feature ensures that after reshuffling the event is still in an FKL
		  configuration.

		\item Matching for non-FKL configurations:

		  Away from regions in phase space where the quarks and gluons are
		  well-separated, the non-FKL configurations will play a more significant
		  r\^ole.  These have so far not been accounted for at all, and hence we add
		  three exclusive samples of leading-order two-jet, three-jet and four-jet
		  leading-order events to our resummed events.
	\end{enumerate}

	These two matching schemes complete our description of the production of $\zg$
	with at least two jets, including the leading high-energy logarithms at all
	orders in $\alpha_s$.  Tabs~\eqref{tab:matching1} and~\eqref{tab:matching2} show
	the effect of the matching to leading order on the total cross sections of various FKL
	configurations for 2 and 3-4 jet processes respectively.  We see that although the
	resummation-only result often gives a good approximation to the exact leading order result
	it sometimes differs but that this difference is corrected for by our matching.  Tab.~\eqref{tab:matching3}
	shows the total cross sections generated by \HEJ for 2-, 3- and 4-jet processes which contain non-FKL
	configurations.  Once again we see that after the inclusion of the extra exclusive sums we are in good
	agreement with the leading order result.

	\begin{table}[hbt]
	\centering
	\caption{The effect of matching on the total cross-section of the 2 jet final state FKL configurations.}
	\label{tab:matching1}
	\begin{tabular}{c|c|c|c}
	Incoming & Resum. & Resum.+FKL & Leading Order \\ \hline
	$(1,  1)$       &  $0.6550 \pm 0.0172$  &  $0.6742 \pm 0.0170$  & $0.6742 \pm 0.0170$ \\
	$(1,  2)$       &  $1.1030 \pm 0.0581$  &  $1.1030 \pm 0.0581$  & $1.1029 \pm 0.0581$ \\
	$(1,  3)$       &  $0.2667 \pm 0.0077$  &  $0.2667 \pm 0.0077$  & $0.2667 \pm 0.0077$ \\
	$(1,  4)$       &  $0.1991 \pm 0.0113$  &  $0.1992 \pm 0.0113$  & $0.1992 \pm 0.0113$ \\
	$(1,  5)$       &  $0.1085 \pm 0.0036$  &  $0.1085 \pm 0.0036$  & $0.1085 \pm 0.0036$ \\
	$(2,  2)$       &  $1.3672 \pm 0.0980$  &  $1.3910 \pm 0.0935$  & $1.3910 \pm 0.0935$ \\
	$(2,  3)$       &  $0.4832 \pm 0.0174$  &  $0.4832 \pm 0.0174$  & $0.4832 \pm 0.0174$ \\
	$(2,  4)$       &  $0.2744 \pm 0.0203$  &  $0.2744 \pm 0.0203$  & $0.2744 \pm 0.0203$ \\
	$(2,  5)$       &  $0.2033 \pm 0.0082$  &  $0.2033 \pm 0.0082$  & $0.2033 \pm 0.0082$ \\
	$(3,  3)$       &  $0.0837 \pm 0.0021$  &  $0.0880 \pm 0.0022$  & $0.0880 \pm 0.0022$ \\
	$(3,  4)$       &  $0.0630 \pm 0.0034$  &  $0.0630 \pm 0.0034$  & $0.0630 \pm 0.0034$ \\
	$(3,  5)$       &  $0.0313 \pm 0.0008$  &  $0.0313 \pm 0.0008$  & $0.0313 \pm 0.0008$ \\
	$(4,  4)$       &  $0.0310 \pm 0.0018$  &  $0.0326 \pm 0.0017$  & $0.0326 \pm 0.0017$ \\
	$(4,  5)$       &  $0.0236 \pm 0.0016$  &  $0.0236 \pm 0.0016$  & $0.0236 \pm 0.0016$ \\
	$(5,  5)$       &  $0.0114 \pm 0.0003$  &  $0.0121 \pm 0.0003$  & $0.0121 \pm 0.0003$ \\
	$(1, 21)$       &  $4.3680 \pm 0.1600$  &  $2.7868 \pm 0.0909$  & $2.7868 \pm 0.0909$ \\
	$(2, 21)$       &  $5.6100 \pm 0.3344$  &  $3.6284 \pm 0.1957$  & $3.6284 \pm 0.1957$ \\
	$(3, 21)$       &  $1.8842 \pm 0.0732$  &  $1.1718 \pm 0.0353$  & $1.1718 \pm 0.0353$ \\
	$(4, 21)$       &  $1.2172 \pm 0.1449$  &  $0.7136 \pm 0.0405$  & $0.7136 \pm 0.0405$ \\
	$(5, 21)$       &  $0.7480 \pm 0.0335$  &  $0.4697 \pm 0.0153$  & $0.4697 \pm 0.0153$ \\ \hline
	\end{tabular}
	\end{table}

	\begin{table}[hbt]
	\centering
	\caption{The effect of matching on the total cross-section of the 3 and 4 jet final state FKL
	configurations.}
	\label{tab:matching2}
	\begin{tabular}{c|c|c}
	Incoming & Resum.+FKL & Leading Order \\ \hline
	\multicolumn{3}{c}{3-jet} \\ \hline
	$(1, 1)$  & $0.3467 \pm 0.0202$ & $0.3467 \pm 0.0202$ \\
	$(1, 2)$  & $0.6851 \pm 0.0589$ & $0.6851 \pm 0.0589$ \\
	$(1, 3)$  & $0.1065 \pm 0.0073$ & $0.1065 \pm 0.0073$ \\
	$(1, 4)$  & $0.0684 \pm 0.0059$ & $0.0684 \pm 0.0059$ \\
	$(1, 5)$  & $0.0431 \pm 0.0032$ & $0.0431 \pm 0.0032$ \\ \hline
	\multicolumn{3}{c}{4-jet} \\ \hline
	$(3, 1)$  & $0.0011 \pm 0.0001$ & $0.0011 \pm 0.0001$ \\
	$(3, 5)$  & $0.0097 \pm 0.0006$ & $0.0097 \pm 0.0006$ \\ \hline
	\end{tabular}
	\end{table}

	\begin{table}[hbt]
	\centering
	\caption{The effect of matching on the total cross-section of the 2-, 3- and 4-jet final state non-FKL
	configurations.}
	\label{tab:matching3}
	\begin{tabular}{c|c|c|c|c}
	Incoming & Resum.+FKL & Non-FKL & Leading Order & \HEJ/Leading Order\\ \hline
	\multicolumn{5}{c}{2-jet} \\ \hline
	$(1, 2)$    &   $1.0985 \pm 0.1048$ &   $0.1047 \pm 0.0072$ &   $1.2260  \pm 0.0037$ & $0.9814  \pm 0.0857$ \\
	$(3, 4)$    &   $0.0706 \pm 0.0107$ &   $0.0086 \pm 0.0001$ &   $0.0804  \pm 0.0001$ & $0.98587 \pm 0.1334$ \\
	$(21, 21)$  &   $0.0000 \pm 0.0000$ &   $0.4002 \pm 0.9090$ &   $0.3612  \pm 0.0357$ & $1.00389 \pm 0.0878$ \\
	$(1,-1)$    &   $0.4262 \pm 0.0107$ &   $1.8962 \pm 0.0586$ &   $2.3460  \pm 0.0064$ & $0.98991 \pm 0.0255$ \\
	$(2,-2)$    &   $0.6151 \pm 0.0640$ &   $2.0954 \pm 0.0742$ &   $2.6770  \pm 0.0079$ & $1.01254 \pm 0.0367$ \\
	$(3,-3)$    &   $0.1250 \pm 0.0154$ &   $0.6116 \pm 0.0078$ &   $0.7576  \pm 0.0023$ & $0.97232 \pm 0.0230$ \\
	$(4,-4)$    &   $0.0733 \pm 0.0319$ &   $0.2308 \pm 0.0040$ &   $0.3096  \pm 0.0010$ & $0.98228 \pm 0.1037$ \\
	$(5,-5)$    &   $0.0186 \pm 0.0003$ &   $0.1211 \pm 0.0036$ &   $0.1447  \pm 0.0005$ & $0.96592 \pm 0.0254$ \\ \hline
	\multicolumn{5}{c}{3-jet} \\ \hline
        $(1,-1)$    &   $0.1713 \pm 0.0026$ &   $0.4848 \pm 0.0104$ &   $0.6112  \pm 0.0031$ & $1.0354 \pm 0.0183$ \\
        $(21, 21)$  &   $0.0000 \pm 0.0000$ &   $6.8566 \pm 0.2022$ &   $6.7920  \pm 0.0220$ & $1.0095 \pm 0.0299$ \\
        $(3, 1)$    &   $0.0008 \pm 0.0000$ &   $0.1615 \pm 0.0041$ &   $0.1633  \pm 0.0005$ & $0.9937 \pm 0.9937$ \\
        $(2,-5)$    &   $0.0724 \pm 0.0053$ &   $0.0627 \pm 0.0046$ &   $0.1300  \pm 0.0005$ & $1.0392 \pm 0.0544$ \\
        $(1, 21)$   &   $0.5804 \pm 0.0930$ &   $3.4127 \pm 0.2900$ &   $4.1498  \pm 0.0149$ & $0.9624 \pm 0.0735$ \\ \hline
	\multicolumn{5}{c}{4-jet} \\ \hline
	$(1, 2)$   &    $0.2308 \pm 0.0230$ &   $0.3199 \pm 0.0447$ &   $0.5491 \pm 0.0032$  & $1.0030 \pm 0.0917$ \\
	$(4, 21)$  &    $0.0133 \pm 0.0007$ &   $0.2728 \pm 0.0370$ &   $0.2780 \pm 0.0509$  & $1.0292 \pm 0.1334$ \\
	$(1,-1)$   &    $0.0545 \pm 0.0030$ &   $0.1544 \pm 0.0111$ &   $0.1965 \pm 0.0011$  & $1.0631 \pm 0.0588$ \\
	$(3,-3)$   &    $0.0002 \pm 0.0000$ &   $0.0366 \pm 0.0042$ &   $0.0333 \pm 0.0001$  & $1.1059 \pm 0.1251$ \\ \hline
	\end{tabular}
	\end{table}

	In the next sections, we discuss the computational aspects of the work presented
	here and compare the resulting predictions from this formalism to LHC data from
	recent ATLAS and CMS analyses.

\section{$\zg$+Jets: Computational Aspects}
	\label{sec:comp}

	The physics presented in the preceding sections is a significant departure from
	work previously done by the High Energy Jets collaboration.  As such developing
	the Monte Carlo for $\zg$ plus jets was a serious undertaking; the inclusion of
	the aforementioned interference terms required that two $t$-channel `chains' of
	momenta and vertices be computed and carried throughout the evaluation.
	Furthermore to correctly calculate at the amplitude level the way the \hej currents
	are constructed in the codebase needed to be modified.  Since the virtual corrections
	are scale dependent it was necessary to change large sections of code to work with
	multiple copies of $t$-chains and multiple scales.

	In previous \HEJ releases the matching mentioned above was performed using
	\texttt{MadGraph} version 4.  However, due to increases in speed and efficiency
	we chose to match the $\zg$+jets \HEJ matrix elements to \texttt{MadGraph\_aMC@NLO} version 5.
	While this might seem like a trivial change the underlying computational work
	was anything but simple; since the latest version of \texttt{MadGraph} outputs
	matrix elements in C++ (as well as Fortran 90) a completely new approach
	to incorporating matching was required.  A novel `abstract factory' design pattern
	was employed to efficiently construct and call the relevant leading order
	matrix element.  In this way we avoided the necessity of having extremely large
	matching files containing $\mathcal{O}(18,000)$ lines of code which was very difficult to
	debug and improve; instead this new structure allowed the matching code to
	be reduced to only a few thousand lines since the abstract factory class
	presents a uniform interface and therefore almost all of the process specific
	lines became unnecessary (some process specific content remains due to the
	distinction made between FKL and non-FKL configurations).

	Throughout the course of this work it became apparent that the \hej
	codebase, as it was at the time, needed to be restructured.  Each physics
	process (pure jets, $W^\pm$+jets, Higgs+jets and $\zg$+jets) was structured
	individually as a stand-alone piece of code.  This became a problem when testing
	and modifying \HEJ since there are large sections of code which are the same
	regardless of what electroweak boson emission (if any) we are concerned with,
	for example the parton distribution function calls are almost entirely the same
	no matter which code is run.  To improve upon this situation a unified \HEJ
	package was created.  This was a complete restructuring of the code into a form
	in which a general \texttt{HejGen} polymorphic base class can be
	constructed abstractly and then made concrete depending on user input.
	The unified version of the code is an improvement in that it is much more user
	friendly, and significantly easier to test and extend.  This work entailed
	being one of the primary authors of the new design for \hej and will soon be
	released as \HEJ (v2).

	Lastly a word about the generation of \hej predictions for comparisons to data.
	The sections and chapters which follow contain theoretical predictions to
	experimental analyses.  These predictions were generated using distributed
	computing both locally in Edinburgh and using the CERN \texttt{GRID} system.
	The former required the development of a set-up to distribute, execute and
	finalise jobs across a large network of machines standard desktop machines (i.e.~not
	actual computing nodes) distributed through Edinburgh University.  This was
	a time-consuming process however without this system it would not have been
	possible to produce the interesting results of chapter~\ref{chap:ATLAS} (which
	also contains a discussion of the computational challenges of generating \HEJA
	predictions). The \texttt{GRID} distributed computing work was available only in
	the final stages of this work because it became clear it was necessary (had it
	been available sooner the aforementioned local distributed computing set-up could
	have been avoided completely).  This involved a good deal of learning to work with
	distributed systems and working with the \texttt{Ganga} batch submission
	system which was, again, time-consuming.

\section{$\zg$+Jets at the LHC}

	\subsection{$\zg$+Jets at the ATLAS Experiment}
		\label{sub:ATLASZsec}

		We now compare the results of the formalism described in the previous sections
		to data.  We begin with a recent ATLAS analysis of $Z^0$-plus-jets events from
		7~TeV collisions~\cite{Aad:2013ysa}.  We summarise the cuts in tab.~\eqref{tab:atlascuts}.

		\begin{table}[bth]
		  \centering
		  \begin{tabular}{|l|c|}
		    \hline
		    Lepton Cuts & $p_{T\ell}>20$~GeV, \; $|\eta_\ell|<2.5$ \\
		    & $\Delta R^{\ell^+\ell^-} > 0.2$, \; $66$~GeV $\leq m^{\ell^+\ell^-} \leq
		      116$~GeV \\ \hline
		    Jet Cuts (anti-$k_T$, 0.4) & $p_{Tj}>30$~GeV, \; $|y_j|<4.4$ \\
		    & $\Delta R^{j\ell} >0.5$ \\
		\hline
		  \end{tabular}
		  \caption{Cuts applied to theory simulations in the ATLAS
		    $Z^0$-plus-jets analysis results shown in Figs.~\eqref{fig:ATLAS_2a}--\eqref{fig:ATLAS_7b}.}
		  \label{tab:atlascuts}
		\end{table}

		Any jet which failed the final isolation cut was removed from the event, but the
		event itself is kept provided there are a sufficient number of other jets
		present.  Throughout the central value of the \texttt{HEJ} predictions has been
		calculated with factorisation and renormalisation scales set to
		$\mu_F=\mu_R=H_T/2$, and the theoretical uncertainty band has been determined by
		varying these independently by up to a factor of 2 in each direction (removing
		the corners where the relative ratio is greater than two).  Also shown in the
		plots taken from the ATLAS paper are theory predictions from
		\texttt{Alpgen}~\cite{Mangano:2002ea}, \texttt{Sherpa}~\cite{Gleisberg:2008ta,Hoeche:2012yf},
		\texttt{MC@NLO}~\cite{Frixione:2002ik} and
		\texttt{BlackHat+Sherpa}~\cite{Berger:2010vm,Ita:2011wn}.  We will also comment on the
		recent theory description of Ref.\cite{Frederix:2015eii} .

		In Fig.~\eqref{fig:ATLAS_2a}, we begin this set of comparisons with predictions
		and measurements of the inclusive jet rates.  \texttt{HEJ} and most of the other theory
		descriptions give a reasonable description of these rates.  The \texttt{MC@NLO}
		prediction drops below the data because it only contains the hard-scattering
		matrix element for $\zg$ production and relies on a parton shower for additional
		emissions. The \texttt{HEJ} predictions have a larger uncertainty band which largely
		arises from the use of leading-order results in the matching procedures.  The effect
		of normalising our predictions (taking scale variations into account both in the
		numerator and denominator) on the size of our scale uncertainty bands is discussed
		later in this section.

		The first differential distribution we consider here is the distribution of the
		invariant mass between the two hardest jets, Fig.~\eqref{fig:ATLAS_11b}.  The
		region of large invariant mass is particularly important because this is a
		critical region for studies of vector boson fusion (VBF) processes in
		Higgs-plus-dijets.  Radiation patterns are largely universal between these
		processes, so one can test the quality of theoretical descriptions in
		$\zg$-plus-dijets and use these to inform the VBF analyses.  It is also a
		distribution which will be studied to try to detect subtle signs of new physics.
		In this study, \texttt{HEJ} and the other theory descriptions all give a good description
		of this variable out to 1~TeV, with \texttt{HEJ} being closest throughout the range.  The
		merged sample of Ref.~\cite{Frederix:2015eii} (Fig.~9 in that paper) combined
		with the \texttt{Pythia8} parton shower performs reasonably well throughout the range
		with a few deviations of more than 20\%, while that combined with \texttt{Herwig++}
		deviates badly.  Fig.~\eqref{fig:wJetsEg} shows the equivalent distribution from
		a recent ATLAS analysis of $W^\pm$+dijet events~\cite{Aad:2014qxa}, that distribution
		was extended out to an invariant mass of $2$~TeV and, as discussed in section~\ref{sec:HEJ},
		almost all of the theoretical predictions deviated significantly while the \texttt{HEJ}
		prediction remained flat.  This is one region where the high-energy logarithms
		which are only included in \texttt{HEJ} are expected to become large.

		In Fig.~\eqref{fig:ATLAS_11a}, we show the comparison of various theoretical
		predictions to the distribution of the absolute rapidity difference between the
		two leading jets.  It is clear in the left plot that \texttt{HEJ} gives an excellent
		description of this distribution.  This is to some extent expected as
		high-energy logarithms are associated with rapidity separations.  However, this
		variable is only the rapidity separation between the two hardest jets which is
		often not representative of the event as harder jets tend to be more central.
		Nonetheless, the \texttt{HEJ} description performs well in this restricted scenario.  The
		next-to-leading order (NLO) calculation of \texttt{Blackhat+Sherpa} also describes the
		distribution quite well while the other merged, fixed-order samples deviate from
		the data at larger values.  The merged samples of Ref.~\cite{Frederix:2015eii}
		(Fig.~8 in that paper) describe this distribution well for small values of this
		variable up to about 3 units when combined with \texttt{Herwig++} and for most of the
		range when combined with the \texttt{Pythia8} parton shower, only deviating above 5 units.

		The final distribution in this section is that of the ratio of the transverse
		momentum of the second hardest jet to the hardest jet.  The perturbative
		description of \texttt{HEJ} does not contain any systematic evolution of transverse
		momentum and this can be seen where its prediction undershoots the data at low
		values of $p_{T2}/p_{T1}$.  However, for values of $p_{T2} \gtrsim 0.5 p_{T1}$,
		the ratio of the \texttt{HEJ} prediction to data is extremely close to 1.  The
		fixed-order based predictions shown in Fig.~\eqref{fig:ATLAS_2a} are all fairly
		flat above about 0.2, but the ratio of the data differs by about 10\%.

		We have seen that in all of the figures presented in this section \HEJ has scale uncertainty bands
		significantly larger than other theoretical descriptions, including \texttt{Alpgen} who also
		have only leading order accuracy.  From fig.~\eqref{fig:ATLAS_2a} we see that we describe
		the experimentally observed inclusive two jet rate very well and, as such, do not require
		normalisation to agree with the data.  However, applying a normalisation procedure which
		consistently applies scale variation simultaneously in numerator and denominator significantly
		reduces the size of the scale uncertainty bands for High Energy Jets (or any theoretical prediction).
		In figs.~\eqref{fig:ATLAS_Z_7b_norm},~\eqref{fig:ATLAS_Z_11a_norm} and~\eqref{fig:ATLAS_Z_11b_norm}
		we show the normalised results from figs.~\eqref{fig:HEJ_ATLAS_7b},~\eqref{fig:HEJ_ATLAS_11a} and~\eqref{fig:HEJ_ATLAS_11b}.
		We see that, as expected, the central value of HEJ still describes the data well in the
		regions discussed above and now the size of the theoretical uncertainty band is significantly
		reduced (for example a reduction of approximately a factor of 250 is seen in the last bin of the
		$p_{\perp 2}/p_{\perp 1}$-distribution).  This illustrates that varying $\mu_R$ and $\mu_F$
		leads to a change in overall normalisation but not to any significant change in shape.  Therefore,
		it is still valuable to discuss the quality of agreement of the central line, despite their
		apparently large accompanying uncertainty bands.

		\begin{figure}[H]
		  \centering
		  \begin{subfigure}[b]{0.48\textwidth}
		    \includegraphics[width=\textwidth, height=1.2\textwidth]{ATLAS_Z_2a}
		    \label{fig:HEJ_ATLAS_2a}
		  \end{subfigure}
		  ~
		  \begin{subfigure}[b]{0.48\textwidth}
		    \includegraphics[width=\textwidth, height=1.2\textwidth]{Comparison_2a}
		    \caption{}
		    \label{fig:MC_ATLAS_2a}
		  \end{subfigure}
		  \caption{These plots show the inclusive jet rates from (a) \texttt{HEJ} and (b) other
		    theory descriptions and data~\cite{Aad:2013ysa}.  \texttt{HEJ} events all contain at
		    least two jets and do not contain matching for 5 jets and above, so these
		    bins are not shown.}
		  \label{fig:ATLAS_2a}

		  \begin{subfigure}[b]{0.48\textwidth}
		    \includegraphics[width=\textwidth, height=1.2\textwidth]{ATLAS_Z_11b}
		    \caption{}
		    \label{fig:HEJ_ATLAS_11b}
		  \end{subfigure}
		  ~
		  \begin{subfigure}[b]{0.48\textwidth}
		    \includegraphics[width=\textwidth, height=1.2\textwidth]{Comparison_11b}
		    \caption{}
		    \label{fig:MC_ATLAS_11b}
		  \end{subfigure}
		  \caption{These plots show the invariant mass between the leading and
		    second-leading jet in $p_T$.  As in Fig.~\eqref{fig:ATLAS_2a}, predictions are
		    shown from (a) \texttt{HEJ} and (b) other theory descriptions and
		    data~\cite{Aad:2013ysa}. These studies will inform Higgs plus dijets
		    analyses, where cuts are usually applied to select events with large
		    $m_{12}$.}
		  \label{fig:ATLAS_11b}
		\end{figure}

		\begin{figure}[H]
		  \centering
		  \begin{subfigure}[b]{0.48\textwidth}
		    \includegraphics[width=\textwidth, height=1.2\textwidth]{ATLAS_Z_11a}
		    \caption{}
		    \label{fig:HEJ_ATLAS_11a}
		  \end{subfigure}
		  ~
		  \begin{subfigure}[b]{0.48\textwidth}
		    \includegraphics[width=\textwidth, height=1.2\textwidth]{Comparison_11a}
		    \caption{}
		    \label{fig:MC_ATLAS_11a}
		  \end{subfigure}
		  \caption{The comparison of (a) \texttt{HEJ} and (b) other theoretical descriptions and
		    data~\cite{Aad:2013ysa} to
		    the distribution of the absolute rapidity different between the two leading
		    jets.  \texttt{HEJ} and \texttt{Blackhat+Sherpa} give the best description.}
		  \label{fig:ATLAS_11a}

		  \begin{subfigure}[b]{0.48\textwidth}
		    \includegraphics[width=\textwidth, height=1.2\textwidth]{ATLAS_Z_7b}
		    \caption{}
		    \label{fig:HEJ_ATLAS_7b}
		  \end{subfigure}
		  ~
		  \begin{subfigure}[b]{0.48\textwidth}
		    \includegraphics[width=\textwidth, height=1.2\textwidth]{Comparison_7b}
		    \caption{}
		    \label{fig:MC_ATLAS_7b}
		  \end{subfigure}
		  \caption{These plots show the differential cross section in the ratio of the leading
		     and second leading jet in $p_T$ from (a) \texttt{HEJ} and (b) other
		    theory descriptions and data~\cite{Aad:2013ysa}.}
		  \label{fig:ATLAS_7b}
		\end{figure}

		\begin{figure}[H]

			\centering

			\begin{subfigure}[b]{0.48\textwidth}
			  \includegraphics[width=0.9\textwidth]{ATLAS_Z_7b_norm}
			  \vspace{0.2cm}
			  \caption{}
			  \label{fig:ATLAS_Z_7b_norm}
			\end{subfigure}

			\begin{subfigure}[b]{0.48\textwidth}
			  \includegraphics[width=0.9\textwidth]{ATLAS_Z_11a_norm}
			  \vspace{0.2cm}
			  \caption{}
			  \label{fig:ATLAS_Z_11a_norm}
			\end{subfigure}
			~
			\begin{subfigure}[b]{0.48\textwidth}
			  \includegraphics[width=0.96\textwidth]{ATLAS_Z_11b_norm}
			  \caption{}
			  \label{fig:ATLAS_Z_11b_norm}
			\end{subfigure}
			\caption{The predictions of figs.~\eqref{fig:HEJ_ATLAS_7b},~\eqref{fig:HEJ_ATLAS_11a} and~\eqref{fig:HEJ_ATLAS_11b}
			normalised to the total cross-section, with scale variation consistently applied to numerator and denominator.}

			\label{fig:ATLAS_norm}
		\end{figure}

	\subsection{The $W^\pm$+Jets to $\zg$+Jets Ratio at the ATLAS Experiment}
		\label{sub:ATLASWZsec}

		In this section we present predictions for the ratio of $\zg$+Jets to
		$W^\pm$+Jets at all orders in $\alpha_s$.  We compare to the recent study undertaken
		by the ATLAS collaboration~\cite{Aad:2014rta}.  While $W^\pm$ plus jets and $\zg$ plus
		jets are both relevant separately for Standard Model physics and beyond the ratio of
		the two processes is particularly interesting as a precision test since many of the
		systematic errors which limit the $W^\pm, \zg$-plus-jets measurements cancel in the
		ratio.  The cuts for both final states are summarised in tab.~\eqref{tab:atlasWZcuts}.

		\begin{table}[h]
		\centering
		\begin{tabular}{|l|l|}
			\hline
			Lepton Cuts & $p_{T\ell}>25$~GeV, \; $|\eta_\ell|<2.5$ \\
			&  $\Delta R^{\ell^+\ell^-} > 0.2$ \\ \hline
			Reconstructed $Z$ Cuts &  $66$~GeV $< m^{\ell^+\ell^-} <116$~GeV \\
			\hline
			Reconstructed $W^\pm$ Cuts & $m_{TW} > 40$~GeV\; $\slashed E_{T} > 25$~GeV \\ \hline
			Jet Cuts (anti-$k_T$, 0.4) & $p_{Tj}>30$~GeV, \; $|y_j|<4.4$ \\
			& $\Delta R^{j\ell} >0.5$ \\
			\hline
			\end{tabular}
			\caption{Cuts applied to theory simulations in the analysis of the ATLAS $W^\pm$+jets/$Z$+jets ratio
		  	predictions shown in tabs.~\eqref{tab:RJetsIncl}--\eqref{tab:RJetsExcl}.}
			\label{tab:atlasWZcuts}
		\end{table}

		\begin{table}[!h]
			\begin{center}
			\begin{tabular}{| c | c | c | c |}
		        \hline
			$N_{jets}$ & Data $(\pm \text{stat.}\pm \text{syst.})$ & HEJ $(\pm \text{stat.}\pm \text{s.v.})$ & HEJ/Data $(\pm \text{stat.}\pm \text{s.v.})$ \\ \hline
			$\ge2$ & $8.64\pm0.04\pm0.33$ & $8.66\pm0.12^{+0.14}_{-0.16}$ & $1.00\pm0.01^{+0.02}_{-0.01}$ \\ \hline
			$\ge3$ & $8.18\pm0.08\pm0.52$ & $7.96\pm0.25^{+0.01}_{-0.01}$ & $0.97\pm0.03^{+0.01}_{-0.00}$ \\ \hline
			$\ge4$ & $7.62\pm0.20\pm0.95$ & $8.55\pm0.69^{+0.02}_{-0.02}$ & $1.12\pm0.09^{+0.00}_{-0.00}$ \\ \hline
			\end{tabular}
			\caption{The HEJ prediction for inclusive $R_{jet}$ rates at 2, 3 and 4 jets compared with ATLAS data.}
			\label{tab:RJetsIncl}
			\end{center}
		\end{table}

		\begin{table}[!h]
			\begin{center}
			\begin{tabular}{| c | c | c | c |}
		        \hline
			$N_{jets}$ & Data $(\pm \text{stat.}\pm \text{syst.})$ & HEJ $(\pm \text{stat.}\pm \text{s.v.})$ & HEJ/Data $(\pm \text{stat.}\pm \text{s.v.})$ \\ \hline
			2 & $8.76\pm0.05\pm0.31$ & $8.88\pm0.135^{+0.15}_{-0.18}$ & $1.01\pm0.02^{+0.021}_{-0.02}$ \\ \hline
			3 & $8.33\pm0.10\pm0.45$ & $7.85\pm0.265^{+0.01}_{-0.01}$ & $0.94\pm0.01^{+0.001}_{-0.03}$ \\ \hline
			4 & $7.69\pm0.21\pm0.71$ & $8.44\pm0.684^{+0.04}_{-0.04}$ & $1.10\pm0.01^{+0.005}_{-0.09}$ \\ \hline
			\end{tabular}
			\caption{The HEJ prediction for exclusive $R_{jet}$ rates at 2, 3 and 4 jets compared with ATLAS data.}
			\label{tab:RJetsExcl}
			\end{center}
		\end{table}

	\subsection{$\zg$+Jets at the CMS Experiment}
		\label{sub:CMS}

		We now compare to data from a CMS analysis of events with a $\zg$ boson produced
		in association with jets~\cite{Khachatryan:2014zya}.  We show, for comparison,
		the plots from that analysis which contain theoretical predictions from
		Sherpa~\cite{Gleisberg:2008ta,Hoeche:2012yf}, \texttt{Powheg}~\cite{Alioli:2010qp} and
		\texttt{MadGraph\_aMC@NLO}~\cite{Alwall:2014hca}.The cuts used for this analysis are summarised in
		tab.~\eqref{tab:cmscuts}.

		\begin{table}[hbt]
		  \centering
		  \begin{tabular}{|l|c|}
		    \hline
		    Lepton Cuts & $p_{T\ell}>20$~GeV, \; $|\eta_\ell|<2.4$ \\
		    &\; $71$~GeV $\leq m^{\ell^+\ell^-} \leq
		      111$~GeV \\ \hline
		    Jet Cuts (anti-$k_T$, 0.5) & $p_{Tj}>30$~GeV, \; $|y_j|<2.4$ \\
		    & $\Delta R^{j\ell} >0.5$ \\
		\hline
		  \end{tabular}
		  \caption{Cuts applied to theory simulations in the CMS
		    $Z^0$-plus-jets analysis results shown in
		    Figs.~\eqref{fig:CMS_2a}--\eqref{fig:CMS_3c}}
		  \label{tab:cmscuts}
		\end{table}

		As in the previous section, any jet which failed the final isolation cut was
		removed from the event, but the event itself is kept provided there are a
		sufficient number of other jets present.  The main difference to these cuts and
		those of ATLAS in the previous section is that the jets are required to be more
		central; $|\eta|<2.4$ as opposed to $|y|<4.4$.  This allows less room for
		evolution in rapidity; however, \texttt{HEJ} predictions are still relevant in this
		scenario.  Once again, the central values are given by $\mu_F=\mu_R=H_T/2$ with
		theoretical uncertainty bands determined by varying these independently by
		factors of two around this value.  \texttt{HEJ} events always contain a minimum of two
		jets and therefore here we only compare to the distributions for an event sample
		with at least two jets or above.

		We begin in Fig.~\eqref{fig:CMS_2a} by showing the inclusive jet rates for these
		cuts.  The \texttt{HEJ} predictions give a good description, especially for the 2- and
		3-jet inclusive rates in this narrower phase space. The uncertainty bands are
		larger for \texttt{HEJ} than for the \texttt{Sherpa} and \texttt{Powheg} predictions
		due to our LO matching prescription (those for \texttt{MadGraph\_aMC@NLO} are not shown).

		In Figs.~\eqref{fig:CMS_3b}--~\eqref{fig:CMS_3c}, we show the transverse momentum
		distributions for the second and third jet respectively (the leading jet
		distribution was not given for inclusive dijet events).  Beginning with the
		second jet in Fig.~\eqref{fig:CMS_3b}, we see that the \texttt{HEJ} predictions overshoot
		the data at large transverse momentum.  In this region, the non-FKL matched
		components of the \texttt{HEJ} description become more important and these are not
		controlled by the high-energy resummation.  The \texttt{HEJ} predictions are broadly
		similar to \texttt{Powheg}'s $Z^0$-plus-one-jet NLO calculation matched with the Pythia
		parton shower.  In contrast, \texttt{Sherpa}'s prediction significantly undershoots the
		data at large transverse momentum.  Here the \texttt{MadGraph\_aMC@NLO} prediction gives the best
		description of the data.

		Fig.~\eqref{fig:CMS_3c} shows the transverse momentum distribution of the third
		jet in this data sample.  Here, the ratio of the \texttt{HEJ} prediction to data shows a
		linear increase with transverse momentum (until the last bin where all the
		theory predictions show the same dip).  Both the \texttt{Sherpa} and \texttt{Powheg} predictions
		show similar deviations for this variable while the \texttt{MadGraph\_aMC@NLO} prediction again
		performs very well.

		\begin{figure}[H]
		  \centering
		  \begin{subfigure}[b]{0.46\textwidth}
		    \includegraphics[width=\textwidth, height=1.2\textwidth]{CMS_Z_2a}
		    \caption{}
		    \label{fig:HEJ_CMS_2a}
		  \end{subfigure}
		  ~
		  \begin{subfigure}[b]{0.48\textwidth}
		    \includegraphics[width=\textwidth, height=1.2\textwidth]{ComparisonCMS_2a}
		    \caption{}
		    \label{fig:MC_CMS_2a}
		  \end{subfigure}
		  \caption{The inclusive jet rates as given by (a) the \texttt{HEJ} description and (b)
		    by other theoretical descriptions, both plots compared to the CMS data in~\cite{Khachatryan:2014zya}.}
		  \label{fig:CMS_2a}

		  \begin{subfigure}[b]{0.46\textwidth}
		    \includegraphics[width=\textwidth, height=1.2\textwidth]{CMS_Z_3b}
		    \caption{}
		    \label{fig:HEJ_CMS_7b}
		  \end{subfigure}
		  ~
		  \begin{subfigure}[b]{0.48\textwidth}
		    \includegraphics[width=\textwidth, height=1.2\textwidth]{ComparisonCMS_3b}
		    \caption{}
		    \label{fig:MC_CMS_7b}
		  \end{subfigure}
		  \caption{The transverse momentum distribution of the second hardest jet in
		    inclusive dijet events in~\cite{Khachatryan:2014zya}, compared to (a) the
		    predictions from \texttt{HEJ} and (b) the predictions from other theory descriptions.}
		  \label{fig:CMS_3b}
		\end{figure}

		\begin{figure}[H]
		  \centering
		  \begin{subfigure}[b]{0.46\textwidth}
		    \includegraphics[width=\textwidth, height=1.2\textwidth]{CMS_Z_3c}
		    \caption{}
		    \label{fig:HEJ_CMS_7b}
		  \end{subfigure}
		  ~
		  \begin{subfigure}[b]{0.48\textwidth}
		    \includegraphics[width=\textwidth, height=1.2\textwidth]{ComparisonCMS_3c}
		    \caption{}
		    \label{fig:MC_CMS_7b}
		  \end{subfigure}
		  \caption{The transverse momentum distribution of the third hardest jet in
		    inclusive dijet events in~\cite{Khachatryan:2014zya}, compared to (a) the
		    predictions from \texttt{HEJ} and (b) the predictions from other theory descriptions.}
		  \label{fig:CMS_3c}
		\end{figure}

	\subsection{Differential Drell-Yan at the CMS Experiment}
		\label{sub:CMS2}

		Throughout the course of this work many analyses were used to compare \hej to experimental
		data.  Though we usually only take part in studies illuminating for a discussion of higher order
		logarithmic corrections of QCD processes at hadronic colliders the CMS collaboration asked that I
		contribute numbers (on behalf of the \hej collaboration) to a study of double-differential Drell-Yan
		at the LHC~\cite{CMS:2014vtk}.

		In particular, we consider
		the Drell-Yan cross-section differential in both the pseudo-rapidity gap between the reconstructed
		$\zg$ boson and the leading jet in $p_\perp$ and in the invariant mass of the di-lepton decay
		products.  This is \emph{not} a region of phase space where resummation is expected to reign
		supreme and we therefore anticipate that the effect of matching our \hej amplitudes to exact leading
		order results obtained using \texttt{MadGraph\_aMC@NLO} will be significant.

		The study focussed on 0-, 1- and 2-jet events in addition to a di-muon pair; since we can
		only describe final states with at least two jets we only consider the later final state.  The
		final state cuts applied in this analysis are shown in tab.~\eqref{tab:cmscuts2}.

		\begin{table}[hbt]
		  \centering
		  \begin{tabular}{|l|c|}
		    \hline
		    Lepton Cuts & $p_{\mu1\perp}>20$~GeV, \; $p_{\mu2\perp}>10$~GeV, \\
		    & \; $|\eta_\mu|<2.1$, \; $|\eta_{\zg}|<2.5$ \\ \hline
		    Jet Cuts (anti-$k_T$, 0.5) & $p_{Tj}>30$~GeV, \; $|\eta_j|<4.5$ \\ \hline
		  \end{tabular}
		  \caption{Cuts applied to theory simulations in the CMS
		   Drell-Yan analysis results shown in
		    Figs.~\eqref{fig:HEJ_CMS2_1}--\eqref{fig:HEJ_CMS2_5}}
		  \label{tab:cmscuts2}
		\end{table}

		Fig.~\eqref{fig:HEJ_CMS2_1} shows the distribution in the absolute rapidity gap
		between the reconstructed $\zg$ and the leading jet in $p_\perp$ for a reconstructed
		di-muon mass in the $Z^0$ peak range (defined as 60-120~GeV).  We see that both the
		leading order exact and \HEJ give a good description of data in this range.
		Figs.~\eqref{fig:HEJ_CMS2_3} and ~\eqref{fig:HEJ_CMS2_5} show the same distribution
		but for a reconstructed $\zg$ mass of between 30-60~GeV and 120-1500~GeV respectively.
		Here we see that both \texttt{MadGraph\_aMC@NLO} and \HEJ give a poor description of the data but
		agree well with one another.  This is exactly because the \HEJ predictions are being
		driven by the leading order matching in the regions considered.  This is a good
		consistency check of both the \hej matching scheme and the implementation of the
		importance sampling in \HEJ.

		In \HEJ the production of the Drell-Yan decay products via a $Z^0$ boson or an off-shell photon
		is implemented by using exactly the importance sampling scheme shown in chapter~\ref{chap:theory},
		eqn.~\eqref{eqn:schematicZ} and fig.~\eqref{fig:breitWigner} - that is we focus our matrix element
		evaluations predominantly around the $Z^0$ mass peak.  Fig.~\eqref{fig:HEJ_CMS2_2} was
		generated using the standard importance sampling (which is \emph{not} designed for this range of invariant masses) while
		fig.~\eqref{fig:HEJ_CMS2_3} was generated using a modified importance
		sampling approach which was better suited for probing the low mass end of the Breit-Wigner
		distribution.  Clearly using an intelligently chosen importance sampling scheme makes a big
		difference to the results - since, although the integrals are formally equal, when we come to
		perform the Monte Carlo integration we must use what computational resources we have to focus on
		the regions which contribute most to the integral.  Figs.~\eqref{fig:HEJ_CMS2_4}
		and~\eqref{fig:HEJ_CMS2_5} show the same distribution now for an invariant mass of between 120
		and 1500~GeV, once again fig.~\eqref{fig:HEJ_CMS2_4} was calculated with the default importance
		sampling whereas fig.~\eqref{fig:HEJ_CMS2_5} used a modified scheme:  the same behaviour can be
		seen with the default sampling describing the data poorly and being statistically limited.

		\begin{figure}[H]
			\centering
			\begin{subfigure}[b]{0.47\textwidth}
				\includegraphics[width=\textwidth]{CMS2_Z_M60_120H}
				\caption{}
				\label{fig:HEJ_CMS2_1}
			\end{subfigure}

			\begin{subfigure}[b]{0.47\textwidth}
				\includegraphics[width=\textwidth]{CMS2_Z_M30_60H}
				\caption{}
				\label{fig:HEJ_CMS2_2}
			\end{subfigure}
			~
			\begin{subfigure}[b]{0.47\textwidth}
				\includegraphics[width=\textwidth]{CMS2_F_M30_60H}
				\caption{}
				\label{fig:HEJ_CMS2_3}
			\end{subfigure}

			\begin{subfigure}[b]{0.47\textwidth}
				\includegraphics[width=\textwidth]{CMS2_Z_M120_1500H}
				\caption{}
				\label{fig:HEJ_CMS2_4}
			\end{subfigure}
			~
			\begin{subfigure}[b]{0.47\textwidth}
				\includegraphics[width=\textwidth]{CMS2_H_M120_1500H}
				\caption{}
				\label{fig:HEJ_CMS2_5}
			\end{subfigure}
			\caption{Comparisons of \HEJ and \texttt{MadGraph\_aMC@NLO} to data from a CMS study of double-differential
			Drell-Yan production.  Fig.~\eqref{fig:HEJ_CMS2_1} shows the mass range focussed on the $Z^0$ peak
			(60-120~GeV), figs.~\eqref{fig:HEJ_CMS2_2} and ~\eqref{fig:HEJ_CMS2_3} show the di-lepton invariant
			mass range from 30-60~GeV and lastly figs.~\eqref{fig:HEJ_CMS2_4} and ~\eqref{fig:HEJ_CMS2_5} show
			the mass range from 120-1500~GeV.  For figs.~(2.9b-e) which probe regions away from the Breit-Wigner peak
			two \HEJ lines are shown - figs.~\eqref{fig:HEJ_CMS2_2} and ~\eqref{fig:HEJ_CMS2_4} use the
			na\"ive Breit-Wigner sampling while figs.~\eqref{fig:HEJ_CMS2_2} and ~\eqref{fig:HEJ_CMS2_4}
			use a modified importance sampling scheme.}
		\end{figure}

\section{$\zg$+Jets Conclusions}

	In this chapter we have discussed augmenting the theoretical description of
	inclusive $\zg$-plus-dijets processes with the dominant logarithms in the High
	Energy limit at all orders in $\alpha_s$.  In particular, the description
	constructed here is accurate to leading logarithm in $s/t$.  This is
	achieved within the \hej framework.  We began in
	chapter~\ref{chap:HEQCD} by motivating and describing the construction of an
	approximation to the hard-scattering matrix element for an arbitrary number of
	gluons in the final state.  This uses factorised currents for electroweak boson
	emission and outer jet production combined with a series of (gauge-invariant)
	effective vertices for extra QCD real emissions.

	In contrast to previous HEJ constructions (for pure jets, $W$-plus-jets and
	Higgs boson-plus-jets), the complete description of the interference
	contributions between $Z$ and $\gamma^*$ processes \emph{and} between forward
	and backward emissions required a new regularisation procedure.  This is
	described in section~\ref{sec:regularising} where we showed explicitly the
	cancellation of real and virtual divergences by using the Lipatov ansatz to
	include the dominant contributions in the High Energy limit of the all-order
	virtual contributions.  The method by which we match our matrix element to the
	leading order matrix elements was also outlined here. In this way we
	achieve the formal accuracy of our Monte Carlo predictions to Leading
	Logarithmic in $s/t$ and merge Leading Order predictions in $\alpha_s$ for the
	production of two, three or four jets.

	In sections~\ref{sub:ATLASZsec} and~\ref{sub:CMS} we compared the predictions of our
	construction to $\zg$-plus-jets data collected at the ATLAS and CMS experiments
	during Run I.  We see excellent agreement for a wide range of observables and
	can be seen to describe regions of phase space well where some other
	fixed-order-based predictions do not fare as well.  Discrepancies between \HEJ and data which occur
	only do so in regions where we do not expect this description to perform as
	well, for example where there is a large ratio between $p_{T1}$ and $p_{T2}$.
	We also discuss properties of other available theoretical descriptions.

	This all-order description of $\zg$-plus-dijets allows predictions for the ratio
	of $W^\pm$+dijets to $\zg$+dijets at all-orders in $\alpha_s$ for the first
	time.  This is an extremely important analysis as many theoretical and
	experimental uncertainties cancel in this ratio and in section~\ref{sub:ATLASWZsec},
	we show that we correctly reproduce the ratios of the total cross sections.

	Just as for previous analyses of LHC data, it is found that the high-energy logarithms
	contained in HEJ are necessary for a satisfactory description of data in key regions of phases
	space, e.g.~at large values of jet invariant mass. Such regions of phase
	space are crucial for the analysis of Higgs boson production in association
	with dijets for example. The impact of the high-energy
	logarithms will only be more pronounced at the larger centre-of-mass energy of
	LHC Run II, and beyond at a possible future circular collider.  The HEJ
	framework and Monte
	Carlo is the unique flexible event generator to contain these corrections and
	will provide important theoretical input for the study of important processes
	at LHC Run II and beyond.

