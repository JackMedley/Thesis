% !TEX root = thesis.tex
\chapter{Dijets and Gap Jets at \texttt{ATLAS}}
\label{chap:ATLAS}

	Here we present the results of a complex experimental study of the effects of jet vetoes and
	azimuthal decorrelations in dijet events~\cite{Aad:2014pua}.  High Energy Jets is compared
	to both data and state-of-the-art fixed-order perturbative QCD predictions supplemented with
	radiation by merging with a parton shower from \texttt{POWHEG+
	PYTHIA8} and \texttt{POWHEG+HERWIG} (both implemented through the \texttt{POWHEG BOX} package
	\cite{1126-6708-2004-11-040}).

	The data compared to are 7~TeV proton-proton collisions as observed by the ATLAS
	experiment in two distinct data sets referred to as 2010 data and 2011 data.  The
	experimental cuts applied to these data sets differs and both are outlined in table
	\eqref{tab:atlasPcuts}.

	\begin{table}[bth]
	  \centering
	  \begin{tabular}{|l|c|}
	    \hline
	    2010 Jets (anti-$k_T$, 0.6) & $p_{Tj}>20$~GeV, \; $|y_j|<4.4$ \\
	    & $p_{T1}> 60.0$~GeV, \; $p_{T2}> 50.0$~GeV, \; \\
	    & $Q_0=20$~GeV \\
	    \hline
	    2011 Jets (anti-$k_T$, 0.6) & $p_{Tj}>30$~GeV, \; $|y_j|<2.4$ \\
	    & $p_{T1}> 60.0$~GeV, \; $p_{T2}> 50.0$~GeV \\
	    & \; $Q_0=30$~GeV, \; $\Delta y_{jj}>1.0$\\
	\hline
	  \end{tabular}
	  \caption{Cuts applied to theory simulations in the ATLAS dijets analyses.
	  $Q_0$ is the gap jet veto scale. The results are shown in
	  figs.~\eqref{fig:atlasPJ1}--\eqref{fig:atlasPJ6}.}
	  \label{tab:atlasPcuts}
	\end{table}

	This study focused on additional jet activity in dijet events where the dijet system
	is constructed using the two leading jets in $p_T$ - this, is in stark contrast to
	defining dijets by the most forward and most backward jets: naturally the $p_\perp$
	choice favours high transverse momentum dijets in the central region (and hence with
	relatively small rapidity gaps) while the forward-backward dijet system definition
	will lead to softer systems with bigger rapidity spans since this corresponds
	to the High Energy limit as shown in chapter \ref{chap:HEQCD}.  While both are of interest
	\HEJ tends to describe data best when when large rapidity  These two jets are required
	to be significantly harder than any additional jets with extra cuts on the leading
	and sub-leading jets of $50$~GeV and $60$~GeV respectively.  After tagging the two
	leading jets in the event and additional jet radiation is only considered in the rapidity
	interval bounded by the dijets as a QCD correction.  Within in the two data sets
	defined in table \eqref{tab:atlasPcuts} a further subdivision was made.  For both the
	2010 and 2011 data a subset of events was defined by vetoing events with extra
	QCD radiation in the rapidity interval bounded by the dijet system above some veto
	scale $Q_0$.  It should also be noted that the 2011 data set required a minimum
	rapidity gap of 1.0 between all jets (for both cases with and without the gap jet
	veto) in order to encourage a rapidity span into which extra radiation may arise.
	The full breakdown of this analysis then is into four event categories;
	2010 data with and without a veto applied to gap jets and the 2011 data with and
	without a veto applied to gap jets.

	The predictions for these analyses I generated on behalf of the High Energy Jets
	collaboration are given for both the partonic \HEJ calculation (shown in green in
	this chapter) and the \HEJA calculation (shown in orange). \ARIADNE is a parton
	shower package based on the Lund colour cascade dipole model.
	Due to the steps in this interfaced package necessary to remove the double counting
	(which arises from \HEJ and \ARIADNE both generating soft radiation) generating large
	data samples which can be used to give statistically significant predictions is
	\emph{extremely} computationally demanding.  In a standard partonic \HEJ run we
	evaluate the matrix element for each event at 76 different scale choices (since
	we allow for four choices of `central' scale and 19 combinations of the renormalisation
	scale, $\mu_r$, and factorisation scale, $\mu_f$ based on each central choice).
	After we chose a central scale (this choice is done while blind to the data for
	fairness) one set of these 19 scales combinations form an `envelope' of predictions
	for each bin in each plot - this spread is then represented by the green bands shown
	in the figs.~\eqref{fig:atlasPJ1}--\eqref{fig:atlasPJ6}.
	However, because of the structure of the matrix element evaluations within \HEJA
	we can only afford one scale choice per event and, as such, we cannot provide
	scale variation uncertainty bands with our showered numbers.  To be clear this
	is not a limitation of the physics since it is entirely possible to evaluate
	each matrix element multiple times - it is a computational consideration. \footnote{
	As seen in chapter \ref{chap:Zs}, the renormalisation scale appears in a non-trivial way
	in the High Energy Jets matrix element.  It is contained implicitly in the strong coupling
	constant which is contained within the virtual corrections exponential.  As such
	we cannot simply generate predictions at a single scale and then re-weight events
	at the post-analysis level.  In principle, we \emph{could} do this for our PDF
	evaluations though in practice this is very computationally cheap and therefore we
	chose not to.}  The orange bands shown with the \HEJA predictions throughout this
	chapter are the statistical bands (shown, as usual, at the 68\% confidence level).

	Observables were studied
	as a function of two properties of the dijets.  The rapidity gap between the dijets,
	$\Delta y$, and the mean traverse momenta of the dijet system, $\overline{p_T}$.\\
	Na\"ively for small $\Delta y$ we expect to see fewer gap jets since there is
	a limited region in which extra jets may be clustered before they are clustered in
	to the dijet system itself.   Conversely, as we pull the dijets apart in rapidity
	we expect to see an increase in additional QCD radiation (since there is a larger
	phase space in which to radiate).  Similarly for $\overline{p_T}$, we expect that
	events with harder dijet systems will have higher gap activity simply because they
	have more energy available to radiate off additional jets.

	Throughout the remainder of this chapter the (a) sub-figures show data and
	predictions for the 2010 data set with respect to the rapidity span of the dijet
	system, $\Delta y$, while the (b) sub-figures show data and predictions from the
	2011 data set with respect to the mean transverse momenta of the dijet system,
	$\overline{p_T}$.

	We begin by discussing the gap fraction, $f(Q_0)$, defined as:

	\begin{equation}
		f(Q_0) = \frac{\sigma_{jj}(Q_0)}{\sigma_{jj}},
	\end{equation}

	where $\sigma_{jj}$ is the total dijet cross section passing the cuts in
	table~\ref{tab:atlasPcuts} and $\sigma_{jj}(Q_0)$ is the dijet cross section passing
	the veto on additional gap scales for a scale choice $Q_0$.  Fig.~\eqref{fig:atlasPJ1}
	shows the gap fraction with a veto scale of $20$~GeV in
	fig.~\eqref{fig:atlasPJ1a} and a veto scale of $30$~GeV in fig.~\eqref{fig:atlasPJ1b}.
	The data are shown in black with the inner bars represent the statistical uncertainty
	while the outer lines are the total uncertainty arising from statistical and systematic
	effects.  The behaviour observed is in line with our expectation discussed previously since the
	gap fraction decreases at both large $\Delta y$ and large $\overline{p_T}$.  We can see the
	best description of both data sets is given by \HEJA (excluding two very high $\overline{p_T}$
	bins where the predictions from \HEJA are visibly statistically limited). The partonic \HEJ
	prediction overshoots both data sets meaning that we underestimate the jet activity in the
	gap region while the predictions from \texttt{POWHEG} plus parton showers overestimate the
	QCD radiation here.  From fig. \eqref{fig:atlasPJ1} it is clear that in this situation it
	is not sufficient to only describe the wide-angle logarithmically enhanced terms resummed
	within the High Energy Jets framework and that the soft and collinear logarithms added by
	interfacing to \ARIADNE play a very important r\^ole - equally it is clear that we cannot
	completely describe data without resumming the High Energy logarithms.  It is visible from
	fig.~\eqref{fig:atlasPJ1} that the scale variations shown in \emph{partonic} \HEJ (since the
	\HEJA bands are statistical only) are quite significantly bigger than that shown by the other
	predictions.  This is expected since we only include matching up to leading-order whereas
	\texttt{POWHEG} is formally NLO accurate; it is well observed that as we include higher order
	terms (in the fixed order sense) we gain a better control of the scale uncertainties.

	\begin{figure}[bth]
		\centering
		\begin{subfigure}[b]{0.48\textwidth}
			\includegraphics[width=\textwidth, height=1.0\textwidth]{pureJets3a}
			\caption{}
			\label{fig:atlasPJ1a}
		\end{subfigure}
		~
		\begin{subfigure}[b]{0.48\textwidth}
			\includegraphics[width=\textwidth, height=1.0\textwidth]{pureJets3b}
			\caption{}
			\label{fig:atlasPJ1b}
		\end{subfigure}
		\caption{The gap fraction, $f(Q_0)$, as a function of (a) the rapidity gap,
		$\Delta y$, and (b) the average $p_T$, $\overline{p_T}$, of the dijet system.}
		\label{fig:atlasPJ1}
	\end{figure}

	Similarly, when we study the mean number of jets in the rapidity interval shown for in
	fig.~\eqref{fig:atlasPJ2} we see that \HEJA and \texttt{POWHEG+PYTHIA8} give the best
	description of the data.  Once again the partonic \HEJ prediction undershoots the data
	significantly when describing both of the dijet characteristics while \texttt{POWHEG+HERWIG}
	overestimates the jet activity.

	\begin{figure}[bth]
		\begin{subfigure}[b]{0.48\textwidth}
			\includegraphics[width=\textwidth, height=1.0\textwidth]{pureJets4a}
			\caption{}
			\label{fig:}
		\end{subfigure}
		~
		\begin{subfigure}[b]{0.48\textwidth}
			\includegraphics[width=\textwidth, height=1.0\textwidth]{pureJets4b}
			\caption{}
			\label{fig:}
		\end{subfigure}
		\caption{The average number of jets, $\langle N_{\text{jets}}
		\text{ in the rapidity interval}\rangle$, in the rapidity gap
		         bounded by the dijet system, as a function of (a) the
		         rapidity gap, $\Delta y$, and (b) the average $p_T$,
		         $\overline{p_T}$, of the dijet system.}
		\label{fig:atlasPJ2}
	\end{figure}

	We now turn to look at the azimuthal decorrelations of dijet systems.  These are defined
	as $\langle\cos(n(\pi-\Delta\phi))\rangle$ with $n=1, 2, \ldots$.  Here we only consider
	the first and second moments.  We follow the notation of \cite{Aad:2014pua}
	and rewrite the second moment as $\langle\cos(2\Delta\phi)\rangle$.  Clearly for a final state
	with two partons momentum conservation will enforce that the jets be in a back-to-back configuration
	i.e. they will have $\Delta \phi=0$ and both $\langle\cos(\pi-\Delta\phi)\rangle$ and
	$\langle\cos(2\Delta\phi)\rangle$ will simply be a flat line $1.0$.  As we allow additional
	jets to be emitted the moments will depart from the straight line as the constraint softens
	and the extra radiation allows for $\Delta\phi<\pi$.  These moments have long been seen
	as an excellent test of the difference between DGLAP QCD parton showers and BFKL-like
	resummations \cite{Ducloue:2012bm}.  Indeed it is in these figures where we see the biggest
	difference between \HEJ and the \texttt{POWHEG} plus parton shower results.  This large
	difference between the high energy resummation of \HEJ and the parton shower approach clearly
	shows that the parton shower effects is being probed in these regions of phase space.

	Fig.~\eqref{fig:atlasPJ3} shows the first azimuthal moment for the inclusive selection.
	We see that in the 2010 study \HEJA and both \texttt{POWHEG} descriptions slightly
	underestimate $\langle\cos(\pi-\Delta\phi)\rangle$ while the partonic \HEJ result slightly
	overestimates.  However, the evolution of the first azimuthal moment with respect to
	$\overline{p_T}$ shows a clear difference between the two formalisms.  \HEJ does not
	radiate sufficiently to predict the correct decorrelation at low mean transverse momentum
	(which is understood since it does not include the parton shower effects) while both
	\texttt{POWHEG} descriptions cause too much decorrelation at low $\overline{p_T}$.
	The best description of the data is given by \HEJA since it adds extra emissions to High
	Energy Jets which improves our description of the decorrelation.

	\begin{figure}[bth]
		\centering
		\begin{subfigure}[b]{0.48\textwidth}
			\includegraphics[width=\textwidth, height=1.0\textwidth]{pureJets5a}
			\caption{}
			\label{fig:}
		\end{subfigure}
		~
		\begin{subfigure}[b]{0.48\textwidth}
			\includegraphics[width=\textwidth, height=1.0\textwidth]{pureJets5b}
			\caption{}
			\label{fig:}
		\end{subfigure}
		\caption{The first azimuthal angular moment, $\langle \cos(\pi-\Delta\phi)\rangle$,
		as a function of (a) the rapidity gap, $\Delta y$ and (b) the average $p_T$,
		$\overline{p_T}$, of the dijet system.}
		\label{fig:atlasPJ3}
	\end{figure}

	Another variable thought to be a good test of DGLAP vs. BFKL physics is the ratio of
	the second moment to the first moment.  This is shown in fig.~\eqref{fig:atlasPJ4}.  Once
	again we do see a sizeable difference in the predictions given by the two resummations.
	Similarly to fig.~\eqref{fig:atlasPJ3} we see that interfacing to the \ARIADNE package
	brings the partonic \HEJ prediction into a much better agreement of the data.

	\begin{figure}[bth]
		\begin{subfigure}[b]{0.48\textwidth}
			\includegraphics[width=\textwidth, height=1.0\textwidth]{pureJets5c}
			\caption{}
			\label{fig:}
		\end{subfigure}
		~
		\begin{subfigure}[b]{0.48\textwidth}
			\includegraphics[width=\textwidth, height=1.0\textwidth]{pureJets5d}
			\caption{}
			\label{fig:}
		\end{subfigure}
		\caption{The ratio of the second azimuthal angular moment, $\langle \cos(2\Delta\phi)\rangle$,
		to the first azimuthal angular moment, $\langle \cos(\pi-\Delta\phi)\rangle$, as a function of
		(a) the rapidity gap, $\Delta y$, and (b) the average $p_T$, $\overline{p_T}$, of the dijet system.}
		\label{fig:atlasPJ4}
	\end{figure}

	The two remaining figures are similar to figs. \eqref{fig:atlasPJ3} and \eqref{fig:atlasPJ4} but
	with the addition of the jet veto.  Similarly to the inclusive case we see that the partonic \HEJ
	predictions overestimates the correlation for the 2010 and the 2011 data sets while the NLO plus
	shower predictions, once again, understood the decorrelation.  Given the statistical limited
	data available (especially for the 2010 gap jet vetoed data set) it is more difficult to draw
	clear conclusions here but certainly the inclusion of the \ARIADNE shower improves the High
	Energy Jets results.

	Lastly we have the ratio of the second azimuthal moment to the first azimuthal moment for the
	events which pass the additional jet veto.  \HEJA and \texttt{POWHEG+PYTHIA8} come closest
	to describing the data however there is some disagreement; in particular no one gives a good
	description of the evolution of this ratio at low $\overline{p_T}$.

	In summary, the best description of the data overall is given by \HEJA and \texttt{POWHEG+PYTHIA8}
	while parton level \HEJ overestimates (underestimate) the gap fraction (the mean number of jets
	in the rapidity gap bounded between the dijet system) and overshoots both the first azimuthal
	moment and the ratio of the second to the first azimuthal moment.  \texttt{POWHEG+HERWIG} describes
	the data poorly for the gap fraction, the mean number of gap jets and the azimuthal decorrelations.
	From this it is clear that while the logarithmically enhanced resummed in the High Energy Jets
	framework are important in regions of phase space where we have large rapidity gaps (such as the
	analyses described here) there are equally important contributions arising from the logarithms
	given to us by the interface with a parton shower.  It is clear that no one package describes all
	of the data presented and therefore the LHC is probing challenging reasons of phase space and thus
	more attention is needed.

	\begin{figure}[bth]
		\centering
		\begin{subfigure}[b]{0.48\textwidth}
			\includegraphics[width=\textwidth, height=1.0\textwidth]{pureJets6a}
			\caption{}
			\label{fig:}
		\end{subfigure}
		~
		\begin{subfigure}[b]{0.48\textwidth}
			\includegraphics[width=\textwidth, height=1.0\textwidth]{pureJets6b}
			\caption{}
			\label{fig:}
		\end{subfigure}
		\caption{The first azimuthal angular moment, $\langle \cos(\pi-\Delta\phi)\rangle$,
		for events passing the veto on gap activity above $Q_0=20\text{GeV}$ as a function
		of (a) the rapidity gap, $\Delta y$, and (b) the average $p_T$, $\overline{p_T}$,
		of the dijet system.}
		\label{fig:atlasPJ5}
	\end{figure}

	\begin{figure}[bth]
		\begin{subfigure}[b]{0.48\textwidth}
			\includegraphics[width=\textwidth, height=1.0\textwidth]{pureJets6c}
			\caption{}
			\label{fig:}
		\end{subfigure}
		~
		\begin{subfigure}[b]{0.48\textwidth}
			\includegraphics[width=\textwidth, height=1.0\textwidth]{pureJets6d}
			\caption{}
			\label{fig:}
		\end{subfigure}
		\caption{The ratio of the second azimuthal angular moment,
		$\langle \cos(2\Delta\phi)\rangle$, to the first azimuthal angular moment,
		$\langle \cos(\pi-\Delta\phi)\rangle$, as a function of (a) the rapidity gap,
		$\Delta y$, and (b) the average $p_T$, $\overline{p_T}$, of the dijet system.  A
		veto of $Q_0=20\text{GeV}$, for (a), and $Q_0=30\text{GeV}$, for (b), is applied
		on activity in the rapidity gap is applied.}
		\label{fig:atlasPJ6}
	\end{figure}

\chapter{$\zg$+Jets at 100TeV}
\label{chap:100TeV}

	Even though the Large Hadron Collider is still in its infancy there is an ever growing effort to discuss
	where we go next as a high energy collider physics community.  A wide range of options have been put forward
	including the linear colliders Compact Linear Collider (CLIC) \cite{Abramowicz:2013tzc} and the International
	Linear Collider (ILC) \cite{BrauJames:2007aa}.  While both of these machines are designed to be precision
	electron-positron linear colliders they have very different designs; CLIC would operate at around a
	centre-of-mass energy of $3$~TeV and use cutting edge accelerating technology whereas the ILC would collide
	at $0.5$~TeV (with a possible upgrade to $1$~TeV).

	However, there are other suggestions on the table.  Of particular interest for this work is the prospect
	of a hadronic Future Circular Collider (FCC-hh).  There are other possible initial states such as
	hadron-lepton or a lepton-lepton being discussed but for obvious reasons it is the FCC-hh which we will
	focus on here.

	One particularly exciting scenario is that of a \htev hadronic collider housed in an extended tunnel
	approximately $100$~km in circumference at the CERN site in Geneva.  Such a machine would make an
	excellent `discovery machine' since it would cover a vast range in partonic centre-of-mass energies.
	The energies probed here would be orders of magnitude higher than ever seen at a hadronic collider
	and so this would be an invaluable test of high scale QCD.  Similarly to physics at the current LHC
	the dominant background would be QCD in nature and so in order for us to be able to extract useful
	information about potential new physics we would need to be able to model this QCD background with
	incredible precision.  Current state-of-the-art for many QCD processes is still limited to next-to-leading
	order in $\alpha_s$ although progress is being made towards improving this to next-to-next-to-leading
	order in some key physics processes, for example Higgs production via gluon fusion is known at N3LO \cite{Anastasiou:2015ema}.
	However, as in the preceding chapters we will instead investigate the effects of the higher-order
	logarithmically enhanced contributions to the perturbative series.  As discussed in chapter \ref{chap:theory}
	these terms are not all captured by NNLO (or any fixed-order scheme $\text{N}^m$LO for that matter).

	The results of chapters \ref{chap:Zs} and \ref{chap:ATLAS} clearly show that these effects are already important
	at a the \stev for both dijets and $\zg$+dijets respectively.  We therefore expect that at a \htev FCC-hh we would
	see a greater effect from the terms enhanced in the High Energy limit beyond a fixed-order only prediction.

	Here we present a study of $\zg$+dijets at a centre-of-mass energy of \htev.  The final state cuts
	are outlined in table~\eqref{tab:atlascuts100}.  For each figure we show the equivalent result calculated
	at \stev with a jet $p_T$ cut of $30$~GeV (which was found to be in excellent agreement with data) as
	well as the \htev predictions for jet cuts of $30$~GeV, $60$~GeV and $100$~GeV.  The jet cut choice
	is an interesting problem since it the best variable for weeding out physics other than the hard
	perturbative scatter.  For example, even at the \stev LHC a QCD study with a jet cut of, say,
	$10$~GeV would be as much a test of our theoretical understanding of perturbative physics as it would
	a test of our descriptions for parton showers, multiple parton interactions and underlying event.
	While this is a perfectly valid analysis to do it is \emph{not} the best choice if our aim is to
	improve our understanding of perturbative QCD.  The same argument applies for a \htev collider
	only more so!  As we go to increasingly higher centre-of-mass energies we may need to raise our
	jet cuts so as to ensure the data we hope to describe is as unpolluted as possible.  Each figure
	also shows the ratio of the \htev prediction to the \stev prediction to emphasise any features
	which may otherwise be hard to see - such as changes in shape.

	\begin{table}[bth]
	  \centering
	  \begin{tabular}{|l|c|}
	    \hline
	    Lepton Cuts & $p_{T\ell}>20$~GeV, \; $|\eta_\ell|<2.5$ \\
	    & $\Delta R^{\ell^+\ell^-} > 0.2$, \; $66$~GeV $\leq m^{\ell^+\ell^-} \leq
	      116$~GeV \\ \hline
	    Jet Cuts (anti-$k_T$, 0.4) & $p_{Tj}>30$~GeV, $60$~GeV, $100$~GeV \\
	    &  $|y_j|<4.4$, \;$\Delta R^{j\ell} >0.5$,  \\
	\hline
	  \end{tabular}
	  \caption{Cuts applied to theory simulations for the \htev
	    $Z$-plus-jets analysis results shown in Figs.~\eqref{fig:100tev_12a}--\eqref{fig:100tev_10b}.}
	  \label{tab:atlascuts100}
	\end{table}

	We begin by discussing what is by far the most uninteresting figure in this thesis (at least at first glance!.
	Fig.~\ref{fig:100tev_12a} shows the differential distribution with respect to the azimuthal separation of the
	two leading jets in $p_T$, $\Delta\phi_{j1, j2}$.  It is clear that although the cross-section of the \htev
	result is significantly greater than that of the \stev result the increase in cross-section is uniform
	throughout the range of $\Delta\phi_{j1, j2}$ - this is clear from the ratio.  What makes the
	uninteresting fig.~\ref{fig:100tev_12a} interesting is that if QCD behaved exactly the same at \htev
	as it did at \stev we would expect all of the plots in this chapter to have a ratio line which
	was a perfectly straight line which merely reflected the increase in cross-section.  However, this
	turns out not to be the case.

	\begin{figure}[bth]
		\centering
		\includegraphics[width=0.7\linewidth]{ATLAS_Z_100TeV_12a}
		\caption{The differential cross-section for $\zg$ plus inclusive dijets as a
		function of the azimuthal separation of the dijet system shown for centre-of-mass
		energies of 7TeV (blue) and 100TeV (pink).}
		\label{fig:100tev_12a}
	\end{figure}

	Fig~\ref{fig:100tev_2a} shows the inclusive $\zg$+dijets cross-section as a function of the number
	of jets, $N_{\text{jet}}$.  Once again we see that the total integrated cross-section grows as we
	go to higher energy but we also see that the relative contribution to the cross-section increases
	as we go to higher jet multiplicity.  This is direct evidence that the convergence of the
	QCD perturbative expansion worsens as we go to higher energy.  Clearly then resummation effects
	such as those described by \hej become more important at a prospective FCC-hh machine and will need
	to be included not only in order to understand the QCD background well enough to extract and
	study new physics but also in order for precision tests of QCD.

	\begin{figure}[bth]
		\centering
		\includegraphics[width=0.7\linewidth]{ATLAS_Z_100TeV_2a}
		\caption{The cross-section for $\zg$ plus inclusive dijets as a function of the number of
		jets, $N_{\text{jet}}$, shown for centre-of-mass energies of 7TeV (blue) and 100TeV (pink).}
		\label{fig:100tev_2a}
	\end{figure}

	Fig~\ref{fig:100tev_11a} shows the differential distribution with respect to the absolute value of the
	rapidity span between the two leading jets in $p_T$, $\Delta y^{j1, j2}$.  We see that as we go to
	large rapidity gaps between the dijets the relative increase in the cross-section grows by almost a
	factor or $10$.  This is precisely

	\begin{figure}[bth]
		\centering
		\includegraphics[width=0.7\linewidth]{ATLAS_Z_100TeV_11a}
		\caption{The differential cross-section for $\zg$ plus inclusive dijets as a function of the absolute value of the
		         rapidity gap between the dijets, $\Delta y^{j1, j2}$ shown for centre-of-mass energies of 7TeV (blue) and
		         100TeV (pink).}
		\label{fig:100tev_11a}
	\end{figure}

	Fig. (\eqref{fig:100tev_11a}) notes:

	\begin{itemize}
		\item dy plot,
		\item O(10) increase in cross-section as we go to large rapidities,
		\item More energy in initial state means we can get more jets further into the outer regions of y-space,
		\item The increase seen is \emph{exactly} the large logs we capture at play
	\end{itemize}

	\begin{figure}[bth]
		\centering
		\includegraphics[width=0.7\linewidth]{ATLAS_Z_100TeV_11b}
		\caption{The differential cross-section for $\zg$ plus inclusive dijets as a function of the invariant mass
		         of the dijets, $m^{jj}$, shown for centre-of-mass energies of 7TeV (blue) and 100TeV (pink).}
		\label{fig:100tev_11b}
	\end{figure}

	Fig. (\eqref{fig:100tev_11b}) notes:

	\begin{itemize}
		\item $dm_jj$ plot,
		\item O(10) increase in cross-section as we go to large invariant masses,
		\item Invariant masses again correlate with the logs we resum (show this explicitly if you havent already),
		\item Similar to fig. (\eqref{fig:100tev_11a})
	\end{itemize}

	\begin{figure}[bth]
		\centering
		\begin{subfigure}[b]{0.48\textwidth}
			\includegraphics[width=\textwidth, height=1.3\textwidth]{ATLAS_Z_100TeV_5b}
			\caption{}
			\label{fig:100tev_5b}
		\end{subfigure}

		\begin{subfigure}[b]{0.48\textwidth}
			\includegraphics[width=\textwidth, height=1.3\textwidth]{ATLAS_Z_100TeV_6a}
			\caption{}
			\label{fig:100tev_6a}
		\end{subfigure}
		~
		\begin{subfigure}[b]{0.48\textwidth}
			\includegraphics[width=\textwidth, height=1.3\textwidth]{ATLAS_Z_100TeV_6b}
			\caption{}
			\label{fig:100tev_6b}
		\end{subfigure}
		\caption{The differential cross-section for $\zg$ plus inclusive dijets as a function of the transverse momentum
		         of the first, second and third leading jets in $p_T$ shown in fig. \eqref{fig:100tev_5b}, \eqref{fig:100tev_6a}
		         and \eqref{fig:100tev_6b} respectively and for centre-of-mass energies of 7TeV (blue) and 100TeV (pink).}
	\end{figure}

	Fig. (\eqref{fig:100tev_5b}-\eqref{fig:100tev_6b}) notes:

	\begin{itemize}
		\item pT distributions,
		\item Heavy tails...soooo?
		\item More energy in initial state means we can get more jets further into the outer regions of y-space,
		\item What effect would a shower have on these distributions?  Plenty of spare pT to radiate.
	\end{itemize}

	\begin{figure}[bth]
		\centering
		\begin{subfigure}[b]{0.48\textwidth}
			\includegraphics[width=\textwidth, height=1.3\textwidth]{ATLAS_Z_100TeV_9b}
			\caption{}
			\label{fig:100tev_9b}
		\end{subfigure}

		\begin{subfigure}[b]{0.48\textwidth}
			\includegraphics[width=\textwidth, height=1.3\textwidth]{ATLAS_Z_100TeV_10a}
			\caption{}
			\label{fig:100tev_10a}
		\end{subfigure}
		~
		\begin{subfigure}[b]{0.48\textwidth}
			\includegraphics[width=\textwidth, height=1.3\textwidth]{ATLAS_Z_100TeV_10b}
			\caption{}
			\label{fig:100tev_10b}
		\end{subfigure}
		\caption{The differential cross-section for $\zg$ plus inclusive dijets as a function of the absolute value of the rapidity
		         of the first, second and third leading jets in rapidity shown in fig. \eqref{fig:100tev_9b}, \eqref{fig:100tev_10a}
		         and \eqref{fig:100tev_10b} respectively and for centre-of-mass energies of 7TeV (blue) and 100TeV (pink).}
	\end{figure}

	Fig. (\eqref{fig:100tev_9b}-\eqref{fig:100tev_10b}) notes:

	\begin{itemize}
		\item Not much more to say about these - mostly covered in dy plots,
	\end{itemize}

