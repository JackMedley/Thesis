\chapter*{Lay Summary}
\noindent

Coloured final states are ubiquitous at hadron colliders such as the Large Hadron Collider (LHC). Therefore understanding high energy perturbative quantum chromodynamics (QCD) at these experiments is essential not only as a test of the Standard Model, but also because these processes form the dominant background to many searches for new physics.  One such `standard candle' is the production of a dilepton pair in association with dijets.  Here we present a new description of this final state (through the production of a $Z^0$ boson and virtual photon).  This calculation adds to the fixed-order accuracy the dominant logarithms in the limit of large partonic centre-of-mass energy to all orders in the strong coupling $\alpha_s$.  This is achieved within the framework of High Energy Jets.  This calculation is made possible by extending the high energy treatment to take into account the multiple t-channel exchanges arising from $Z^0$ and $\gamma^*$ -emissions off several quark lines. The correct description of the interference effects from the various t-channel exchanges requires an extension of the subtraction terms in the all-order calculation.  We describe this construction and compare the resulting predictions to a number of recent analyses of LHC data. The description of a wide range of observables is good, and, as expected, stands out from other approaches in particular in the regions of large dijet invariant mass and large dijet rapidity spans.

In addition we also present the application of the High Energy Jets framework to two new experimental scenarios.  Firstly, we show a comparison of High Energy Jets matched to the ARIADNE parton shower to an ATLAS study of gap activity in dijet events. We see that our description agrees well with the data throughout and in many distributions gives the best theoretical description.  This shows the extra logarithmic corrections are essential to describe data already in LHC Run I.  Secondly, we present a study of $Z^0$/$\gamma^*$ plus dijets at 100~TeV.  We compare the behaviour of the high energy logarithmic enhancements to the QCD perturbative series at 7~TeV and 100~Tev and see that at any high energy hadronic Future Circular Collider (FCC) the effects described by our resummation become significantly more important.


\vspace{10mm}
\normalsize

\chapter*{Abstract}
\noindent

Coloured final states are ubiquitous at hadron colliders such as the Large Hadron Collider (LHC). Therefore understanding high energy perturbative quantum chromodynamics (QCD) at these experiments is essential not only as a test of the Standard Model, but also because these processes form the dominant background to many searches for new physics.  One such `standard candle' is the production of a dilepton pair in association with dijets.  Here we present a new description of this final state (through the production of a $Z^0$ boson and $\gamma^*$).  This calculation adds to the fixed-order accuracy the dominant logarithms in the limit of large partonic centre-of-mass energy to all orders in the strong coupling $\alpha_s$.  This is achieved within the framework of High Energy Jets.  This calculation is made possible by extending the high energy treatment to take into account the multiple t-channel exchanges arising from $Z^0$ and gamma* -emissions off several quark lines. The correct description of the interference effects from the various t-channel exchanges requires an extension of the subtraction terms in the all-order calculation.  We describe this construction and compare the resulting predictions to a number of recent analyses of LHC data. The description of a wide range of observables is good, and, as expected, stands out from other approaches in particular in the regions of large dijet invariant mass and large dijet rapidity spans.

In addition we also present the application of the High Energy Jets framework to two new experimental scenarios.  Firstly, we show a comparison of High Energy Jets matched to the ARIADNE parton shower to an ATLAS study of gap activity in dijet events. We see that our description agrees well with the data throughout and in many distributions gives the best theoretical description.  This shows the extra logarithmic corrections are essential to describe data already in LHC Run I.  Secondly, we present a study of $Z^0$/$\gamma^*$ plus dijets at 100~TeV.  We compare the behaviour of the high energy logarithmic enhancements to the QCD perturbative series at 7~TeV and 100~Tev and see that at any high energy hadronic Future Circular Collider (FCC) the effects described by our resummation become significantly more important.


\vspace{10mm}
\normalsize

