\chapter*{Lay Summary}
\noindent
	At experiments like the Large Hadron Collider (LHC) we seek to explore new areas of particle
	physics by accelerating hadrons (such as protons) into one another.  Because hadrons are
	comprised of more fundamental particles, namely; quarks, anti-quarks and gluons, what we
	see after these collisions is the result of two (or more) of these particles
	scattering off each other.  We use a Relativistic Quantum Field Theory called Quantum
	Chromodynamics (QCD) to describe these encounters.  When using QCD to investigate an
	interaction between any two of these fundamental particles we quickly find that there
	are an infinite number of possible ways in which the collision could have happened (each
	possible way being symbolised by a Feynman Diagram); since we cannot hope to calculate
	all of these we must chose the most important subset of this infinity and calculate those.

	Traditional approaches focussed on selection the subset with the fewest factors of the
	strong coupling constant, known as $\alpha_s$, because $\alpha_s$ is small and therefore
	any diagrams with extra factors will contribute less to the overall sum.  Here we use a
	more subtle technique known broadly as `resummation'.  In this approach we focus not only
	on the number of $\alpha_s$ factors present but also the number of `large logarithms' at
	play.  In this way we find a different subset of this infinity of diagrams which we consider
	to be the most important and instead focus on calculating those.

	Here we present a new calculation for the final state where we have a $Z^0$ boson or a
	high energy photon, $\gamma^*$, decaying to an electron-positron pair in
	association with at least two high energy QCD fundamental particles (which we observes
	experimentally as `jets').  Our resummation captures the `leading' (i.e. the largest)
	logarithms in this process and is further improved by matching our result to the `Leading
	Order' result (the result obtained by the aforementioned traditional techniques).

	We present comparisons of our new theoretical prediction to data gathered at the ATLAS
	and CMS experiments at the LHC and see that it gives good agreement across a wide range of
	observables.  Further we also present two new experimental studies.  Firstly, we show a
	comparison of our prediction matched to an extra `parton shower' resummation to an ATLAS
	study of QCD radiation patterns.  We see that our description agrees well with
	the data throughout.  Secondly, we present a study of $\zg$ plus dijets at \htev (a collision
	energy roughly ten times higher than that used at the LHC).  We compare the behaviour of the
	high energy logarithmic enhancements at \stev and \htev and see that at any high energy
	hadronic Future Circular Collider (FCC) the effects described by our resummation become
	significantly more important.
\vspace{10mm}
\normalsize

\chapter*{Abstract}
\noindent
	QCD final states are ubiquitous at hadron colliders such as the Large Hadron Collider (LHC).
	Therefore understanding high energy perturbative quantum chromodynamics at these experiments
	is essential not only as a test of the Standard Model, but also because many of the dominant
	background to many new physics searches is QCD in nature.  One such `standard candle' is the
	production of a dilepton pair in association with dijets.  Here we present a new description
	of this final state (through $Z^0$ boson and virtual photon).  This calculation adds to the
	fixed-order accuracy the dominant logarithms in the limit of large partonic centre-of-mass
	energy to all orders in the strong coupling $\alpha_s$.  This is achieved within the framework
	of High Energy Jets.

	This calculation is made possible by extending the high energy treatment to take into account
	the multiple $t$-channel exchanges arising from $Z$ and $\gamma^*$-emissions off several
	quark lines. The correct description of the interference effects from the various $t$-channel
	exchanges requires an extension of the subtraction terms in the all-order calculation.  We
	describe this construction and compare the resulting predictions to a number of recent analyses
	of LHC data. The description of a wide range of observables is good, and, as expected, stands
	out from other approaches in particular in the regions of large dijet invariant mass and large
	dijet rapidity spans.

	In addition we also present two new experimental studies.  Firstly, we show a comparison of \hej
	matched to the \texttt{ARIADNE} parton shower to an ATLAS study of gap activity in dijet events,
	this is also compared to several other state-of-the-art next-to-leading order (in $\alpha_s$)
	Monte Carlo generators matched with both with \texttt{PYTHIA} and \texttt{HERWIG} parton shower
	codes.  We see that our description agrees well with the data throughout.  Secondly, we present
	a study of $\zg$ plus dijets at \htev.  We compare the behaviour of the high energy logarithmic
	enhancements to the QCD perturbative series at \stev and \htev and see that at any high energy
	hadronic Future Circular Collider (FCC) the effects described by our resummation become
	significantly more important.

\vspace{10mm}
\normalsize

