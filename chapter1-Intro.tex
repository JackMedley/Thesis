\chapter{Introduction}
\label{chap:intro}

\section{A Little History}
\label{sec:history}

	The current `Standard Model' of particle physics was completed in June 2012 with the announcement of the Higgs boson by the \texttt{ATLAS} collaboration.
	This discovery with the final piece needed to complete the picture developed over the course of the past century beginning with the discovery of the electron in 1897.

	The Standard Model is a gauge quantum field theory describing three of the four observed fundamental forces - with the inclusion of gravity remaining elusive.
	The local gauge structure is given by:

	\begin{equation}
		SU(3)\times SU(2)\times U(1),
		\label{eq:SMGauge}
	\end{equation}

	where the $SU(3)$ describes the strong nuclear force (Quantum Chromodynamics or QCD), while the $SU(2)_L\times U(1)$ describes the electroweak (EW) sector.

	The fundamental particle content can be broken down in to the following three catagories:

	\begin{itemize}
		\item Fermions: These spin-$1/2$ particles obey the spin-statistics theorem and comprise all the known visible matter in the universe.
		Comprising three so-called `generations' fermions can be further subdivided in to quarks and leptons.  Both quarks and leptons
		carry fractional electromagnetic charge but only quarks are charged under the strong $SU(3)$ group.
		\item Gauge bosons: These spin-$1$ excitations arise from the quantisation of the above gauge fields and mediate the three forces detailed above.
		The gauge bosons of the strong and electromagnetic sectors are massless while the electroweak bosons acquire mass through famous Higgs mechanism.
		\item The Higgs boson: The Higgs boson is seen as the result of the spontaneous symmetry breaking of a continuous symmetry.
		This is a gauge-invariant way to give mass to fermions and vector bosons in the standard model which was crutial since such states had long been
		observed at experiments.
	\end{itemize}

\section{Thesis Outline}
\label{sec:outline}

	The aim of this thesis is to detail the importance of a certain class of perurbatibely high-order terms in events with QCD radiation in the final state.  In particular
	corrections to parton-parton collisions with a $\zg$ electroweak boson in the final state will be considered.

	Firstly, in Chapter \ref{chap:theory} I will introduce quantum chromodynamics, the theory of the strong sector in the standard model, and detail how we might use this
	to calculate physical observables (such as cross-sections and differential distributions) at hadron colliders such as the Large Hadron Collider.  I will discuss how
	these observables fall prey to divergences in QCD-like quantum field theoreies with massless states and mention briefly how they may be dealt with.  I will then
	describe how the compuationally expensive integrals derived may be efficiently evaluated using Monte-Carlo coupled with variance reduction techniques such as
	importance sampling.

	In Chapter \ref{chap:HEQCD} the details of QCD in the `High Energy' limit are discussed.  After completing a few instructive calculations we will see how,
	in this limit, the traditional fixed-order perturbative view of scattering cross-sections fades as another subset of terms, namely the Leading
	Logarithmic contributions, become more important.  I will discuss previous work in the High Energy limit of QCD and how this can be used to factorise
	parton-parton scattering amplitudes into convenient `currents' which, when combined with gauge-invariant effective gluon emission terms can be used to
	construct approximate high-multiplicity matrix elements.

	In Chapter \ref{chap:Zs} the work of the previous chapter is extended to the case where there is a massive $Z^0$ boson or an off-shell photon, $\gamma^*$,
	in the final state.  A current for this process is derived and the complexities arising from two separate interferences are explained and dealt with.  I
	compare this new matrix element to the results obtained from a Leading Order (in the strong coupling, $\alpha_s$) generator \texttt{MadGraph} in wide
	regions of phase space and see good agreement.  This result must then be regularised to avoid the divergences discussed in Chapter \ref{chap:theory} and
	this is work is presented.  The procedure for matching this regularised result to well-known Leading Order results is shown and the importance of these
	non-resummation terms is shown.  Lastly three comparisons of the High Energy Jets Z+Jets Monte-Carlo generator to recent experimental studies
	\texttt{ATLAS} and \texttt{CMS} at the LHC are shown.

