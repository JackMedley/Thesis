% !TEX root = thesis.tex
\chapter{Introduction}

	\label{chap:intro}

\section{A Little History}
	\label{sec:history}

	The Standard Model is a gauge quantum field theory describing three of the four observed
	fundamental forces - with the inclusion of gravity remaining elusive.
	Its local gauge structure is given by:

	\begin{equation}
		SU(3)_c\times SU(2)_L\times U(1)_Y.
		\label{eq:SMGauge}
	\end{equation}

	The subscripts on the groups are simply a convenient notation.  The `c' on $SU(3)$ indicates
	that it is the strong `colour' coupling being described.  The `L' on $SU(2)$ indicates that
	all right-handed states are in the trivial representation of the group and the `Y' on the
	$U(1)$ indicates that this is the hypercharge group and not the electromagnetic group. The
	$SU(3)_c$ group describes the strong nuclear force (Quantum Chromodynamics or QCD) and its
	8 gauge generators give us the massless spin-1 gluons, $G_a^\mu(x),\ a=1,\ldots,8$, present
	in the standard model. There are three weak boson states, $W_a^\mu(s),\ a=1,\ldots,3$, associated
	with the $SU(2)_L$ group and a further one, $B^\mu(x)$, which comes from the $U(1)_Y$ group.

	The only remaining boson to complete the standard model arises from the complex scalar Higgs
	field whose ground state is not invariant under the action of $SU(2)_L\times U(1)_Y$. This
	field breaks the standard model gauge symmetry to

	\begin{equation}
		SU(3)_c\times U(1)_{em},
		\label{eq:SMGaugeBroken}
	\end{equation}

	\noindent where the $U(1)_{em}$ refers to the electromagnetic charge.  After this `Spontaneous
	Symmetry Breaking' occurs three of the four aforementioned bosons, $W_a^\mu(s)$ and $B^\mu(x)$
	acquire mass and combinations of them are physically realised as the experimentally observer
	electroweak boson; The massive states $W^\pm, Z^0$ and the massless photon, $\gamma$.
	The photon and the $Z^0$ bosons are of particular importance in the work that follows.

	The fundamental particle content of the Standard Model also includes fermions. These are spin-$1/2$
	particles which obey the spin-statistics theorem (and hence the Pauli exclusion principle) and
	comprise, along with the gluons which binds the nucleus together, all known visible matter in the
	universe. The fermions are structured in three so-called `generations', shown in tab. \ref{tab:fermions}
	and can be further subdivided into quarks and leptons. Quarks are colour triplets under QCD but are
	also charged under the electroweak group.  The up ($u$), charm ($c$) and top ($t$) quarks have electric
	charge $+\frac{2}{3}$ while the down ($d$), strange ($s$) and bottom ($b$) quarks have $-\frac{1}{3}$.
	Leptons are singlets under $SU(3)$ and so do not couple to the strong sector.  The charged leptons
	$e$, $\mu$ and $\tau$ have electric charge $-1$ and the neutrinos are neutral.

	\begin{table}[htp!]
	\begin{center}
	\begin{tabular}{c | c | c | c}
	        & First Generation & Second Generation & Third Generation   \\ \hline
	Quarks  &  $u$, $d$        & $c$, $s$          & $t$,               \\ \hline
	Leptons &  $e$, $\nu_e$    & $\mu$, $\nu_\mu$  & $\tau$, $\nu_\tau$ \\
	\end{tabular}
	\caption{The fermion content of the standard model.}
	\label{tab:fermions}
	\end{center}
	\end{table}

\section{Thesis Outline}
	\label{sec:outline}

	The aim of this thesis is to detail the importance of a certain class of perurbatively higher-order terms
	in events with QCD radiation in the final state.  In particular we will consider corrections to parton-parton
	collisions with a $Z^0$ or $\gamma$ in association with high energy QCD radiation in the final state.

	In chapter \ref{chap:theory} I will begin by introducing quantum chromodynamics, the theory of the strong
	sector in the standard model, and detail how we might use this to calculate physical observables (such as
	cross-sections and differential distributions) at hadron colliders such as the Large Hadron Collider.  I
	will discuss how these observables fall prey to divergences in QCD-like quantum field theories with massless
	states and mention briefly how such divergences can be handled.  I will then describe how the computationally
	expensive integrals derived in subsequent chapters may be efficiently evaluated using Monte-Carlo techniques.

	In chapter \ref{chap:HEQCD} the details of QCD in the `High Energy' limit are discussed.  After completing a
	few instructive calculations we will see how, in this limit, the traditional fixed-order perturbation theory
	view of calculating cross-sections fades as another subset of terms, namely the `Leading Logarithmic' terms in
	$\frac{s}{t}$, become more important.  I will discuss previous work in the High Energy limit of QCD and how
	this can be used to factorise complex parton-parton scattering amplitudes into combinations of `currents' which,
	when combined with gauge-invariant effective gluon emission terms can be used to construct approximate
	high-multiplicity matrix elements.

	In chapter \ref{chap:Zs} the work of the previous chapter is extended to the case where there is a massive
	$Z^0$ boson or an off-shell photon, $\gamma^*$, in the final state.  A `current' for this process is derived
	and the complexities arising from two separate sources of interference are explored.  This new result for the
	matrix element is compared to the results obtained from a Leading Order (in the strong coupling, $\alpha_s$)
	generator \texttt{MadGraph} at the level of the matrix element squared in wide regions of phase space is seen
	to be in exact agreement. This result must then be regularised to treat the divergences discussed in chapter
	\ref{chap:theory} and this process is presented.  The procedure for matching this regularised result to Leading
	Order results is shown and the importance of the inclusion of these non-resummation terms is discussed. Lastly
	three comparisons of the High Energy Jets Z+Jets Monte-Carlo generator to recent experimental studies
	\texttt{ATLAS} and \texttt{CMS} at the LHC are shown.

	From here we use the results of chapter \ref{chap:Zs}, and the resulting publicly available Monte Carlo package,
	to compare our description to a recent experimental prediction of the ratio of the $W^\pm$+jets rate to the
	$\zg$+jets rate.  Our predictions are compared against next-to-leading order (in $\alpha_s$) results from
	\texttt{NJet} and leading order results from \texttt{MadGraph}.

	In chapter \ref{chap:ttbar} we apply the massive spinor-helicity to the production of a $t\overline t$ pair
	in hadronic collisions.  Using the \texttt{PySpinor} package we calculate values for the full-mass matrix
	element and compare them to leading-order (in $\alpha_s$) results from \texttt{MadGraph}.  This is a process
	in which the leading logarithmic contribution starts at one order higher than in previous work and so the effects
	of the resummation are not as expected to be as crucial as in the case of chapter \ref{chap:Zs} - however at
	large values for the centre-of-mass energy (such as that a future high energy circular collider) these
	`next-to-leading' logarithms will once again lead to the breakdown of fixed-order perturbation theory.

	In chapter \ref{chap:ATLAS} we discuss the results of a lengthy study of jet production from the \texttt{ATLAS}
	collaboration.  This analysis was a thorough look at BFKL-like dynamics in proton-proton colliders and the HEJ
	predictions are seen to describe the data well in the regions of phase-space where we know the effects of our
	resummation become relevant.  We compare the predictions from both standalone HEJ and HEJ interfaced with
	\texttt{ARIADNE}, a parton shower based on a dipole-cascade model.  Although the interface to \texttt{ARIADNE}
	increases the computational complexity significantly; we see that the Sudakov logarithms added by significantly
	improve the description of data.

	In chapter \ref{chap:100TeV}, with a study of $\zg$+Jets at a centre-of-mass energy of 100TeV relevant for the
	discussion of the next wave of high energy particle physics experiments (such as any Future Circular Collider)
	which are of great interest to the community at large.  We see that the higher-order perturbative terms are
	much larger at 100TeV relative to 7TeV data and predictions.  Moreover, the regions of phase-space relevant
	for this thesis; that of high energy wide-angle QCD radiation is especially enhanced and, therefore resumming
	these contributions will be essential for precision physics at any `Future Circular Collider'.

	Finally, in chapter \ref{chap:conclusion} I summarise the results of the above chapters and provide a short
	outlook for future work.

