\chapter{High Energy QCD}
\label{chap:HEQCD}

\section{The High Energy Limit of $2\rightarrow2$ QCD scattering}

	\subsection{Mandelstam Variables in the High Energy Limit}
	\label{sub:Mandelstam Variables in the High Energy Limit}

	The $2\rightarrow 2$ QCD scattering amplitudes can be expressed in terms of the well-known Mandestam variables $s$, $t$ and $u$.  Which, in terms of the momenta in the process, are given by:

	\begin{subequations}
		\begin{equation}
			s = (p_1 + p_2)^2
		\end{equation}
		\begin{equation}
			t = (p_1 - p_2)^2
		\end{equation}
		\begin{equation}
			u = (p_2 - p_3)^3
		\end{equation}
	\end{subequations}

	When working in the high energy limit it is convenient to re-express these in terms of the perpendicular momentum of the outgoing partons, $p_\perp$, and the difference in rapidity between the two final state partons, $\delta y$:

	\begin{subequations}
		\begin{equation}
			s = 4p_\perp^2 \cosh^2\frac{\Delta y}{2}
		\end{equation}
		\begin{equation}
			t = -2p_\perp^2 \cosh\frac{\Delta y}{2}e^{-\frac{\Delta y}{2}}
		\end{equation}
		\begin{equation}
			u = -2p_\perp^2 \cosh\frac{\Delta y}{2}e^{\frac{\Delta y}{2}}
		\end{equation}
	\end{subequations}

	In the limit of hard jets well separated in rapidity these can be approximated to give

	\begin{subequations}
		\begin{equation}
			s \approx p_\perp^2 e^{\Delta y}
		\end{equation}
		\begin{equation}
			t \approx -p_\perp^2
		\end{equation}
		\begin{equation}
			u \approx -p_\perp^2 e^{\Delta y}
		\end{equation}
	\end{subequations}

	From equation (above) it is clear that the `hard, wide-angle jet' limit is equivalent to the High Energy limit since:

	\begin{equation}
		\Delta y \approx \ln \left(\frac{s}{-t}\right)
	\end{equation}

	\subsection{HE limit of the three-gluon vertex}
	\label{sub:subsection_name}

	The three gluon vertex shown in figure (X) has the following Feynman rule:

	\begin{equation}
		g_s f^{abc} \left((p_1+p_3)^\nu g^{\mu_1\mu_3} + (q-p_3)^{\mu_1}g^{\mu_3\nu} - (q+p_1)^{\mu_3}g^{\mu_1\nu}\right)
	\end{equation}

	In the high energy limit the emitted gluon with momenta $q$ is much softer that the emitting gluon with momenta $p_1$ i.e. $p_1^\mu \gg q^\mu$  $\forall \mu$ and therefore $p_1\sim p_3$ - using this we can approximate the vertex by

	\begin{equation}
		\approx g_s f^{abc} \left(2p_1^\nu g^{\mu_1\mu_3} + p_3^{\mu_1}g^{\mu_3\nu} - p_3^{\mu_3}g^{\mu_1\nu}\right)
	\end{equation}

	Furthermore, since the hard gluons in a high energy process are external they must satisfy the Ward identities; $\epsilon_1\cdot p_1 = \epsilon_3\cdot p_3 = 0$.  Hence, the vertex can be expressed simply as:

	\begin{equation}
		\approx 2g_s f^{abc}p_1^\nu g^{\mu_1\mu_3}
	\end{equation}

	\subsection{At Leading Order in $\alpha_s$}
	\label{sub:HE22_LO}

	Talk through the limit of $2\rightarrow2$ scattering of gluons.  Introduce mandelstam variables, show the equivalence of large delta y and large s.

	\subsection{At Next-to-Leading Order in $\alpha_s$}
	\label{sub:HE22_NLO}

	Calculate the NLO calcuations to the 2j ME and show that there explicitly is a delta y (large log) enhancement.

	\subsection{High Energy Jets `Currents'}
	\label{sub:currents}

	\subsection{Effective Vertices For Real Emissions}
	\label{sub:effective_vertices_for_real_emissions}

\section{High Energy Jets}
\label{sec:section_name}

	\subsection{The Multi-Regge Kinematic limit of QCD amplitudes}
	\label{sub:subsection_name}

	\subsection{Logarithms in HEJ observables}
	\label{sub:subsection_name}

		Here you should take a $2\rightarrow n$ ME, apply the HE limit to it, do a PS integration and show the logs you get.  Need the HE limit of PS integral from JA thesis and/or from VDD talk

	\subsection{HEJ currents}
	\label{sub:currents}

	\subsection{High Energy Phase-space Integration}
	\label{sub:HEPhaseSpace}

