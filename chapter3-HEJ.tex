\chapter{High Energy QCD}
\label{chap:HEQCD}

	{\color{red}
	Stuff for this section
	\begin{itemize}
		\item Tiny bit about cuts S matrix to Optical theorem,
		\item NLO calculation of leading part of gg->gg (explain sub-leading bits) - do this bit using cuts,
		\item Extra real corrections and writing these as contractions of currents,
		\item A bit on HE phase-space integrals?
	\end{itemize}
	}

	In this chapter we look in detail at the `High Energy' limit of QCD.  We begin by defining this limit and
	looking at how basic $2\rightarrow2$ scattering behaves at leading order and next-to-leading order in
	$\alpha_s$.

	\section{The `High Energy' limit}
		\label{sub:HElimit}

		The `High Energy' limit of QCD, also referred to as the Multi-Regge Kinematic (MRK) limit is
		defined in terms of the kinematics of the final state.  We require a \emph{strong rapidity ordering}
		of all outgoing radiation as well as all the emissions having \emph{similar transverse momenta}.
		Mathematically this is:

		\begin{equation}
			y_1\gg y_2\gg\cdots\gg\\y_n \text{ and } |p_{\perp1}| \approx |p_{\perp2}| \approx\cdots\approx|p_{\perp(n-1)}|,
			\label{eqn:MRK}
		\end{equation}

  		where we define the rapidity of a final states particle as

		\begin{equation}
			y = \half\frac{E+p_z}{E-p_z}
			\label{eqn:rap}
		\end{equation}

		where $E$ is the energy of particle and $p_z$ it the $z$ component of its momentum. We can
		state the of the criteria in eq. \eqref{eqn:MRK} equivalently as $s_{ij}\rightarrow\infty$ where
		$s_{ij} = (p_i + p_j)^2$.  We sometimes instead use the pseudo-rapidity, $\eta$, which is
		simply related to the angle of the outgoing state to the beam, $\theta$:

		\begin{equation}
			\eta = -\ln\tan\frac{\theta}{2}.
			\label{eqn:prap}
		\end{equation}

		For massless states eqs. \eqref{eqn:rap} and \eqref{eqn:prap} are equivalent.

	\section{Mandelstam Variables in the High Energy Limit}
		\label{sub:MandelstamVariables}

		The $2\rightarrow 2$ QCD scattering amplitudes can be expressed in terms of the well-known Mandestam
		variables $s$, $t$ and $u$.  Which, in terms of the momenta in the process, are given by:

		\begin{subequations}
			\begin{equation}
				s = (p_1 + p_2)^2
			\end{equation}
			\begin{equation}
				t = (p_1 - p_2)^2
			\end{equation}
			\begin{equation}
				u = (p_2 - p_3)^3
			\end{equation}
			\label{eqn:mandel}
		\end{subequations}

		When working in the high energy limit it is convenient to re-express these in terms of the
		perpendicular momentum of the outgoing partons, $p_\perp$, and the difference in rapidity
		between the two final state partons, $\Delta y$.  If we parametrise our outgoing states as

		\begin{align}
		\begin{split}
			p_1 = p_{\perp1}\big(\cosh (y_1), \cos(\phi_1), \sin(\phi_1), \sinh (y_1)\big),\\
			p_2 = p_{\perp2}\big(\cosh (y_2), \cos(\phi_2), \sin(\phi_2), \sinh (y_2)\big),
		\end{split}
		\end{align}

		then we can express eqs. \eqref{eqn:mandel} as follows

		\begin{subequations}
			\begin{equation}
				s = 4p_\perp^2 \cosh^2\frac{\Delta y}{2},
			\end{equation}
			\begin{equation}
				t = -2p_\perp^2 \cosh\frac{\Delta y}{2}e^{-\frac{\Delta y}{2}},
			\end{equation}
			\begin{equation}
				u = -2p_\perp^2 \cosh\frac{\Delta y}{2}e^{\frac{\Delta y}{2}}.
			\end{equation}
		\end{subequations}

		In the limit of hard jets well separated in rapidity, i.e. $\Delta y\rightarrow\infty$,
		these are well approximated by

		\begin{subequations}
			\begin{equation}
				s = p_\perp^2 e^{\Delta y}
			\end{equation}
			\begin{equation}
				t = -p_\perp^2
			\end{equation}
			\begin{equation}
				u = -p_\perp^2 e^{\Delta y}
			\end{equation}
			\label{eqn:mandel2}
		\end{subequations}

		From this it is clear that the `hard, wide-angle jet' limit is equivalent to the High Energy
		limit since as $\Delta y$ grows large $s$ will grow exponentially while $t$ will stay fixed.
		Rearranging for $\Delta y$ in the above equations yields:

		\begin{equation}
			\Delta y = \ln \left(\frac{s}{-t}\right).
			\label{eqn:largeLogs}
		\end{equation}

		This is a useful result because it relates the simple kinematics of an event to a (potentially)
		large logarithm.  It is already naively clear from eq. \eqref{eqn:largeLogs} that a final state
		with large rapidity gaps could require a more careful inspection than the fixed-order approach
		discussed in section \ref{sec:pqcdAndResum}.

	\section{$qQ$-scattering at High Energy (at LO)}
		\label{sec:qQScat}

		Here we begin with the simplest example; the case of $qQ\rightarrow qQ$ for all negative helicity partons
		\footnote{where the capital $Q$ implies it is a different flavour to $q$ - and hence we need not consider the crossed
		diagrams which would contribute if they were the same flavour}. There is only one diagram which contributes shown
		in fig. (\ref{fig:TwoToTwo}).  Using the Feynman rules detailed in section \ref{sec:partonicCrossSection} we can
		write the matrix element as:

		\begin{align}
			i\mathcal{M}_{q^-Q^-\rightarrow q^-Q^-}^{\text{LO}} &= ig_s^2T^d_{1a}T^d_{2b}\frac{\overline{u}^-(p_1)\gamma^\mu
			  u^-(p_a)\overline{u}^-(p_2)\gamma_\mu u^-(p_b)}{t}\\
			  &= ig_s^2T^d_{1a}T^d_{2b}\frac{\bk{1}{\mu}{a}\cdot\bk{2}{\mu}{b}}{t},
			  \label{eqn:similarBrackets}
		\end{align}

		where $t = (p_a - p_1)^2$ and we have used the shorthand $\overline{u}^-(p_i)\gamma^\mu u^-(p_j) = \bk{i}{\mu}{j}$ in the second line.
		Writing the contraction of these two `current' terms in terms of light-cone coordinates we have:

		\begin{equation}
			i\mathcal{M}_{q^-Q^-\rightarrow q^-Q^-}^{\text{LO}} = ig_s^2T^d_{1a}T^d_{2b}\frac{2\sqrt{p_a^-p_b^+}}{t}
			\left(\sqrt{p_1^+p_2^-}e^{i\phi_2} + \sqrt{p_1^-p_2^+}e^{i\phi_1}\right),
			\label{eqn:qQ2qQ}
		\end{equation}

		where $e^{i\phi_i} = \frac{p_{\perp i}}{|p_{\perp i}|}$.  We now approximate the kinematics in such a way that we may write
		eq. \eqref{eqn:qQ2qQ} in a `factorised' form once again.  Specifically we consider that the scattering can be thought
		of as two partons glancing off one another.  That is, we assume that $p_1^+\ll p_1^-$ and $p_2^-\ll p_2^+$.  We can further
		assume that $p_1^-\approx p_a^-$ and $p_2^-\approx p_b^-$ and with this we see that \eqref{eqn:qQ2qQ} becomes:

		\begin{equation}
			i\mathcal{M}_{q^-Q^-\rightarrow q^-Q^-}^{\text{LO}} = \frac{2s}{t}\left(g_sT^d_{1a}e^{i\phi_1}\right)\left(-ig_sT^d_{2b}\right),
			\label{eqn:reggeTraj}
		\end{equation}

		which is `factorised' in the sense that each scalar term in brackets depends only on one quark line; either on the $p_{a/1}$
		line or the $p_{b/2}$ line.  We see that the amplitude for $qQ\rightarrow qQ$ is dominated by the $s$ kinematic variable.
		We can express this as:

		\begin{equation}
			\mathcal{M}_{q^-Q^-\rightarrow q^-Q^-}^{\text{LO}} \sim s^{\alpha(t)},
		\end{equation}

		which is exactly the behaviour expected when a particle exchanged in the $t$-channel has `reggeised'
		\cite{sabioThesis,DelDuca:1995hf,lipatovBook}.  $\alpha(t)$ is the Regge trajectory and is equal to
		the intrinsic spin of the state exchanged.  In our example we have have a spin-one gluon exchanged
		and accordingly we can see from eq. \eqref{eqn:reggeTraj} that $\alpha(t)=2$ for $qQ\rightarrow qQ$.

		\begin{figure}
			\begin{center}
			\includegraphics[width=0.35\linewidth]{TwoToTwo}
			\caption{The only diagram which contributes to $qQ\rightarrow qQ$ at leading order in $\alpha_s$.}
			\label{fig:TwoToTwo}
			\end{center}
		\end{figure}

	\section{$qg$ scattering at High Energy}

		We now with explore the more involved case of $2\rightarrow 2$ quark-gluon scattering.  At leading order this
		consists of three diagrams shown in fig. (\ref{fig:TwoToTwo}).  Here we only show the calculations
		for the helicity structure where both quark lines have fixed, and opposite, helicities.  We use
		the following gauge choice for the gluon polarisations:

		\begin{align}
		\epsilon^{+*}_{2\sigma}&=\frac{\langle b|\sigma|2\rangle}{\sqrt{2}\langle b2\rangle} & \epsilon^{-*}_{2\sigma} &= -\frac{\langle b|\sigma|2\rangle}{\sqrt{2}[b2]} \\
		\epsilon^{+}_{b\sigma}&=-\frac{\langle b|\sigma|2\rangle}{\sqrt{2}[2b]} & \epsilon^{-*}_{2\sigma} &= -\frac{\langle b|\sigma|2\rangle}{\sqrt{2}\langle 2b\rangle}
		\end{align}

		For simplicity we alter the notation slightly and choose to treat everything as having negative helicity.
		To describe positive helicities we can use the transposition property of spinor-helicity brackets discussed
		from section \ref{sec:SpinorHelicity}.

		\begin{figure}[h]
			\centering
			\begin{subfigure}[b]{0.3\textwidth}
				\includegraphics[width=\textwidth]{qg2qg-s}
				\caption{}
				\label{fig:qg2qg-s}
			\end{subfigure}

			\begin{subfigure}[b]{0.3\textwidth}
				\includegraphics[width=\textwidth]{qg2qg-t}
				\caption{}
				\label{fig:qg2qg-t}
			\end{subfigure}
			~
			\begin{subfigure}[b]{0.3\textwidth}
				\includegraphics[width=\textwidth]{qg2qg-u}
				\caption{}
				\label{fig:qg2qg-u}
			\end{subfigure}
			\caption{The $s$, $t$ and $u$ channel diagrams contributing to $qg\rightarrow qg$ at leading
			         order in $\alpha_s$ in figures (\ref{fig:qg2qg-s}), (\ref{fig:qg2qg-t}) and (\ref{fig:qg2qg-u})
			         respectively.}
		\end{figure}

		\subsection{$s$-channel}

			The matrix element for the $s$-diagram, shown in fig. \eqref{fig:qg2qg-s}, is:

			\begin{align}
				\mathcal{M}_{qg\rightarrow qg, s}^{\text{LO}}=&\overline{u}^-(p_1)\left(-\frac{ig_s}{2}\gamma^\mu\right)\epsilon_\mu^{*+}(p_2)
					\frac{i(\slashed q+m)}{q^2-m^2}\left(-\frac{ig_s}{2}\gamma^\nu\right)\epsilon^+_\nu(p_b)u^-(p_a), \\
				=&-\frac{g^2_s}{4q^2}\epsilon^{*+}_{2\mu}\epsilon^+_{b\nu}\overline{u}^-_1\gamma^\mu\slashed q\gamma^\nu u^-_a,
			\end{align}

			where we have used $q\gg mc$ for the high energy case.  The propagator has momentum $q=p_a+p_b=p_1+p_2$ and therefore:

			\begin{align}
				\mathcal{M}_{qg\rightarrow qg, s}^{\text{LO}}=&-\frac{g^2_s}{4q^2}\frac{\langle{b}|\mu|2\rangle}{\sqrt{2}\langle{b2\rangle}}
				\frac{\langle{b}|\nu|2\rangle}{\sqrt{2}[2b]}\overline{u}^-_1\gamma^\mu(\slashed{p}_a+\slashed{p}_b)\gamma^\nu u^-_a,\\
				=&-\frac{g^2_s}{8q^2s_{2b}}\langle{b}|\mu|2\rangle\langle{b}|\nu|2\rangle
				\left(\overline{u}^-_1\gamma^\mu\gamma^\sigma\gamma^\nu u^-_ap_{a\sigma}+
				\overline{u}^-_1\gamma^\mu\gamma^\sigma\gamma^\nu u^-_ap_{b\sigma}\right).
			\end{align}

			The gamma matrices satisfy the Clifford algebra, $\{\gamma^\mu, \gamma^\nu\}=2g^{\mu\nu}$ and so we may write:

			\begin{equation}
				\overline{u}^-_1\gamma^\mu\gamma^\sigma\gamma^\nu u^-_ap_{a\sigma}=
				\overline{u}^-_1\gamma^\mu\gamma^\nu\gamma^\sigma u^-_ap_{a\sigma} -
				2\overline{u}^-_1\gamma^\mu g^{\sigma\nu}u^-_ap_{a\sigma}
			\end{equation}

			but in the High Energy limit the Dirac equation is $\slashed{p}_au^-_a=0$:

			\begin{equation}
				\mathcal{M}_{qg\rightarrow qg, s}^{\text{LO}}=-\frac{g^2_s}{8q^2s_{2b}}\langle{b}|\mu|2\rangle\langle{b}|\nu|2\rangle
				\left(-2\overline{u}^-_1\gamma^\mu u^-_ap_{a}^\sigma+\overline{u}^-_1\gamma^\mu\gamma^\sigma\gamma^\nu u^-_ap_{b\sigma}\right)
			\end{equation}

			For the second term we must use the following identity:

			\begin{equation}
				\gamma^\mu\gamma^\sigma\gamma^\mu=g^{\mu\sigma}\gamma^\nu + g^{\sigma\nu}\gamma^\mu
				- g^{\mu\nu}\gamma^\sigma - i\epsilon^{\rho\mu\sigma\nu}\gamma_\rho\gamma^5,
			\end{equation}

			where $\epsilon^{\rho\mu\sigma\nu}$ is the four dimensional totally antisymmetric symbol:

			\begin{align}
			\begin{split}
				\mathcal{M}_{qg\rightarrow qg, s}^{\text{LO}} = &-\frac{g^2_s}{8q^2s_{2b}} \bk{b}{\mu}{2} \bk{b}{\nu}{2} \Big(\bk{1}{\mu}{a} p_a^\nu \\
				&+ p_{b\sigma}\overline{u}^-_1(g^{\mu\sigma}\gamma^\nu + g^{\sigma\nu}\gamma^\mu
				- g^{\mu\nu}\gamma^\sigma - i\epsilon^{\rho\mu\sigma\nu}\gamma_\rho\gamma^5)\Big)\\
				= &-\frac{g^2_s}{8q^2s_{2b}}\langle{b}|\mu|2\rangle\langle{b}|\nu|2\rangle
				\Big(\langle 1|\mu|a\rangle \langle a|\nu|a\rangle + \langle b|\mu|b\rangle
				\langle 1|\nu|a\rangle \\
				&+ \langle b|\sigma|b\rangle\langle1|\sigma|a\rangle
				g^{\mu\nu} - i\langle b|\sigma|b\rangle\langle 1|\epsilon^{\rho\mu\sigma\nu}\gamma_\rho\gamma^5|a\rangle\Big).
			\end{split}
			\end{align}

			The second, third and fourth terms are zero because, for example:

			\begin{equation}
				\langle b|\mu|2\rangle\langle b|\mu|b\rangle = 2[2b]\langle b b\rangle = 0,
			\end{equation}

			and therefore

			\begin{equation}
				\mathcal{M}_{qg\rightarrow qg, s}^{\text{LO}}=-\frac{g^2_s}{4q^2s_{2b}}[2a]\langle ab\rangle\langle{b}|\mu|2\rangle\langle{1}|\mu|a\rangle
			\end{equation}

			Using $q^2=s_{ab}=\langle ab\rangle[ba]$ and $s_{2b}=\langle2b\rangle[b2]$ we have:

			\begin{equation}
			\mathcal{M}_{qg\rightarrow qg, s}^{\text{LO}}=-\frac{g^2_s}{4}\frac{[2a]\langle ab\rangle}{\langle ab\rangle[ba]
			\langle2b\rangle[b2]}\langle{b}|\mu|2\rangle\langle{1}|\mu|a\rangle.
			\end{equation}

			Now we must calculate the spinor products.  We use the conventions for spinors outlined in the
			previous chapter.  For example:

			\begin{align}
				[2a] = &\overline{u}^+_2u^-_a=(u^+_2)^\dagger\gamma^0u^-_a=\left(\sqrt{p^+_2},
				\sqrt{p^-_2}\frac{p^{\perp}_2}{|p_2^\perp|}, 0, 0\right)
				\gamma^0\left( \begin {array}{c} 0\\ \noalign{\medskip}0\\ \noalign{\medskip}0\\ \noalign{\medskip}-\sqrt{p^+_a}\end {array}\right)\\
				[2a] = &\left(\sqrt{p^+_2}, \sqrt{p^-_2}\frac{p^{\perp}_2}{|p_2^\perp|}, 0, 0\right)\left( \begin {array}{c} 0\\ \noalign{\medskip}-\sqrt{p^+_a}\\
				\noalign{\medskip}0\\ \noalign{\medskip}0\end {array}\right)=-\frac{\sqrt{p_a^+p_2^-}p_2^\perp}{|p_2^\perp|}
			\end{align}

			And upon calculating the other brackets we see:

			\begin{equation}
				\mathcal{M}_{qg\rightarrow qg, s}^{\text{LO}} = -\frac{g_s^2}{4}\sqrt{\frac{p_2^-}{p_b^-}}\frac{1}{p_2^+p_b^-} \frac{p_{2\perp}^*}{|p_{2\perp}|} \bk{b}{\mu}{2} \bk{1}{\mu}{a}
			\end{equation}

			Which can be simplified slightly since $\hat{t}=s_{2b}$ to give the final result:

			\begin{equation}
				\mathcal{M}_{qg\rightarrow qg, s}^{\text{LO}}=-\frac{g_s^2}{2\hat{t}}\sqrt{\frac{p_2^-}{p_b^-}}\frac{p_{2\perp}^*}
				{|p_{2\perp}|}\langle{b}|\mu|2\rangle\langle{1}|\mu|a\rangle
				\label{eqn:s-channel}
			\end{equation}

		\subsection{$t$-channel}

			The matrix element for the $t$-diagram, shown in fig. \eqref{fig:qg2qg-t}, is:

			\begin{align}
			\begin{split}
				-i\mathcal{M}_{qg\rightarrow qg, t}^{\text{LO}} =&-\overline{u}^-_1\Bigg(-\frac{ig_s}{2}\gamma^\mu\Bigg)\Bigg(-\frac{ig_{\mu\nu}}{q^2}\Bigg)u^-_ag_sf^{\gamma\beta\delta}\\
				&\Big(g_{\sigma\nu}(p_b-q)_\rho + g_{\nu\rho}(p_b-q)_\sigma - g_{\rho\sigma}(p_b-q)_\nu)\Big)\epsilon^{\rho *}_{2+}\epsilon^{\sigma}_{b+}\\
				=&-\frac{g_s^2}{2q^2s_{2b}}\left(\overline{u}^-_1\gamma^{\nu}u^-_a\right)\left(\overline{u}^-_b\gamma^{\rho}u^-_2\right)\\
				&\left(\overline{u}^-_b\gamma^{\sigma}u^-_2\right)\left(g_{\sigma\nu}(p_b-q)_\rho + g_{\nu\rho}(p_b-q)_\sigma - g_{\rho\sigma}(p_b-q)_\nu)\right)
			\end{split}
			\end{align}

			Now using $q=p_a-p_1=p_2-p_b$:

			\begin{align*}
			i\mathcal{M}_{qg\rightarrow qg, t}^{\text{LO}} = -\frac{g_s^2}{2q^2s_{2b}}[(2p_{2\rho}-p_{b\rho})&\left(\overline{u}^-_1\gamma_{\sigma}u^-_a\right)\left(\overline{u}^-_b\gamma^{\rho}u^-_2\right)\left(\overline{u}^-_b\gamma^{\sigma}u^-_2\right)+\ldots \\
			\ldots + \hspace{2pt} (2p_{2\sigma}-p_{b\sigma})&\left(\overline{u}^-_1\gamma_{\nu}u^-_a\right)\left(\overline{u}^-_b\gamma^{\nu}u^-_2\right)\left(\overline{u}^-_b\gamma^{\sigma}u^-_2\right)-\ldots \\
			\ldots - \hspace{2pt} (p_{2\nu}\hspace{4pt}+p_{b\nu})&\left(\overline{u}^-_1\gamma^{\nu}u^-_a\right)\left(\overline{u}^-_b\gamma_{\sigma}u^-_2\right)\left(\overline{u}^-_b\gamma^{\sigma}u^-_2\right)]
			\end{align*}

			Once again in the high energy case the Dirac equation is $\slashed pu^\pm=0$ and $\overline{u}^\pm\slashed p=0$.  The first line of the above expression reads:

			\begin{equation}
			2\langle1|\sigma|a\rangle\langle b|\sigma|2\rangle\overline{u}^-_b\slashed p_2u^-_2 - \langle1|\sigma|a\rangle\langle b|\sigma|2\rangle\overline{u}^-_b\slashed p_b,
			\end{equation}

			which cancels completely.  The other two lines contain similar factors and therefore

			\begin{equation}
				\mathcal{M}_{qg\rightarrow qg, t}^{\text{LO}}=0.
				\label{eqn:t-channel}
			\end{equation}

		\subsection{$u$-channel}

			The matrix element for the $u$-diagram, shown in fig. \eqref{fig:qg2qg-t}, is:

			\begin{equation}
			\begin{split}
			-i\mathcal{A}_u &= \overline{u}^-(p_1)\left(-\frac{ig_s}{2}\gamma^\mu\right)\frac{i(\slashed q+mc)}{q^2-m^2c^2}\left(-\frac{ig_s}{2}\gamma^\nu\right)u^-(p_a)\epsilon_\mu^{*+}(p_b)\epsilon^+_\nu(p_2)\\
			\mathcal{A}_u &= \frac{g_s^2}{4q^2}\overline{u}^-_1\gamma^\mu\slashed q\gamma^\nu u^-_a\epsilon^{+*}_{b\mu}\epsilon^*_{2\nu}\\
			&= \frac{g_s^2}{8q^2s_{2b}}\langle b|\mu|2\rangle\langle b|\nu|1\rangle\overline{u}^-_1\gamma^\mu(\slashed{p}_a-\slashed{p}_2)\gamma^\nu u^-_a\\
			&= \frac{g_s^2}{8q^2s_{2b}}\langle b|\mu|2\rangle\langle b|\nu|1\rangle(\overline{u}^-_1\gamma^\mu\gamma^\sigma\gamma^\nu u^-_ap_{a\sigma} - \overline{u}^-_1\gamma^\mu\gamma^\sigma \gamma^\nu u^-_ap_{2\sigma})
			\end{split}
			\end{equation}

			Where we have used $q=p_a-p_2$.  By direct comparison with the procedure used for the $s$-channel we can see the result will be:

			\begin{equation}
			\mathcal{A}_u=\frac{g_s^2}{2\hat{t}}\sqrt{\frac{p_b^-}{p_2^-}}\frac{p_2^{\perp^*}}{|p_2^\perp|}\langle{b}|\mu|2\rangle\langle{1}|\mu|a\rangle.
			\label{eqn:u-channel}
			\end{equation}

			The total total matrix element is given by the sum of eqs. \eqref{eqn:s-channel}, \eqref{eqn:t-channel} and \eqref{eqn:u-channel} which is:

			\begin{equation}
				\mathcal{A}=\frac{g_s^2}{2}\frac{p_2^{\perp^*}}{|p_2^\perp|}\left(\sqrt{\frac{p_b^-}{p_2^-}}-\sqrt{\frac{p_2^-}{p_b^-}}\right)\frac{\langle{b}|\mu|2\rangle\langle{1}|\mu|a\rangle}{\hat{t}},
				\label{eqn:fullsum}
			\end{equation}

			Which is exactly in the form of two `currents' contracted as seen in section \ref{sec:qQScat}.  We also
			see that eqn. \eqref{eqn:fullsum} has the same spinor-helicity brackets contracted as eqn. \eqref{eqn:similarBrackets}
			and so the dominant behaviour of $gs\rightarrow qg$ in the high energy limit is $\frac{s}{t}$.\\In the High Energy
			limit we have $p_b^-\thicksim p_2^-$ and so eqn. \eqref{eqn:fullsum} could be simplified further; though we
			actually choose \emph{not} to do this so as to approximate as little as possible.  The important thing about eqn.
			\eqref{eqn:fullsum} is that this helicity structure can be described exactly at the High Energy limit by the
			exchange of a soft $t$-channel gluon.\\  All of the other helicity combinations can be calculated too we see
			they also have a pole in the $\hat t$ channel.

	\section{$qQ$-scattering at High Energy (at NLO)}
		\label{sub:HE22_NLO}

		Before we continue on to look at how we might construct high multiplicity scattering matrix
		elements in terms of `currents' and effective vertices we briefly look at higher order (in
		$\alpha_s$) corrections to the process we studied in section \ref{sec:qQScat}.
		So far we have seen the leading order diagrams with a $t$-channel exchange are enhanced but in
		eqn. \eqref{eqn:schematicExpn} we sketched out a form for the perturbative expansion which also
		had enhanced higher order (in $\alpha_s$) terms.  We might naively expect that the next-to-leading
		order diagrams with the most $t$-channel exchanges will give the greatest enhancement and, indeed,
		this turns out to be the case.  These diagrams are shown for the case of $qQ\rightarrow qQ$ in fig.
		\eqref{fig:NLO-leadingContrib.}. The full calculation can be found in \cite{DelDuca:1995hf} (for
		the case of $gg\rightarrow gg$) and \cite{sabioThesis} (for the case of $qQ\rightarrow qQ$).  These
		diagrams can be elegantly computed by employing the `Cutkosky rules' which are used to relate two
		sub-diagrams to the imaginary part of a higher order diagram thought the Optical theorem.  Pictorially
		we `cut' propagators by forcing them on-shell with delta functions and inserting a complete set of
		states.  E.g. the uncrossed amplitude in fig. \eqref{fig:NLO-uncrossed},
		$\mathcal{M}_{qQ\rightarrow qQ}^{\text{NLO, II}}$, may be expressed as a combination of two copies
		of the amplitude arising from fig. \eqref{fig:TwoToTwo}:

		\begin{equation}
			\mathcal{M}_{q^-Q^-\rightarrow q^-Q^-}^{\text{LO}} = -g_s^2T^d_{1a}T^d_{2b}\frac{\bk{1}{\mu}{a}\cdot\bk{2}{\mu}{b}}{t},
		\end{equation}

		as follows:

		\begin{align}
			\text{Im}\mathcal{M}_{qQ\rightarrow qQ}^{\text{NLO, II}} = \frac{1}{2(2\pi)^2}\int &d^4k\delta((p_a-k)^2)
			\delta((p_b+k)^2)\\ &\mathcal{M}_{q^-Q^-\rightarrow q^-Q^-}^{\text{LO}}(k)
			\mathcal{M}_{q^-Q^-\rightarrow q^-Q^-}^{\dagger\text{LO}}(k-q),
		\end{align}

		where $\text{Im}(\cdot)$ denotes the imaginary part, $k$ is the loop momentum, $q$ is the momentum
		transfer and $\dagger$ denotes Hermitian conjugation.  In the High Energy limit we can perform the
		integration to give

		\begin{equation}
			\text{Im}\mathcal{M}_{qQ\rightarrow qQ}^{\text{NLO, II}} = 4\alpha_s^2 s\mathcal{C}_1(T^a,T^b)
			\int \frac{d\vec{k}}{\vec{k}(\vec{k} - \vec{q})},
		\end{equation}

		where $\mathcal{C}_1(T^a,T^b)$ is the colour factor for the diagram and $\vec{k}$ is the transverse
		component of $k$.  We can now relate the imaginary part of the amplitude to the full amplitude by
		conjecturing that the amplitude will be logarithmically enhanced as follows

		\begin{align}
			\mathcal{M}_{qQ\rightarrow qQ}^{\text{NLO, II}} = &\text{Re}\mathcal{M}_{qQ\rightarrow qQ}^{\text{NLO, II}} +
			i\text{Im}\mathcal{M}_{qQ\rightarrow qQ}^{\text{NLO, II}}\\
			=&\widetilde{\mathcal{M}}_{qQ\rightarrow qQ}^{\text{NLO, II}}\ln\frac{s}{t} + \text{sub-leading}\\
			=&\widetilde{\mathcal{M}}_{qQ\rightarrow qQ}^{\text{NLO, II}}\left(\ln\left|\frac{s}{t}\right| -i\pi\right),
		\end{align}

		where we have used that $\frac{s}{t} < 0$.  Comparing real and imaginary parts we see that

		\begin{equation}
			\text{Re}\mathcal{M}_{qQ\rightarrow qQ}^{\text{NLO, II}} = -\frac{1}{\pi}\text{Im}\mathcal{M}_{qQ\rightarrow qQ}^{\text{NLO, II}}
		\end{equation}

		and we can reconstruct the real part of the amplitude as:

		\begin{equation}
			\text{Re}\mathcal{M}_{qQ\rightarrow qQ}^{\text{NLO, II}} = -\frac{4\alpha_s^2u}{\pi} \mathcal{C}_1(T^a,T^b)
			\int \frac{d\vec{k}}{\vec{k}(\vec{k} - \vec{q})}\ln\left|\frac{u}{t}\right|.
			\label{eqn:uncrossedNLOcontrib}
		\end{equation}

		The crossed-diagram, \eqref{fig:NLO-crossed}, also contributes a leading logarithmic piece and is related to
		eqn. \eqref{fig:NLO-uncrossed} by a crossing symmetry and so we simply replace $s$ with $u$ in eqn.
		\eqref{eqn:uncrossedNLOcontrib} and calculate a new colour factor, $\mathcal{C}_2(T^a,T^b)$:

		\begin{equation}
			\text{Re}\mathcal{M}_{qQ\rightarrow qQ}^{\text{NLO, X}} = -\frac{4\alpha_s^2s}{\pi} \mathcal{C}_2(T^a,T^b)
			\int \frac{d\vec{k}}{\vec{k}(\vec{k} - \vec{q})}\ln\left|\frac{s}{t}\right|.
			\label{eqn:crossedNLOcontrib}
		\end{equation}

		But in the high energy limit $s\sim -u$ (this is clear from eqs. \eqref{eqn:mandel2}) and so we
		can combine these terms and express the leading logarithmic NLO term in terms of the leading order
		result:

		\begin{equation}
			\mathcal{M}_{qQ\rightarrow qQ}^{\text{NLO}} = \frac{3\alpha_ss}{\pi^2}
			\hat{\alpha}(q)\ln\left|\frac{s}{t}\right|
			\mathcal{M}_{qQ\rightarrow qQ}^{\text{LO}},
			\label{eqn:enhancedNLO}
		\end{equation}

		where:

		\begin{equation}
			\hat{\alpha}(q) = \int d\vec{k}\frac{\vec{q}^2}{\vec{k}(\vec{k} - \vec{q})}
		\end{equation}

		From eqn. \ref{eqn:enhancedNLO} we can see the logarithmic enhancement explicitly; there is still
		a suppression from the inclusion of an extra factor of $\alpha_s$ with respect to the leading
		order term but as we have seen previously the logarithm is related to the kinematics of the final
		state - namely - the rapidity gap between the outgoing quarks $p_1$ and $p_2$.

		\begin{figure}[tpb]

			\centering
			\captionsetup[subfigure]{oneside, margin={-1.5cm, 0cm, 0cm}}
			\begin{subfigure}[b]{0.48\textwidth}
				\includegraphics[width=0.8\textwidth]{NLO-uncrossed}
				\caption{}
				\label{fig:NLO-uncrossed}
			\end{subfigure}
			\begin{subfigure}[b]{0.48\textwidth}
				\includegraphics[width=0.8\textwidth]{NLO-crossed}
				\caption{}
				\label{fig:NLO-crossed}
			\end{subfigure}
			\caption{The leading logarithmic contributions to $qg\rightarrow qg$ at NLO.  The uncrossed
 			         diagram, $\mathcal{M}_{qQ\rightarrow qQ}^{\text{NLO}, II}$, shown in (a) exchanges
 			         two gluons in the $t$ channel and the crossed diagram,
 			         $\mathcal{M}_{qQ\rightarrow qQ}^{\text{NLO}, X}$, case (b) exchanges two gluons
 			         in the $u$ channel and is related to (a) (up to a colour factor) via a crossing
 			         symmetry}
			\label{fig:NLO-leadingContrib.}
		\end{figure}

	\section{High Energy Jets `Currents'}
		\label{sub:currents}

		It should not come as a surprise that the $qQ\rightarrow qQ$ and $qg\rightarrow qg$ channels closely resemble
		each another since if we consider the sub-diagrams involved in the $t$-channel exchange namely the quark and
		gluon `currents' in the high energy limit we see that they themselves are similar.  The gluon emission
		current, $j_{g}^\sigma$, shown in fig. (Is this a figure worth having?) is given by

		\begin{equation}
			j_{g}^\sigma = \left((p_a+p_1)^\sigma g^{\mu\nu} + (q - p_1)^\mu g^{\nu\sigma} -
			(q + p_a)^\nu g^{\mu\sigma}\right)\epsilon_{a, \mu}\epsilon_{1, \nu}^*.
		\end{equation}

		In the High Energy limit we can take $p_a\sim p_1$ and therefore this becomes:

		\begin{equation}
			j_{g}^\sigma \sim \left(2p_a^\sigma g^{\mu\nu} - p_a^\mu g^{\nu\sigma} -
			p_1^\nu g^{\mu\sigma}\right)\epsilon_{a, \mu}\epsilon_{1, \nu}^*.
		\end{equation}

		But the second and third terms here are zero once we take the contraction with the gluon polarisations $\epsilon_{a, \mu}$
		and $\epsilon_{1, \nu}^*$ respectively so we are left with simply:

		\begin{equation}
			j_{g}^\sigma \sim 2p_a^\sigma\epsilon_{a}^\mu\cdot\epsilon_{1, \mu}^*.
			\label{eqn:gluonCurrent}
		\end{equation}

		If we compare this to the quark current, $j_{q}^\sigma$, shown in fig. (Is this a figure worth having?),
		which is given by:

		\begin{equation}
			j_{q}^\sigma = \bk{1}{\sigma}{a},
		\end{equation}

		and using the spinor-helicity property for spinors with the same momenta:

		\begin{equation}
			j_{q}^\sigma \sim 2p_a^\sigma
		\end{equation}

		which is identical to \eqref{eqn:gluonCurrent} aside from a scalar gauge-dependent term.  In what
		follows we choose to construct high multiplicity matrix elements by decomposing the full result
		in to the contraction of two currents and a number of effective vertices.  This choice allows us
		to construct matrix elements with a simple form and which contain the leading logarithms by
		ensuring that they have the maximal number of gluons exchanged in the $t$-channel.\\
		As a simple example we consider the production of exclusive quad-jets in the High Energy limit.
		Fig. \eqref{fig:quadJets} shows three diagrams which all contribute at leading order in $\alpha_s$;
		fig. \eqref{fig:quadJets1} has three gluons exchanged in the $t$-channel and so its amplitude will
		have a term akin to\footnote{where we have used the High Energy limit to disregard the quark mass terms.}:

		\begin{equation}
			\mathcal{M}_{\text{(a)}}\sim\frac{1}{(p_a - p_1)^2(p_a - p_1 - p_2)^2(p_a - p_1 - p_2 - p_3)^2},
		\end{equation}

		arising from the gluon propagators.  By contrast figs. \eqref{fig:quadJets2} and \eqref{fig:quadJets3} will
		have, respectively:

		\begin{align}
			\mathcal{M}_{\text{(b)}}\sim\frac{\slashed p_a - \slashed p_1 - \slashed p_2}
			{(p_a - p_1)^2(p_a - p_1 - p_2)^2(p_a - p_1 - p_2 - p_3)^2},\\
			\intertext{and,}
			\mathcal{M}_{\text{(c)}}\sim\frac{(\slashed p_a - \slashed p_1)(\slashed p_4 - \slashed p_b)}
			{(p_a - p_1)^2(p_a - p_1 - p_2)^2(p_a - p_1 - p_2 - p_3)^2},
		\end{align}

		\begin{figure}[hbt]
			\centering
			\begin{subfigure}[b]{0.31\textwidth}
				\includegraphics[width=\textwidth]{quadJets1}
				\caption{}
				\label{fig:quadJets1}
			\end{subfigure}
			\begin{subfigure}[b]{0.31\textwidth}
				\includegraphics[width=\textwidth]{quadJets2}
				\caption{}
				\label{fig:quadJets2}
			\end{subfigure}
			\begin{subfigure}[b]{0.31\textwidth}
				\includegraphics[width=\textwidth]{quadJets3}
				\caption{}
				\label{fig:quadJets3}
			\end{subfigure}
			\caption{Three processes contributing to exclusive quad-jet production. (a) has the
			maximum number of gluons exchanged in the $t$-channel (three) and will dominate in the High
			Energy limit, (b) and (c) only have two and one gluon which can reggeise.  As such as we move
			from left to right we will lose powers of large logarithms but maintain the same power of
			$\alpha_s$ and therefore we can reasonably approximate quad-jet production by neglecting
			(b) and (c).}
			\label{fig:quadJets}
		\end{figure}

		coming from their mix of quark and gluon $t$-channel exchanges.  Since in the High
		Energy limit we have $p_a\sim p_1$ and $p_b\sim p_4$ it is clear that $\mathcal{M}_{\text{(b)}}$
		and $\mathcal{M}_{\text{(c)}}$ will be suppressed with respect to $\mathcal{M}_{\text{(a)}}$.
		Clearly this only a heuristic argument but it is sufficient to motivate the construction of
		high multiplicity amplitudes from $t$-channel gluon exchanges with the understanding that any
		other diagrams will be formally sub-leading.\\Here is a convenient place to define the `$t$-channel
		factorised' form for matrix elements, $\overline{\mathcal{M}}^t_{qQ\rightarrow qQ}$, in which we
		extract the $t$ poles from the rest of the matrix element \cite{Andersen:2009nu}.  We write
		eqn. \eqref{eqn:similarBrackets} as:

		\begin{equation}
			|\overline{\mathcal{M}}^t_{qQ\rightarrow qQ}|^2 = \frac{1}{4(N_c^2-1)}
			\frac{g^2C_F}{t_1}\frac{g^2C_F}{t_2} \sum_{h_a, h_b, h_1, h_2}
			|S_{qQ\rightarrow qQ}^{h_ah_b\rightarrow h_1h_2}|^2,
			\label{eqn:factorised}
		\end{equation}

		where $N_c=3$ and $C_F=\frac{4}{3}$ for QCD, $S$ is the matrix element for a $2\rightarrow2$ process
		in the form of a contraction of two currents, and $t_i$ are the squared $t$-channel momenta - in this
		case $t_1=(p_a-p_1)^2$ and $t_1=(p_2-p_b)^2$.  While for this example eqn. \eqref{eqn:factorised} is
		just an exact rewriting of a previous result we will use the form shown here to generalise to describing extra
		final state radiation in the next section at which point the $t$-channel factorisation weakens becomes an
		approximation of the full result (but a very convenient one nevertheless).

	\section{Effective Vertices For Real Emissions}
		\label{sub:effective_vertices_for_real_emissions}

		In order to generalise what we have done so far to higher multiplicity scattering events we begin
		by considering $qQ\rightarrow qQg$ in the high energy limit.  The five diagrams which contribute at
		leading order and leading logarithm accuracy are shown in fig. \eqref{fig:2To3}.  The diagram
		where the extra gluon is emitted from the $t$-channel gluon is given by:

		\begin{figure}[hbt]
			\begin{center}
			\includegraphics[width=0.7\linewidth]{2To3}
			\caption{The 5 possible emission sites of extra QCD radiation in $qQ\rightarrow qQ$.
			Fig. from \cite{Andersen:2009nu}.}
			\label{fig:2To3}
			\end{center}
		\end{figure}

		\begin{align}
			\mathcal{M}_{t\text{-channel}} = &-\frac{g_s^3}{t_1t_2}C_g\bk{1}{\mu}{a}\bk{3}{\nu}{b}
			\epsilon^*_\rho\\&\left((q_1 + q_2)^\rho g^{\mu\nu} + (p_2 - q_2)^\mu g^{\nu\rho} - (q_1 + p_2)^\nu g^{\mu\rho}\right).
		\end{align}

		where $C_g = f^{i2j}T^i_{1a}T^j_{3b}$.  We can use the High Energy limit to simplify this to:

		\begin{align}
			\mathcal{M}_{t\text{-channel}} \sim -\frac{2g_s^3s}{t_1t_2}f^{i2j}T^i_{1a}T^j_{3b}\epsilon^*_\rho
			\left((q_1 + q_2)^\rho - 2p_a^\rho\frac{s_{2b}}{s} + p_b^\rho\frac{s_{2a}}{s}\right).
		\end{align}

		where $s_{ij} = (p_i + p_j)^2$.  The remaining four diagrams can be treated by viewing the extra gluon emissions as soft with
		respect to the external hard partons (by using the Eikonal approximation).  The resulting
		amplitude for the sum of all four may be written in terms of the tree level amplitude as

		\begin{align}
			\mathcal{M}_{\text{Eikonal}} = i\mathcal{M}_{qQ\rightarrow qQ}g_s\epsilon^*_\rho
			\Big(C_1\frac{p_1^\rho}{p_1\cdot p_2} - C_A\frac{p_a^\rho}{p_a\cdot p_2}
			+ C_3\frac{p_r^\rho}{p_2\cdot p_3} - C_B\frac{p_b^\rho}{p_b\cdot p_2}\Big),
		\end{align}

		where $C_i$ is the colour factor associated with the Eikonal gluon being emitted from the
		parton line $i$.  Again this can be simplified using High Energy considerations yielding:

		\begin{equation}
			\mathcal{M}_{\text{Eikonal}} = i\frac{S_{qQ\rightarrow qQ}}{t_1t_2}g_s^3\epsilon^*_\rho
			\left((C_1-C_A)t_1\frac{p_a^\rho}{p_a\cdot p_2} + (C_3-C_B)t_2\frac{p_b^\rho}{p_b\cdot p_2}\right).
		\end{equation}

		The combinations of colour factors simplify to:

		\begin{align}
			(C_1 - C_a) &=  iC_g \\
			(C_3 - C_b) &= -iC_g.
		\end{align}

		Since all the contributions are proportional to the same colour factor we can simply sum
		them to get:

		\begin{equation}
			\mathcal{M}_{qQ\rightarrow qQg} = \frac{S_{qQ\rightarrow qQ}}{t_1t_2}C_gg_s^3\epsilon^*_\rho V_\rho(q_1, q_2),
		\end{equation}

		where

		\begin{equation}
			V^\rho(q_1, q_2) = -(q_1 + q_2)^\rho +
			p_a^\rho\left(\frac{p_b\cdot p_2}{p_a\cdot p_b} + \frac{q^2_2}{p_a\cdot p_2}\right) -
			p_b^\rho\left(\frac{p_a\cdot p_2}{p_a\cdot p_b} + \frac{q^2_1}{p_b\cdot p_2}\right).
		\end{equation}

		Though in practise we choose to symmetrise this expression so as to include as few
		approximations as possible.  The final result for the effective vertex is then:

		\begin{align}
			V^\rho(q_1, q_2) = -(q_1 + q_2)^\rho +
			&\frac{p_a^\rho}{2}\left(\frac{q^2_1}{p_a\cdot p_2} + \frac{p_2 \cdot p_b}{p_a \cdot p_b} + \frac{p_2 \cdot p_3}{p_a \cdot p_3}\right) + (p_a\leftrightarrow p_1) \\
			- &\frac{p_b^\rho}{2}\left(\frac{q^2_2}{p_b\cdot p_2} + \frac{p_2 \cdot p_a}{p_a \cdot p_b} + \frac{p_2 \cdot p_1}{p_b \cdot p_1}\right) - (p_b\leftrightarrow p_3).
			\label{eqn:effVertex}
		\end{align}

		Eqn. \eqref{eqn:effVertex} is manifestly gauge invariant which can be checked explicitly by
		calculating $p_g\cdot V$.\\Armed with eqn. \eqref{eqn:effVertex} and the quark and gluon
		currents we can calculate high multiplicity matrix elements by generalising eqn.
		\eqref{eqn:factorised} to include contractions of this effective vertex expression.  The
		$2\rightarrow n$ matrix element squared for $qQ$ scattering is therefore given by:

		\begin{align}
			|\overline{\mathcal{M}}^t_{qQ\rightarrow qg\cdots gQ}|^2 = \frac{1}{4(N_c^2-1)}
			\frac{g^2C_F}{t_1}\frac{g^2C_F}{t_2} \sum_{h_a, h_b, h_1, h_2}
			|S_{qQ\rightarrow qQ}^{h_ah_b\rightarrow h_1h_2}|^2\\
			\times\prod_{i=1}^{n-1}\left(\frac{-g_sC_A}{t_it_{i+1}}V^\mu(q_i, q_{i+1})V_\mu(q_i, q_{i+1})\right).
			\label{eqn:factorised2ToN}
		\end{align}

		Using eqn. \eqref{eqn:factorised2ToN} we can describe the real high order corrections
		but this expression is manifestly divergent for the reasons outlined in section
		\ref{sec:divAndReg}.  As in section \ref{sub:eg1loop} we must calculate the virtual
		corrections to render the integrated cross section finite.

	\section{Virtual Corrections To All Orders}
		\label{sub:effective_vertices_for_real_emissions}

	\section{High Energy Phase-space Integration}
		\label{sub:HEPhaseSpace}

