\chapter{High Energy QCD}
	\label{chap:HEQCD}

	In this chapter we look in detail at the `High Energy' limit of QCD.  We begin by defining this limit and
	looking at how basic $2\rightarrow2$ scattering behaves at leading order and next-to-leading order in
	$\alpha_s$ before discussing how, in this limit, scattering amplitudes may be conveniently expressed as
	a contraction between two vector `current' terms.  Finally, we show how we may adorn $2\to2$ matrix
	elements with real and virtual corrections by way of an effective vertex for real emissions and the Lipatov
	ansatz respectively.

\section{The `High Energy' limit}
	\label{sub:HElimit}

	The `High Energy' limit of QCD, also referred to as the Multi-Regge Kinematic (MRK) limit is
	defined in terms of the kinematics of the final state.  We require a \emph{strong rapidity ordering}
	of all outgoing radiation as well as all the emissions having \emph{similar transverse momenta}.
	Mathematically this is:

	\begin{equation}
		y_1\gg y_2\gg\cdots\gg\\y_n \text{ and } |p_{\perp1}| \approx |p_{\perp2}| \approx\cdots\approx|p_{\perp(n-1)}|,
		\label{eqn:MRK}
	\end{equation}

		where we define the rapidity of a final state particle as

	\begin{equation}
		y = \half\ln\frac{E+p_z}{E-p_z}
		\label{eqn:rap}
	\end{equation}

	where $E$ is the energy of particle and $p_z$ is the $z$ component of its momentum. We can
	state the criteria in eqn.~\eqref{eqn:MRK} equivalently as:

	\begin{equation}
		s_{ij}\rightarrow\infty\text{ for all i, j,}
	\end{equation}

	where $s_{ij} = (p_i + p_j)^2$ is the invariant mass of a pair of outgoing partons.  We sometimes
	parametrise the final states using pseudo-rapidity, $\eta$, rather than rapidity. Pseudo-rapidity
	is simply related to the angle of the outgoing state to the beam, $\theta$:

	\begin{equation}
		\eta = -\ln\tan\frac{\theta}{2}.
		\label{eqn:prap}
	\end{equation}

	For massless states eqn.~\eqref{eqn:rap} and eqn.~\eqref{eqn:prap} are equivalent.

\section{Mandelstam Variables in the High Energy Limit}
	\label{sub:MandelstamVariables}

	The $2\rightarrow 2$ QCD scattering amplitudes can be expressed in terms of the well-known Mandestam
	variables $s$, $t$ and $u$.  Which, in terms of the momenta in the process, are given by:

	\begin{align}
	\begin{split}
		s = (p_a + p_b)^2, \\
		t = (p_a - p_b)^2, \\
		u = (p_b - p_2)^2,
		\label{eqn:mandel}
	\end{split}
	\end{align}

	where $p_a$, $p_b$ are the incoming parton momenta and $p_1$, $p_2$ are the outgoing parton momenta.
	When working in the high energy limit it is convenient to re-express these in terms of the
	perpendicular momentum of the outgoing partons, $p_\perp$, and the difference in rapidity
	between the two final state partons, $\Delta y$.  If we parametrise our outgoing states as

	\begin{align}
	\begin{split}
		p_1 = p_{\perp1}\big(\cosh (y_1), \cos(\phi_1), \sin(\phi_1), \sinh (y_1)\big),\\
		p_2 = p_{\perp2}\big(\cosh (y_2), \cos(\phi_2), \sin(\phi_2), \sinh (y_2)\big),
	\end{split}
	\end{align}

	then we can express eqs.~\eqref{eqn:mandel} as follows

	\begin{align}
	\begin{split}
		s &=  4p_\perp^2 \cosh^2\frac{\Delta y}{2}, \\
		t &= -2p_\perp^2 \cosh  \frac{\Delta y}{2}e^{-\frac{\Delta y}{2}}, \\
		u &= -2p_\perp^2 \cosh  \frac{\Delta y}{2}e^{ \frac{\Delta y}{2}}.
	\end{split}
	\end{align}

	In the limit of hard jets well separated in rapidity, i.e. $\Delta y\rightarrow\infty$,
	these are approximated by

	\begin{align}
	\begin{split}
		s &= p_\perp^2 e^{\Delta y},\\
		t &= -p_\perp^2,\\
		u &= -p_\perp^2 e^{\Delta y}.
		\label{eqn:mandel2}
	\end{split}
	\end{align}

	From  eqn.~\eqref{eqn:mandel2} it is clear that the `hard, wide-angle jet' limit, i.e. $\Delta y\to\infty$,
	$p_{i\perp}\to\infty$, is equivalent to the High Energy limit since as $\Delta y$ grows large $s$ will
	grow exponentially while $t$ will stay fixed.  Rearranging for $\Delta y$ in the above equations yields:

	\begin{equation}
		\Delta y = \ln \left(\frac{s}{-t}\right).
		\label{eqn:largeLogs}
	\end{equation}

	This is a useful result because it relates the simple kinematics of an event to a (potentially)
	large logarithm.  It is already apparent from eqn.~\eqref{eqn:largeLogs} that a final state
	with large rapidity gaps between jets will carry with it a large logarithm as seen in
	eqn.~\eqref{eqn:schematicExpn}, $L=\ln \left(\frac{s}{-t}\right)$, and therefore we many need a
	more careful inspection of our perturbative expansion than the fixed-order approach.

\section{$qQ$-scattering at High Energy (at LO)}
	\label{sec:qQScat}

	Here we begin with the simplest example; the case of $qQ\rightarrow qQ$ for all negative helicity partons
	(the capital $Q$ implies it is a different flavour to $q$).  There is only one diagram which contributes shown
	in fig. (\ref{fig:TwoToTwo}).  Using the Feynman rules detailed in section~\ref{sec:partonicCrossSection} we can
	write the matrix element as:

	\begin{align}
		i\mathcal{M}_{q^-Q^-\rightarrow q^-Q^-}^{\text{LO}} &= ig_s^2T^d_{1a}T^d_{2b}
		\frac{\overline{u}^-(p_1)\gamma^\mu
		  u^-(p_a)\text{ }\overline{u}^-(p_2)\gamma_\mu u^-(p_b)}{t}\\
		  &= ig_s^2T^d_{1a}T^d_{2b}\frac{\bk{1}{\mu}{a}\cdot\bk{2}{\mu}{b}}{t},
		  \label{eqn:similarBrackets}
	\end{align}

	where $t = (p_a - p_1)^2$ and we have used the shorthand
	$\overline{u}^-(p_i)\gamma^\mu u^-(p_j) = \bk{i}{\mu}{j}$ in the second line.
	Writing the contraction of these two `current' terms in terms of light-cone coordinates we have:

	\begin{equation}
		i\mathcal{M}_{q^-Q^-\rightarrow q^-Q^-}^{\text{LO}} = ig_s^2T^d_{1a}T^d_{2b}
		\frac{2\sqrt{p_a^-p_b^+}}{t}
		\left(\sqrt{p_1^+p_2^-}e^{i\phi_2} + \sqrt{p_1^-p_2^+}e^{i\phi_1}\right),
		\label{eqn:qQ2qQ}
	\end{equation}

	where $e^{i\phi_i} = \frac{p_{\perp i}}{|p_{\perp i}|}$.  We now approximate the kinematics
	in such a way that we may write eqn.~\eqref{eqn:qQ2qQ} in a `factorised' form once again.
	Specifically we consider that the scattering can be thought of as two partons glancing off
	one another.  That is, we assume that $p_1^+\ll p_1^-$, $p_2^-\ll p_2^+$ and that $p_a$ ($p_b$)
	is moving in the backwards (forward) direction.  We can further assume that $p_1^-\approx p_a^-$
	and $p_2^+\approx p_b^+$ and with this we see that~\eqref{eqn:qQ2qQ} becomes:

	\begin{equation}
		i\mathcal{M}_{q^-Q^-\rightarrow q^-Q^-}^{\text{LO}} =
		\frac{2s}{t}\left(g_sT^d_{1a}e^{i\phi_1}\right)\left(-ig_sT^d_{2b}\right),
		\label{eqn:reggeTraj}
	\end{equation}

	which is `factorised' in the sense that each scalar term in brackets depends only on one
	quark line; either on the $p_{a/1}$ line or the $p_{b/2}$ line.  We see that the amplitude
	for $qQ\rightarrow qQ$ is dominated by the $s$ kinematic variable.  We can express this as:

	\begin{equation}
		\mathcal{M}_{q^-Q^-\rightarrow q^-Q^-}^{\text{LO}} \sim s^{\alpha(t)},
	\end{equation}

	which is exactly the behaviour expected when a particle exchanged in the $t$-channel has `reggeised'
	\cite{sabioThesis,DelDuca:1995hf,lipatovBook}.  $\alpha(t)$ is the Regge trajectory and is equal to
	the intrinsic spin of the state exchanged.  In our example we have a spin-one gluon exchanged
	and accordingly we can see from eqn.~\eqref{eqn:reggeTraj} that $\alpha(t)=1$ for $qQ\rightarrow qQ$.

	It is also interesting to consider the same process but with a helicity structure $q^-Q^+\to q^-Q^+$.
	The calculation proceeds similarly to the $q^-Q^-\to q^-Q^-$ and when we write the result in
	terms of two vector `currents' we get:

	\begin{equation}
		i\mathcal{M}_{q^-Q^+\rightarrow q^-Q^+}^{\text{LO}} = ig_s^2T^d_{1a}T^d_{2b}
		\frac{\bk{1}{\mu}{a}\cdot\bk{b}{\mu}{2}}{t},
		\label{eqn:HEfail}
	\end{equation}

	which is still manifestly expressible as a contraction of two vector currents.  The contraction
	can be written approximately as $2[a2]\langle b1\rangle\sim2[a2]\langle2a\rangle=-2u$.
	However when we continue on and take the High Energy limit of eqn.~\eqref{eqn:HEfail} we get:

	\begin{align}
	\begin{split}
		i\mathcal{M}_{q^-Q^+\rightarrow q^-Q^+}^{\text{LO}} &=
		ig_s^2T^d_{1a}T^d_{2b}\frac{2}{t}\sqrt{p_a^-p_2^+}\sqrt{p_b^+p_1^-}e^{i\phi_1}\\
		&=ig_s^2T^d_{1a}T^d_{2b}\frac{2s}{t}
		\label{eqn:HEfail2}
	\end{split}
	\end{align}

	So we see that since in the full High Energy limit we have that $s=-u$
	exactly.  Whereas when we leave the amplitude in terms of currents we are able to keep
	more of the physics by having the weaker constraint $s\sim-u$; at the LHC $t$ and $k_\perp^2$
	can often differ significantly and so the over-approximating the kinematics here would
	lead to a poor description of the data.

	\begin{figure}
		\begin{center}
		\includegraphics[width=0.35\linewidth]{TwoToTwo}
		\caption{The only diagram which contributes to $qQ\rightarrow qQ$ at leading order in $\alpha_s$.}
		\label{fig:TwoToTwo}
		\end{center}
	\end{figure}

\section{$qg$ scattering at High Energy}
	\label{sec:qg}

	We now explore the more involved case of $q^-g^+\to q^-g^+$ scattering.  At leading order this
	consists of three diagrams shown in fig.~\eqref{fig:TwoToTwo2}.  We use the following gauge
	choice for the gluon polarisations:

	\begin{align}
	\epsilon^{+*}_{2\sigma}&=\frac{\langle b|\sigma|2\rangle}{\sqrt{2}\langle b2\rangle}
	& \epsilon^{-*}_{2\sigma} &= -\frac{\langle b|\sigma|2\rangle}{\sqrt{2}[b2]} \\
	\epsilon^{+}_{b\sigma}&=-\frac{\langle b|\sigma|2\rangle}{\sqrt{2}[2b]}
	& \epsilon^{-*}_{2\sigma} &= -\frac{\langle b|\sigma|2\rangle}{\sqrt{2}\langle 2b\rangle}
	\end{align}

	For simplicity we choose to write everything in terms of negative helicity spinor-helicity brackets;
	to describe positive helicities we can use the transposition property of spinor-helicity brackets discussed
	in section~\ref{sec:SpinorHelicity}.

	\begin{figure}[h]
		\centering
		\begin{subfigure}[b]{0.3\textwidth}
			\includegraphics[width=\textwidth]{qg2qg-s}
			\caption{}
			\label{fig:qg2qg-s}
		\end{subfigure}

		\begin{subfigure}[b]{0.3\textwidth}
			\includegraphics[width=\textwidth]{qg2qg-t}
			\caption{}
			\label{fig:qg2qg-t}
		\end{subfigure}
		~
		\begin{subfigure}[b]{0.3\textwidth}
			\includegraphics[width=\textwidth]{qg2qg-u}
			\caption{}
			\label{fig:qg2qg-u}
		\end{subfigure}
		\caption{The $s$, $t$ and $u$ channel diagrams contributing to $q^-g^+\to q^-g^+$ at leading
		         order in $\alpha_s$ in figures (\ref{fig:qg2qg-s}), (\ref{fig:qg2qg-t}) and (\ref{fig:qg2qg-u})
		         respectively.}
		\label{fig:TwoToTwo2}
	\end{figure}

	\subsection{$s$-channel}

		The matrix element for the $s$-diagram, shown in fig.~\eqref{fig:qg2qg-s}, is:

		\begin{align}
			\mathcal{M}_{q^-g^+\to q^-g^+, s}^{\text{LO}}=&T^b_{ae}T^2_{e1}\overline{u}^-(p_1)\left(-\frac{ig_s}{2}\gamma^\mu\right)\epsilon_\mu^{*+}(p_2)
				\frac{i(\slashed q+m)}{q^2-m^2}\left(-\frac{ig_s}{2}\gamma^\nu\right)\epsilon^+_\nu(p_b)u^-(p_a), \\
			=&-\frac{g^2_s}{4q^2}\epsilon^{*+}_{2\mu}\epsilon^+_{b\nu}\overline{u}^-_1\gamma^\mu\slashed q\gamma^\nu u^-_a,
		\end{align}

		where we have used $q\gg mc$ for the high energy case (i.e. we treat the quarks as massless in the High Energy limit).
		The propagator has momentum $q=p_a+p_b=p_1+p_2$ and therefore:

		\begin{align}
			\mathcal{M}_{q^-g^+\to q^-g^+, s}^{\text{LO}}=&-T^b_{ae}T^2_{e1}\frac{g^2_s}{4q^2}\frac{\langle{b}|\mu|2\rangle}{\sqrt{2}\langle{b2\rangle}}
			\frac{\langle{b}|\nu|2\rangle}{\sqrt{2}[2b]}\overline{u}^-_1\gamma^\mu(\slashed{p}_a+\slashed{p}_b)\gamma^\nu u^-_a.
		\end{align}

		Now we can use the completeness relations for $\slashed p_{a/b}$ and see that:

		\begin{equation}
			\mathcal{M}_{q^-g^+\to q^-g^+, s}^{\text{LO}}=-T^b_{ae}T^2_{e1}\frac{g^2_s}{4q^2t}[2a]\langle ab\rangle\langle{b}|\mu|2\rangle\langle{1}|\mu|a\rangle
		\end{equation}

		Using $q^2=s_{ab}=\langle ab\rangle[ba]$ and $t=\langle2b\rangle[b2]$ we have:

		\begin{equation}
		\mathcal{M}_{q^-g^+\to q^-g^+, s}^{\text{LO}}=-T^b_{ae}T^2_{e1}\frac{g^2_s}{4}\frac{[2a]\langle ab\rangle}{\langle ab\rangle[ba]
		\langle2b\rangle[b2]}\langle{b}|\mu|2\rangle\langle{1}|\mu|a\rangle.
		\end{equation}

		Now we must calculate the spinor products.  We use the conventions for spinors outlined in the
		previous chapter.  For example:

		\begin{align}
			[2a] = &\overline{u}^+_2u^-_a=-\frac{\sqrt{p_a^+p_2^-}p_2^\perp}{|p_2^\perp|},
		\end{align}

		and after calculating the other brackets we see:

		\begin{equation}
			\mathcal{M}_{q^-g^+\to q^-g^+, s}^{\text{LO}} = -T^b_{ae}T^2_{e1}\frac{g_s^2}{4}\sqrt{\frac{p_2^-}{p_b^-}}\frac{1}{p_2^+p_b^-} \frac{p_{2\perp}^*}{|p_{2\perp}|} \bk{b}{\mu}{2} \bk{1}{\mu}{a}
		\end{equation}

		Which can be simplified to give the final result:

		\begin{equation}
			\mathcal{M}_{q^-g^+\to q^-g^+, s}^{\text{LO}}=-T^b_{ae}T^2_{e1}\frac{g_s^2}{2\hat{t}}\sqrt{\frac{p_2^-}{p_b^-}}\frac{p_{2\perp}^*}
			{|p_{2\perp}|}\langle{b}|\mu|2\rangle\langle{1}|\mu|a\rangle
			\label{eqn:s-channel}
		\end{equation}

	\subsection{$t$-channel}

		The matrix element for the $t$-channel diagram, shown in fig.~\eqref{fig:qg2qg-t}, is:

		\begin{align}
		\begin{split}
			-i\mathcal{M}_{q^-g^+\to q^-g^+, t}^{\text{LO}} =
			&-T^e_{a1}f^{b2e}\overline{u}^-_1\Bigg(-\frac{ig_s}{2}\gamma^\mu\Bigg)\Bigg(-\frac{ig_{\mu\nu}}{q^2}\Bigg)u^-_ag_s\\
			&\Big(g_{\sigma\nu}(p_b-q)_\rho + g_{\nu\rho}(q+p_b)_\sigma - g_{\rho\sigma}(p_b+p_2)_\nu)\Big)\epsilon^{\rho *}_{2+}\epsilon^{\sigma}_{b+}\\
		\end{split}
		\end{align}

		Now using $q=p_2-p_b$ and $p_2\cdot\epsilon_2 = p_b\cdot\epsilon_b = 0$:

		\begin{align}
		\begin{split}
			-i\mathcal{M}_{q^-g^+\to q^-g^+, t}^{\text{LO}} =
		        -T^e_{a1}f^{b2e}\frac{g_s^2}{2q^2s_{2b}}\left(\overline{u}^-_1\gamma^{\nu}u^-_a\right)\left(\overline{u}^-_b\gamma^{\rho}u^-_2\right)\left(\overline{u}^-_b\gamma^{\sigma}u^-_2\right)\\
			\Big(2g_{\sigma\nu}p_{b\rho} + 2g_{\nu\rho}p_{2\sigma} - g_{\rho\sigma}(p_b+p_2)_\nu\Big),
		\end{split}
		\end{align}

		which cancels completely and therefore:

		\begin{equation}
			\mathcal{M}_{q^-g^+\to q^-g^+, t}^{\text{LO}}=0,
			\label{eqn:t-channel}
		\end{equation}

		in this gauge.

	\subsection{$u$-channel}

		The matrix element for the $u$-diagram, shown in fig.~\eqref{fig:qg2qg-t}, is:

		\begin{equation}
			\begin{split}
			-i\mathcal{M}_{q^-g^+\to q^-g^+, u}^{\text{LO}} &= T^2_{ae}T^b_{e1}\overline{u}^-(p_1)\left(-\frac{ig_s}{2}\gamma^\mu\right)
			\frac{i(\slashed q+mc)}{q^2-m^2c^2}\left(-\frac{ig_s}{2}\gamma^\nu\right)u^-(p_a)\epsilon_\mu^{*+}(p_b)\epsilon^+_\nu(p_2)\\
			\mathcal{A}_u &= \frac{g_s^2}{4q^2}\overline{u}^-_1\gamma^\mu\slashed q\gamma^\nu u^-_a\epsilon^{+*}_{b\mu}\epsilon^*_{2\nu}\\
			&= \frac{g_s^2}{8q^2s_{2b}}\langle b|\mu|2\rangle\langle b|\nu|1\rangle\overline{u}^-_1\gamma^\mu(\slashed{p}_a-\slashed{p}_2)\gamma^\nu u^-_a\\
			\end{split}
		\end{equation}

		Where we have used $q=p_a-p_2$.  By direct comparison with the procedure used for the $s$-channel we can see the result will be:

		\begin{equation}
		\mathcal{M}_{q^-g^+\to q^-g^+, u}^{\text{LO}}=
		T^2_{ae}T^b_{e1}\frac{g_s^2}{2\hat{t}}\sqrt{\frac{p_b^-}{p_2^-}}\frac{p^*_{2\perp}}{|p_{2\perp}|}\langle{b}|\mu|2\rangle\langle{1}|\mu|a\rangle.
		\label{eqn:u-channel}
		\end{equation}

		The total total matrix element is given by the sum of eqs.~\eqref{eqn:s-channel},~\eqref{eqn:t-channel} and~\eqref{eqn:u-channel} which is:

		\begin{equation}
			\mathcal{M}_{q^-g^+\to q^-g^+}^{\text{LO}}
			\frac{g_s^2}{2}\frac{p_{2\perp}^*}{|p_{2\perp}|}\left(T^2_{ae}T^b_{e1}\sqrt{\frac{p_b^-}{p_2^-}}-T^b_{ae}T^2_{e1}
			\sqrt{\frac{p_2^-}{p_b^-}}\right)\frac{\langle{b}|\mu|2\rangle\langle{1}|\mu|a\rangle}{\hat{t}},
			\label{eqn:fullsum}
		\end{equation}

		We also see that eqn.~\eqref{eqn:fullsum} has the same spinor-helicity brackets contracted as eqn.~\eqref{eqn:similarBrackets}
		and so the dominant behaviour of $q^-g^+\to q^-g^+$ in the high energy limit is $\frac{s}{t}$.
		In the High Energy limit we have $p_b^-\thicksim p_2^-$ and so eqn.~\eqref{eqn:fullsum} could be simplified
		further to:

		\begin{equation}
			\mathcal{M}_{q^-g^+\to q^-g^+}^{\text{LO}}=i\frac{g_s^2}{2}\frac{p_{2\perp}^*}{|p_{2\perp}|}f^{2bc}T^c_{a1}
			\frac{\langle{b}|\mu|2\rangle\langle{1}|\mu|a\rangle}{\hat{t}}.
			\label{eqn:fullsum2}
		\end{equation}

		which is identical to the result found in the previous $qQ\rightarrow qQ$ calculation (save for a phase which cancels
		at the amplitude squared level). Since the kinematics of which is exactly in the form of two `currents' contracted as
		seen in section~\ref{sec:qQScat}.  We have:

		\begin{equation}
			\mathcal{M}_{qg\to qg}^{\text{LO}} = \frac{C_A}{C_f} \mathcal{M}_{qQ\to qQ}^{\text{LO}},
		\end{equation}

		in the High Energy limit. In practice we actually choose \emph{not} to take the High Energy limit to obtain
		eqn.~\eqref{eqn:fullsum2} so as to approximate as little as possible.  Even without this extra approximation
		eqn.~\eqref{eqn:fullsum} is still exactly the form of a $t$-channel gluon exchange we saw in
		eqn.~\eqref{eqn:similarBrackets}.

		In section~\ref{sec:HEJ} we will return this result and the results of section~\ref{sec:qQScat} and discuss how,
		despite their simplicity, they can be used to construct very general approximate forms for matrix elements which
		could otherwise not be evaluated

\section{$qQ$-scattering at High Energy (at NLO)}
	\label{sub:HE22_NLO}

	Before we continue on to look at how we might add extra real and virtual emissions to high energy matrix
	elements we briefly look at higher order (in $\alpha_s$) corrections to the process we studied in
	section~\ref{sec:qQScat}.  So far we have seen the leading order processes with a $t$-channel
	exchange are logarithmically enhanced but in eqn.~\eqref{eqn:schematicExpn} we sketched out a form for
	the perturbative expansion which also had logarithmically enhanced higher order corrections.

	Here we continue on from section~\ref{sec:qQScat} and calculate the virtual diagrams which contribute
	a leading logarithm for $qQ\to qQ$ at next-to-leading in $\alpha_s$~\cite{sabioThesis,DelDuca:1995hf}.

	We might na\"ively expect that the next-to-leading order diagrams with the maximal number of $t$-channel exchanges
	will give the greatest enhancement and, indeed, this turns out to be the case.  These diagrams are shown
	for the case of $qQ\rightarrow qQ$ in fig.~\eqref{fig:NLO-leadingContrib}.  We can rule out the other
	virtual diagrams which contribute at this order since they will contain (anti-)quark propagators along
	the $p_{a/1}$ or $p_{b/2}$ lines and in the high energy limit these will lead to a suppression.

	These diagrams in fig.~\eqref{fig:NLO-leadingContrib} can be elegantly computed by employing the `Cutkosky
	rules' which are used to relate two sub-diagrams to the imaginary part of a higher order diagram through
	the Optical theorem.  This can be seen very quickly since the scattering matrix, $S$, must be unitary i.e.
	$S^\dagger S=1$.  If we write this instead in terms of the the transition matrix, $T$, defined by $S=1+iT$
	then we immediately have that

	\begin{equation}
		-i(T - T^\dagger) = T^\dagger T.
	\end{equation}

	The left hand side of which can be written as twice the imaginary part of $T$.  If we now imagine
	that $T$ represents the transition from some initial state $|i\rangle$ to some final state
	$|f\rangle$ then we can write this as:

	\begin{equation}
		2i\text{Im}(\bk{i}{T}{f}) =  \sum_{p}\bk{i}{T^\dagger}{p}\bk{p}{T}{f},
	\end{equation}

	where we have inserted a sum over a complete set of states $|p\rangle$.  Pictorially we `cut' propagators
	by forcing them on-shell with delta functions and inserting a complete set of states.

	For example the uncrossed amplitude in fig.~\eqref{fig:NLO-uncrossed},
	$\mathcal{M}_{qQ\rightarrow qQ}^{\text{NLO, II}}$, may be expressed as a combination of
	two copies of the amplitude arising from fig.~\eqref{fig:TwoToTwo}:

	\begin{figure}[tpb]

		\centering
		\captionsetup[subfigure]{oneside, margin={-1.5cm, 0cm, 0cm}}
		\begin{subfigure}[b]{0.48\textwidth}
			\includegraphics[width=0.8\textwidth]{NLO-uncrossed}
			\caption{}
			\label{fig:NLO-uncrossed}
		\end{subfigure}
		\begin{subfigure}[b]{0.48\textwidth}
			\includegraphics[width=0.8\textwidth]{NLO-crossed}
			\caption{}
			\label{fig:NLO-crossed}
		\end{subfigure}
		\caption{The leading logarithmic contributions to $qg\rightarrow qg$ at NLO.  The uncrossed
			         diagram, $\mathcal{M}_{qQ\rightarrow qQ}^{\text{NLO}, II}$, shown in (a) exchanges
			         two gluons in the $t$ channel and the crossed diagram,
			         $\mathcal{M}_{qQ\rightarrow qQ}^{\text{NLO}, X}$, case (b) exchanges two gluons
			         in the $u$ channel and is related to (a) (up to a colour factor) via a crossing
			         symmetry}
		\label{fig:NLO-leadingContrib}
	\end{figure}

	\begin{equation}
		\mathcal{M}_{q^-Q^-\rightarrow q^-Q^-}^{\text{LO}} =
		-g_s^2T^d_{1a}T^d_{2b}\frac{\bk{1}{\mu}{a}\cdot\bk{2}{\mu}{b}}{t},
	\end{equation}

	as follows:

	\begin{align}
		2\text{Im}\Big(\mathcal{M}_{qQ\rightarrow qQ}^{\text{NLO, II}}\Big) =
		\frac{1}{(2\pi)^2}\int &d^4k\text{ }\delta((p_a-k)^2)
		\delta((p_b+k)^2)\\ &\mathcal{M}_{q^-Q^-\rightarrow q^-Q^-}^{\text{LO}}(k)
		\mathcal{M}_{q^-Q^-\rightarrow q^-Q^-}^{\dagger\text{LO}}(k-q),
	\end{align}

	where $\text{Im}(\cdot)$ denotes the imaginary part, $k$ is the loop momentum, $q$ is the momentum
	transfer and $\dagger$ denotes Hermitian conjugation.  The sum a complete of states here just
	corresponds to integrating over all possible momenta flowing around the loop. In the High Energy
	limit we can perform the integration to give:

	\begin{equation}
		\text{Im}\Big(\mathcal{M}_{qQ\rightarrow qQ}^{\text{NLO, II}}\Big) =
		4\alpha_s^2 s\text{ }\mathcal{C}_1(T^a,T^b)
		\int \frac{dk_{\perp}}{k_{\perp}(k_{\perp} - q_{\perp})},
	\end{equation}

	where $\mathcal{C}_1(T^a,T^b)$ is the colour factor for the diagram in fig.~\eqref{fig:NLO-uncrossed}
	and $k_{\perp}$ is the transverse component of $k$.  We can now relate the
	imaginary part of the amplitude to the full amplitude by conjecturing that the amplitude will be
	logarithmically enhanced as follows:

	\begin{align}
	\begin{split}
		\mathcal{M}_{qQ\rightarrow qQ}^{\text{NLO, II}} =
		&\text{Re}(\mathcal{M}_{qQ\rightarrow qQ}^{\text{NLO, II}}) +
		i\text{Im}(\mathcal{M}_{qQ\rightarrow qQ}^{\text{NLO, II}}),
	\end{split}
	\end{align}

	and defining $\widetilde{\mathcal{M}}_{qQ\rightarrow qQ}^{\text{NLO, II}}$ as the leading
	logarithmic coefficient of the matrix element:

	\begin{align}
	\begin{split}
		\mathcal{M}_{qQ\rightarrow qQ}^{\text{NLO, II}} =&\widetilde{\mathcal{M}}_{qQ\rightarrow qQ}^{\text{NLO, II}}\ln\frac{s}{t} + \text{sub-leading}\\
		=&\widetilde{\mathcal{M}}_{qQ\rightarrow qQ}^{\text{NLO, II}}
		\ln\left(\left|\frac{s}{t}\right| -i\pi\right)+ \text{sub-leading},
		\label{eqn:rAndI}
	\end{split}
	\end{align}

	where we have used that $\frac{s}{t} < 0$.  Comparing real and imaginary parts of eqn.~\eqref{eqn:rAndI}
	and assuming that $\widetilde{\mathcal{M}}_{qQ\rightarrow qQ}^{\text{NLO, II}}$ is real we see that:

	\begin{equation}
		\text{Re}\Big(\mathcal{M}_{qQ\rightarrow qQ}^{\text{NLO, II}}\Big) =
		-\frac{1}{\pi}\text{Im}\Big(\mathcal{M}_{qQ\rightarrow qQ}^{\text{NLO, II}}\Big)
	\end{equation}

	and we can therefore reconstruct the real part of the amplitude as:

	\begin{equation}
		\text{Re}\Big(\mathcal{M}_{qQ\rightarrow qQ}^{\text{NLO, II}}\Big) =
		-\frac{4\alpha_s^2u}{\pi} \mathcal{C}_1(T^a,T^b)
		\ln\left|\frac{u}{t}\right|\int \frac{dk_{\perp}}{k_{\perp}(k_{\perp} - q_{\perp})}.
		\label{eqn:uncrossedNLOcontrib}
	\end{equation}

	The crossed-diagram,~\eqref{fig:NLO-crossed}, also contributes a leading logarithmic piece and is related to
	eqn.~\eqref{fig:NLO-uncrossed} by a crossing symmetry and so we simply replace $u$ with $s$ in eqn.
	\eqref{eqn:uncrossedNLOcontrib} and calculate a new colour factor, $\mathcal{C}_2(T^a,T^b)$:

	\begin{equation}
		\text{Re}\Big(\mathcal{M}_{qQ\rightarrow qQ}^{\text{NLO, X}}\Big) =
		-\frac{4\alpha_s^2s}{\pi} \mathcal{C}_2(T^a,T^b)
		\ln\left|\frac{s}{t}\right| \int \frac{dk_{\perp}}{k_{\perp}(k_{\perp} - q_{\perp})}.
		\label{eqn:crossedNLOcontrib}
	\end{equation}

	But in the high energy limit $s\sim -u$ (this is clear from eqn.~\eqref{eqn:mandel2}) and so we
	can combine these terms and express the leading logarithmic NLO term in terms of the leading order
	result:

	\begin{equation}
		\mathcal{M}_{qQ\rightarrow qQ}^{\text{NLO}} = \frac{3\alpha_s}{\pi^2}
		\hat{\alpha}(q)\ln\left|\frac{s}{t}\right|
		\mathcal{M}_{qQ\rightarrow qQ}^{\text{LO}},
		\label{eqn:enhancedNLO}
	\end{equation}

	where:

	\begin{equation}
		\hat{\alpha}(q) = \int dk_{\perp}\frac{q_{\perp}^2}{k_{\perp}(k_{\perp} - q_{\perp})}
		\label{eqn:lipDiverge}
	\end{equation}

	From eqn.~\eqref{eqn:enhancedNLO} we can see the logarithmic enhancement explicitly; there is still
	a suppression from the inclusion of an extra factor of $\alpha_s$ with respect to the leading
	order term but as we have seen previously the logarithm is related to the kinematics of the final
	state - namely - the rapidity gap between the outgoing quarks $p_1$ and $p_2$ and compensate
	for the smallness of $\alpha_s$.  Eqn.~\eqref{eqn:lipDiverge} will clearly diverge when we
	come to integrate over the soft region (where $k_\perp$ is very small).  This divergence
	will be treated, that is regularised and shown to cancel, in chapter~\ref{chap:Zs}.

\section{$t$-channel Dominance}
	\label{sec:tChannel}

	In what follows we construct high multiplicity matrix elements by approximating the full result
	by the contraction of two currents and a number of effective vertices.  This choice allows us
	to construct matrix elements with a simple form which contains the leading logarithms by
	ensuring that they have the maximal number of gluons exchanged in the $t$-channel.\\
	As a simple example we consider the production of 4 exclusive jets in the High Energy limit.
	Fig.~\eqref{fig:quadJets} shows three diagrams which all contribute at leading order in $\alpha_s$;
	fig.~\eqref{fig:quadJets1} has three gluons exchanged in the $t$-channel and so its amplitude will
	have propagator terms akin to:

	\begin{equation}
		\mathcal{M}_{\text{(a)}}\sim\frac{1}{(p_a - p_1)^2(p_a - p_1 - p_2)^2(p_a - p_1 - p_2 - p_3)^2},
		\label{eqn:inPlace}
	\end{equation}

	arising from the gluon propagators.  By contrast figs.~\eqref{fig:quadJets2} and~\eqref{fig:quadJets3} will
	have, in place of~\eqref{eqn:inPlace}:

	\begin{align}
		\mathcal{M}_{\text{(b)}}\sim\frac{\slashed p_a - \slashed p_1}
		{(p_a - p_1)^2(p_a - p_1 - p_2)^2(p_a - p_1 - p_2 - p_3)^2},\\
		\intertext{and,}
		\mathcal{M}_{\text{(c)}}\sim\frac{(\slashed p_a - \slashed p_1)(\slashed p_4 - \slashed p_b)}
		{(p_a - p_1)^2(p_a - p_1 - p_2)^2(p_a - p_1 - p_2 - p_3)^2},
	\end{align}

	\begin{figure}[bt]

		\centering

		\begin{subfigure}[b]{0.31\textwidth}
			\includegraphics[width=\textwidth]{quadJets1}
			\caption{}
			\label{fig:quadJets1}
		\end{subfigure}
		\begin{subfigure}[b]{0.31\textwidth}
			\includegraphics[width=\textwidth]{quadJets2}
			\caption{}
			\label{fig:quadJets2}
		\end{subfigure}
		\begin{subfigure}[b]{0.31\textwidth}
			\includegraphics[width=\textwidth]{quadJets3}
			\caption{}
			\label{fig:quadJets3}
		\end{subfigure}

		\caption{Three processes contributing to exclusive quad-jet production. (a) has the
		maximum number of gluons exchanged in the $t$-channel (three) and will dominate in the High
		Energy limit, (b) and (c) only have two and one gluon which can reggeise.  As such as we move
		from left to right we will lose powers of large logarithms but maintain the same power of
		$\alpha_s$ and therefore we can reasonably approximate quad-jet production by neglecting
		(b) and (c).}
		\label{fig:quadJets}
	\end{figure}

	respectively.  Since in the High
	Energy limit we have $p_a\sim p_1$ and $p_b\sim p_4$ it is clear that $\mathcal{M}_{\text{(b)}}$
	and $\mathcal{M}_{\text{(c)}}$ will be suppressed with respect to $\mathcal{M}_{\text{(a)}}$.
	Clearly this only a heuristic argument but it is sufficient to motivate the construction of
	high multiplicity amplitudes from $t$-channel gluon exchanges with the understanding that any
	other diagrams will be formally sub-leading.  We call the configurations with a maximal
	number of gluons exchanged in the $t$-channel an `FKL' configuration;  For example
	fig.~\eqref{fig:quadJets1} is an FKL configuration while fig.~\eqref{fig:quadJets2} and fig.
	\eqref{fig:quadJets3} are `non-FKL' configurations.

	A more formal argument for which processes dominate in this limit was given by Fadin and Lipatov
	\cite{Kuraev:1976ge,Balitsky:1978ic}.  They found that, in the High Energy limit, scattering
	amplitudes scaled in the same way as was predicted by Regge theory.  This states that in the
	large invariant mass region a $2\to n$ matrix element has a limiting behaviour determined by
	the maximum spin of any particle which could be exchanged in the $t$-channel between final
	state partons neighbouring in rapidity.  We can then find the scaling of a process, for example
	$qg\to qg$, in particular regions of phase space where either $y_g\gg y_q$ or $y_q\gg y_g$ simply
	by drawing the associated colour connection diagrams for it.  This is shown in~\eqref{fig:colorConns}.
	Since when we have $y_g\gg y_q$ it is only possible to exchange a colour triplet (with spin one half)
	the cross-section will be dominated by the region where $y_q\gg y_g$.  The case for $2\to n$ is similar
	with the limiting behaviour of the matrix element given by

	\begin{equation}
		\mathcal{M}^{\text{HE}}_{2\to n}\sim s_{12}^{\omega_1}\cdots s_{(n-1)n}^{\omega_{(n-1)}}.
		\label{eqn:reggeScaling}
	\end{equation}

	Eqn.~\eqref{eqn:reggeScaling} now makes the previous discussion regarding figs.~\eqref{fig:quadJets}
	formally clear since fig.~\eqref{fig:quadJets1} will scale like:

	\begin{equation}
		\mathcal{M}^{\text{HE}}_{2\to 4} \sim s_{12}s_{23}s_{34},
	\end{equation}

	in the High Energy limit while figs.~\eqref{fig:quadJets2} and~\eqref{fig:quadJets3} will scale
	like:

	\begin{align}
	\begin{split}
		\mathcal{M}^{\text{HE}}_{2\to 4} \sim s_{12}^{\nicefrac{1}{2}}s_{23}s_{34},\\
		\mathcal{M}^{\text{HE}}_{2\to 4} \sim s_{12}^{\nicefrac{1}{2}}s_{23}s_{34}^{\nicefrac{1}{2}},
	\end{split}
	\end{align}

	respectively.  Since $s_{ij}$ are all large here, the processes with a (anti)quark exchanged
	in the $t$-channel will be high suppressed.

	\begin{figure}
		\begin{center}
		\includegraphics[width=1.0\linewidth]{ColourConns}
		\caption{The limiting behaviour of $qg\to qg$ in the regions of phase space where
		either $y_g\gg y_q$ or $y_q\gg y_g$.  The intermediate diagrams indicate the flow
		of colour through the process.}
		\label{fig:colorConns}
		\end{center}
	\end{figure}

	To further illustrate this we can look at the various processes contributing to the two
	jet exclusive cross-section, and in particular, the High Energy limits of their matrix
	elements.  Table~\ref{tab:LOatHE} shows several examples of parton level processes and
	their exact leading order matrix elements~\cite{pinkBook}.  We can see clearly from this
	that any process which can proceed through a $t$-channel gluon exchange has a term
	proportional to $\nicefrac{s^2}{t^2}$ which will dominate in the High Energy limit; tor
	example $q\bar{Q}\to q\bar{Q}$ has only the diagram which such an exchange.
	Conversely processes in which a $t$-channel gluon diagram can not contribute are
	suppressed in this limit.  For example $q\bar{q}\to Q\bar{Q}$ may only happen via an
	$s$-channel gluon and we can see that in the limit $s\to\infty$ and $t\to0$ it's matrix
	element tends to $\nicefrac{4}{9}$.  Processes like $gg\to gg$ have diagrams with \emph{and}
	without the exchange we are interested in and, as such, only some of the terms from the exact
	leading order matrix element contribute - but they do still contribute.

	\begin{table}[htp!]
		\begin{center}
		\begin{tabular}{c | c }
		Process                 & $\nicefrac{1}{g^4}|\bar{\mathcal{M}}|^2$ \\ \hline
		$qQ\to qQ$              & $\frac{4}{9}\frac{s^2 + u^2}{t^2}$       \\
		$q\bar{Q}\to q\bar{Q}$  & $\frac{4}{9}\frac{s^2 + u^2}{t^2}$       \\
		$qq\to qq$              & $\frac{4}{9}\left(\frac{s^2 + u^2}{t^2} + \frac{s^2 + t^2}{u^2}\right) - \frac{8}{27}\frac{s^2}{ut}$\\
		$q\bar{q}\to Q\bar{Q}$  & $\frac{4}{9}\frac{t^2 + u^2}{s^2}$       \\
		$gg\to gg$              & $\frac{9}{2}\left(3-\frac{tu}{s^2}-\frac{su}{t^2}-\frac{st}{u^2}\right)$\\
		\end{tabular}
		\caption{Some examples of $2\to2$ leading order matrix elements which contribute to the
		two jet exclusive cross-section.}
		\label{tab:LOatHE}
		\end{center}
	\end{table}

\section{Effective Vertices For Real Emissions}
	\label{sec:effectiveVertices}

	In order to generalise what we have done so far to higher multiplicity scattering events we begin
	by considering $qQ\rightarrow qQg$ in the high energy limit.  The five diagrams which contribute at
	leading order are given in fig.~\eqref{fig:2To3}.  The diagram
	where the extra gluon is emitted from the $t$-channel gluon is given by:

	\begin{figure}[hbt]
		\begin{center}
		\includegraphics[width=0.7\linewidth]{2To3}
		\caption{The 5 possible emission sites of extra QCD radiation in $qQ\rightarrow qQ$.
		Fig. from \cite{Andersen:2009nu}.}
		\label{fig:2To3}
		\end{center}
	\end{figure}

	\begin{align}
	\begin{split}
		\mathcal{M}_{t\text{-channel}} = &-\frac{g_s^3}{t_{a1}t_{b2}}f^{i2j}T^i_{1a}T^j_{3b}\bk{1}{\rho}{a}\bk{3}{\mu}{b}
		\epsilon^*_{2\nu}\\&\left(2p_2^\mu g^{\nu\rho} - 2p_2^\rho g^{\mu\nu} - (q_1 + q_2)^\nu g^{\mu\rho} \right),
		\label{eqn:mtchannel}
	\end{split}
	\end{align}

	and the remaining four diagrams contribute like:

	\begin{align}
	\begin{split}
	    \mathcal{M}_{\text{Eik.}} = (ig_s)^3 \epsilon_{2\nu} \Bigg(
	     &T^2_{1i}T^d_{ia}T^d_{3b}\ \frac{2p_1^\nu\bk{1}{\mu}{a} + \bk{1}{\nu}{2}\bk{2}{\mu}{a}} {s_{12}t_{b3}} \bk{3}{\mu}{b} \\
	    +&T^d_{1i}T^2_{ia}T^d_{3b}\ \frac{2p_a^\nu\bk{1}{\mu}{a} - \bk{1}{\mu}{2}\bk{2}{\nu}{a}} {t_{a2}t_{b3}} \bk{3}{\mu}{b} \\
	    +&T^2_{3i}T^d_{ib}T^d_{1a}\ \frac{2p_3^\nu\bk{3}{\mu}{b} + \bk{3}{\nu}{2}\bk{2}{\mu}{b}} {s_{32}t_{a1}} \bk{1}{\mu}{a}\\
	    +&T^d_{3i}T^2_{ib}T^d_{1a}\ \frac{2p_b^\nu\bk{3}{\mu}{b} - \bk{3}{\mu}{2}\bk{2}{\nu}{b}} {t_{b2}t_{a1}} \bk{1}{\mu}{a}\Bigg).
		\label{eq:fulltree}
	\end{split}
	\end{align}

	In the High Energy limit the second term in each of the line is suppressed with respect to the
	first and can therefore be disregarded.  This turns out to be equivalent to if we considered
	$p_2$ as a soft emission using the Eikonal approximation.  The resulting amplitude for the sum
	of all four may be written in terms of the tree level amplitude as:

	\begin{align}
	\begin{split}
	    \mathcal{M}_{\text{Eik.}} = (ig_s)^3 \epsilon_{2\nu} \bk{1}{\mu}{a} \bk{3}{\mu}{b} \Bigg(
	     &T^2_{1i}T^d_{ia}T^d_{3b}\ \frac{2p_1^\nu}{s_{12}t_{b3}}
	    + T^d_{1i}T^2_{ia}T^d_{3b}\ \frac{2p_a^\nu}{t_{a2}t_{b3}} \\
	    +&T^2_{3i}T^d_{ib}T^d_{1a}\ \frac{2p_3^\nu}{s_{32}t_{a1}}
	    + T^d_{3i}T^2_{ib}T^d_{1a}\ \frac{2p_b^\nu}{t_{b2}t_{a1}}\Bigg).
		\label{eq:fulltree2}
	\end{split}
	\end{align}

	We now use that $p_a\sim p_1=p_+$ and $p_b\sim p_2=p_-$:

	\begin{align}
	\begin{split}
		\mathcal{M}_{\text{Eik.}} = (ig_s)^3 \epsilon_{2\nu} \bk{1}{\mu}{a} \bk{3}{\mu}{b} \Bigg(&
		\frac{2p_+^\nu}{p_+\cdot p_2 t_{b3}}(T^2_{1i}T^d_{ia} - T^d_{1i}T^2_{ia})T^d_{3b}\\
		+ &\frac{2p_-^\nu}{p_-\cdot p_2 t_{a1}}(T^2_{3i}T^d_{ib} - T^d_{3i}T^2_{ib})T^d_{1a}\Bigg).\\
	\end{split}
	\end{align}

	Now tidying up the colour factors:

	\begin{align}
	\begin{split}
		\mathcal{M}_{\text{Eik.}} = (ig_s)^3 \epsilon_{2\nu} \bk{1}{\mu}{a} \bk{3}{\mu}{b}
		f^{2de}T^b_{3b}T^e_{1a}\frac{1}{t_{a1}t_{b3}}\Bigg(&
		\frac{2p_+^\nu}{p_+\cdot p_2}t_{a1} - \frac{2p_-^\nu}{p_-\cdot p_2}t_{b3}\Bigg),
		\label{eq:fulltree3}
	\end{split}
	\end{align}

	which has a colour factor similar to that found for the diagrams with a gluon emitted from the
	$t$-channel gluon.  We choose to `symmetrise' eqn.~\eqref{eqn:fulltree3} by returning to $p_{a/1}$
	and $p_{b/3}$ explicitly in place of $p_+$ and $p_-$ respectively:

	\begin{align}
	\begin{split}
		\mathcal{M}_{\text{Eik.}} = &(ig_s)^3 \epsilon_{2\nu} \bk{1}{\mu}{a} \bk{3}{\mu}{b} f^{2de}T^b_{3b}T^e_{1a}\frac{1}{t_{a1}t_{b3}}\\
		&\half\Bigg(\frac{2p_a^\nu}{p_a\cdot p_2}t_{a1} + \frac{2p_1^\nu}{p_1\cdot p_2}t_{a1} -
		\frac{2p_b^\nu}{p_b\cdot p_2}t_{b3} - \frac{2p_3^\nu}{p_3\cdot p_2}t_{b3}\Bigg).
		\label{eq:fulltree4}
	\end{split}
	\end{align}

	We now consider~\eqref{eqn:mtchannel}.  The final term contracts the two currents and so it is
	only the first two terms which need to be massaged into the right form.  Once again we
	approximate using $p_a\sim p_1=p_+$ and $p_b\sim p_3=p_-$ to write the currents as momenta.
	Upon doing this we find:

	\begin{align}
	\begin{split}
		\mathcal{M}_{t\text{-channel}} = &-\frac{g_s^3}{t_{a1}t_{b2}}f^{i2j}T^i_{1a}T^j_{3b} \epsilon^*_{2\nu} \\
		&\left(8e^{i\phi_-}(p_+^\nu p_-\cdot p_2 - p_-^\nu p_+\cdot p_2) - (q_1 + q_2)^\nu \bk{1}{\mu}{a}\bk{3}{\mu}{b} \right),
		\label{eqn:mtchannel2}
	\end{split}
	\end{align}

	where $\phi_-$ is a phase resulting from the spinor conventions detailed in chapter~\ref{chap:theory}.
	Now using that $s\sim2p_+\cdot p_-=\half\bk{1}{\mu}{a}\bk{3}{\mu}{b}e^{-i\phi_-}$ we can write all
	three terms as something proportional to the desired current structure:

	\begin{align}
	\begin{split}
		\mathcal{M}_{t\text{-channel}} = &-\frac{g_s^3}{t_{a1}t_{b2}}f^{i2j}T^i_{1a}T^j_{3b} \epsilon^*_{2\nu}
		\bk{1}{\mu}{a}\bk{3}{\mu}{b} \\&\left(4\left(p_+^\nu \frac{p_-\cdot p_2}{s} - p_-^\nu \frac{p_+\cdot p_2}{s}\right)
		- (q_1 + q_2)^\nu \right).
		\label{eqn:mtchannel3}
	\end{split}
	\end{align}

	Similarly as for $\mathcal{M}_{\text{Eik.}}$ we chose to include as much of the actual
	kinematic information as possible by symmetrising~\eqref{eqn:mtchannel3} to get:

	\begin{align}
	\begin{split}
		\mathcal{M}_{t\text{-channel}} = &-\frac{g_s^3}{t_{a1}t_{b2}}f^{i2j}T^i_{1a}T^j_{3b} \epsilon^*_{2\nu}
		\bk{1}{\mu}{a}\bk{3}{\mu}{b}\\
		&\Bigg(-(q_1 + q_2)^\nu + \half\Big(
		p_a^\nu \frac{p_2\cdot p_b}{p_a\cdot p_b} + p_a^\nu \frac{p_2\cdot p_3}{p_a\cdot p_3} +
		p_1^\nu \frac{p_2\cdot p_b}{p_1\cdot p_b} + p_1^\nu \frac{p_2\cdot p_3}{p_1\cdot p_3}  \\
	       &-p_b^\nu \frac{p_2\cdot p_a}{p_a\cdot p_b} - p_b^\nu \frac{p_1\cdot p_2}{p_b\cdot p_1} -
		p_2^\nu \frac{p_2\cdot p_a}{p_a\cdot p_3} - p_2^\nu \frac{p_1\cdot p_2}{p_1\cdot p_3}
		\Big)
		\Bigg).
		\label{eqn:mtchannel4}
	\end{split}
	\end{align}

	Since eqns.~\eqref{eqn:mtchannel4} and~\eqref{eq:fulltree4} to the same colour factor we can simply sum
	them to get:

	\begin{equation}
		\mathcal{M}_{qQ\rightarrow qQg} = \frac{S_{qQ\rightarrow qQ}}{t_{a1}t_{b2}}
		f^{2de}T^b_{3b}T^e_{1a}g_s^3\epsilon^*_\rho V_\rho(q_1, q_2),
	\end{equation}

	where

	\begin{align}
	\begin{split}
		V^\rho(q_1, q_2) = -(q_1 + q_2)^\rho +
		&\frac{p_a^\rho}{2}\left(\frac{q^2_1}{p_a\cdot p_2} + \frac{p_2 \cdot p_b}{p_a \cdot p_b} +
		\frac{p_2 \cdot p_3}{p_a \cdot p_3}\right) + (p_a\leftrightarrow p_1) \\
		- &\frac{p_b^\rho}{2}\left(\frac{q^2_2}{p_b\cdot p_2} + \frac{p_2 \cdot p_a}{p_a \cdot p_b} +
		\frac{p_2 \cdot p_1}{p_b \cdot p_1}\right) - (p_b\leftrightarrow p_3).
		\label{eqn:effVertex}
	\end{split}
	\end{align}

	Eqn.~\eqref{eqn:effVertex} is manifestly gauge invariant which can be checked explicitly by
	calculating $p_g\cdot V$.  It is, however clearly divergent:  if any of $p_a$, $p_b$, $p_1$,
	$p_2$ or $p_3$ becomes soft then the momenta contractions in the denominators of eqn.
	\eqref{eqn:effVertex} will become zero and the whole expression will explode.  We organise
	the cancellation of divergences is shown to cancel in the following chapter.

	Armed with eqn.~\eqref{eqn:effVertex} and the quark and gluon currents we can calculate
	high multiplicity matrix elements by generalising eqn.~\eqref{eqn:factorised} to include
	contractions of this effective vertex expression.  The $2\rightarrow n$ matrix element
	squared for $qQ$ scattering is therefore given by:

	\begin{align}
	\begin{split}
		|\overline{\mathcal{M}}^t_{qQ\rightarrow qg\cdots gQ}|^2 = \frac{1}{4(N_c^2-1)}
		\frac{g^2C_F}{t_1}\frac{g^2C_F}{t_2} \sum_{h_a, h_b, h_1, h_2}
		|S_{qQ\rightarrow qQ}^{h_ah_b\rightarrow h_1h_2}|^2\\
		\times\prod_{i=1}^{n-1}\left(\frac{-g_sC_A}{t_it_{i+1}}V^\mu(q_i, q_{i+1})V_\mu(q_i, q_{i+1})\right).
		\label{eqn:factorised2ToN}
	\end{split}
	\end{align}

	Using eqn.~\eqref{eqn:factorised2ToN} we can describe the real emission high order
	corrections but this expression is manifestly divergent for the reasons outlined in
	section \ref{sec:divAndReg}.  As in section~\ref{sub:eg1loop} we must calculate the
	virtual corrections to render the integrated cross section finite.

\section{Virtual Corrections To All Orders}
	\label{sub:virtuals}

	Thus far we have a prescription for approximating high energy scattering amplitudes with additional
	real radiation added through the effective vertices described in section~\eqref{sec:effectiveVertices}.
	However, to complete our picture we must also include the virtual corrections to the process in
	a similar way to the example shown in section~\eqref{sub:eg1loop}.  This is important
	not only since these processes obviously contribute to the process but also because, as we saw in
	the one loop $\gamma^*\to q\bar{q}$ calculation, the soft divergences in eqn.~\eqref{eqn:effVertex}
	need to be cancelled.  Both the cancellation in section~\eqref{sub:eg1loop} and the cancellation
	we will see here are examples of the KLN theorem~\cite{mutaBook} which states that the soft and virtual
	divergences in QCD must cancel - though of course we must still show that this is the case and that our
	final result is manifestly finite.

	In the High Energy limit we may include the virtual corrections to all orders in $\alpha_s$ by using
	the Lipatov ansatz \cite{Kuraev:1976ge}.  For $t$-channel gluons we replace the usual
	gluon propagator with a `dressed' version:

	\begin{equation}
		\frac{1}{q_i^2}\rightarrow\frac{1}{q_i^2}e^{\hat{\alpha}(q_i)\Delta_{i,i-1}},
		\label{eqn:lipAns}
	\end{equation}

	where:

	\begin{equation}
		\hat{\alpha}(q_i) = \alpha_sC_Aq_i^2\int \frac{d^{2+2\epsilon}k_{\perp}}{(2\pi)^{2+2\epsilon}}
		\frac{1}{k^2_\perp(k_\perp - q_{i\perp})^2}\mu^{-2\epsilon},
		\label{eqn:lipDiverge2}
	\end{equation}

	and $\Delta_{i,i-1}$ is the rapidity gap between the external gluon legs emitted from
	the dressed gluon.  Similarly to eqn.~\eqref{eqn:effVertex} in the preceding section this
	new expression for the propagator contains divergences arising from the soft limit
	of the integral in the expression for $\hat{\alpha}(q_i)$.  In the following chapter we
	show in some detail that these divergences cancel with those mentioned in section
	\ref{sec:effectiveVertices}.

	The keen reader will have noticed that eqn.~\eqref{eqn:lipDiverge2} is exactly what we
	found in our next-to-leading order calculation in eqn.~\eqref{eqn:lipDiverge} expressed
	in $2+2\epsilon$ dimensions rather than 2 (save for a few numerical factors).  This is
	no coincidence and, indeed, is the source of the ansatz.  Higher order (in $\alpha_s$)
	calculations~\cite{DelDuca:1995hf,9780511524387} have shown that to two loops the leading
	logarithmic part of the full amplitude is found exactly by expanding the exponential
	term in eqn.~\eqref{eqn:lipAns}.

\section{High Energy Jets}
	\label{sec:HEJ}

	\subsection{The \hej Framework}

	The \hej framework is the basis of the later chapters of this thesis.  Details of this framework
	beyond the brief summary presented here may be found in~\cite{ZPaper,Andersen:2009nu,Andersen:2009he,
	Andersen:2011hs,Andersen:2012gk}.

	\subsection{Factorisation Into Currents}
	\label{sec:currents}

		The \hej framework is based, in part, on the observations of sections~\eqref{sec:qQScat}
		and~\eqref{sec:qg}.

		In this sections we saw that in the High Energy limit we can write down matrix elements in the
		form of two vector `currents' contracted over a $t$-channel pole.  While one could argue that
		the fact that the $qQ\to qQ$ matrix element would factorise into a contraction of two vector
		currents with a $t$-channel pole was obvious (since the only contribution was from a $t$-channel
		diagram!), it was not at all obvious that this would also be the case for the $qg\to qg$
		amplitude would.  It can also be shown that the same structure is found even in the case of
		gluon-gluon scattering\cite{Andersen:2011hs}.

		It turns out that this factorisation into a form with only a $t$-channel pole holds for
		all the helicity configurations where the helicities of the incoming-outgoing parton lines
		remain unchanged.  For those diagrams where the helicity \emph{is} flipped we find poles
		in $s$ and $u$ and so these contributions are heavily suppressed in the High Energy limit.
		The fact that all of the approximate helicity averaged matrix elements squared for any
		combination of incoming partons, $a$ and $b$, can be written as:

		\begin{equation}
			|\bar{\mathcal{M}}_{2\to2}| \sim \sum_{h_a, h_b, h_1, h_2}
			\left|\frac{j^\mu_a(p_a, p_1)\ j_{b, \mu}(p_b, p_2)}{t}\right|^2,
		\end{equation}

		is exploited in \hej to express more general matrix elements (those with higher multiplicity or
		more complicated final states) approximately.

		For example by constructing a current describing a $W^\pm$ boson being emitted from an
		incoming-outgoing quark line we can then write down the matrix element for the process
		$q'q\to(W^\pm\to)\nu e^\pm q'Q$ as:

		\begin{equation}
			|\bar{\mathcal{M}}_{2\to2}|^2 \sim \sum_{h_a, h_b, h_1, h_2, h_{e^\pm}, h_\nu}
			\left|\frac{j^\mu_{W^\pm}(p_a, p_1, p_{e^\pm}, p_\nu)\ j_\mu(p_b, p_2)}{t}\right|^2,
			\label{eqn:wExample}
		\end{equation}

		where the quark line which emitted the $W^\pm$ has changed flavour from $q$ to $Q$.  Eqn.~\eqref{eqn:wExample}
		is an approximation which, technically, is only valid in the strict limit of infinite invariance mass
		between the outgoing quarks however, as we will see later, this approximation does a remarkably good
		job at describing data far away from its formal region of applicability.

		Here is a convenient place to define the `$t$-channel factorised' form for matrix elements,
		$\overline{\mathcal{M}}^t_{qQ\rightarrow qQ}$, in which we extract the $t$ poles from the
		rest of the matrix element \cite{Andersen:2009nu}.  We write the square of
		eqn.~\eqref{eqn:similarBrackets} as:

		\begin{equation}
			|\overline{\mathcal{M}}^t_{qQ\rightarrow qQ}|^2 = \frac{1}{4(N_c^2-1)}
			\frac{g^2C_F}{t_1}\frac{g^2C_F}{t_2} \sum_{h_a, h_b, h_1, h_2}
			|S_{qQ\rightarrow qQ}^{h_ah_b\rightarrow h_1h_2}|^2,
			\label{eqn:factorised}
		\end{equation}

		where $N_c=3$ and $C_F=\nicefrac{4}{3}$ for QCD, $S$ is the matrix element for a $2\rightarrow2$ process
		in the form of a contraction of two currents, and $t_i$ are the squared $t$-channel momenta - in this
		case $t_1=(p_a-p_1)^2$ and $t_1=(p_2-p_b)^2$.

		While for the $2\to2$ examples in section~\eqref{sec:qQScat} and section~\eqref{sec:qg} eqn.~\eqref{eqn:factorised}
		is just an exact rewriting of a previous result however we will use the form shown here to generalise to describing
		extra final state radiation in the next section at which point the $t$-channel factorisation weakens to
		an approximation of the full result (but one which contains enough of the underlying physics to be useful
		nonetheless).

		Extending eqn.~\eqref{eqn:factorised} to more higher multiplicity final states within the \hej
		framework is then done by using chains of products of effective vertices discussed in section
		~\eqref{sec:effectiveVertices} (for the real emissions) and the Lipatov ansatz described in section
		~\eqref{sub:virtuals} (for the virtual emissions).

	\subsection{The \hej Monte Carlo}

		The \hej framework is implemented in a general purpose Monte Carlo, referred to simply as
		\texttt{HEJ}, and publicly available at \url{http://hej.web.cern.ch/HEJ/}.  This
		\texttt{C++} package is under continual development to test and improve it and the work
		of chapter~\ref{chap:Zs} (among many other tweaks and improvements) was contributed to
		it throughout the course of this work.

		Here we briefly summarise the main aspects of the software aspects of \hej.  A general
		\HEJ run consists of three main stages.

		\begin{enumerate}
			\item \textbf{Setup}: a setup phase at which point a user defined input file is parsed
			and, based on the specifics of the input, one of several class hierarchies is
			initialised after which essential components for the physics stage are constructed including:
			an interface to a PDF package (either \texttt{MSTW} or
			\texttt{LHAPDF} (v6)), a (pseudo-)random number generator and
			a physics analysis.\\\HEJ comes with a stand-alone analysis class
			which implements many standard operations and is therefore sufficient in most
			cases however it may also be interfaced with the \texttt{Rivet} analysis package;
			this is particularly useful when comparing Monte Carlo results to data since
			many analyses are implemented in \texttt{Rivet} routines and it is, in principle,
			just a matter of plugging in the right analysis name.

			\item \textbf{Monte Carlo Generation}:...

			\item \textbf{Analysis}:..

		\end{enumerate}


	\subsection{Comparisons to Data}

	\subsection{Matching to \ARIADNE}

