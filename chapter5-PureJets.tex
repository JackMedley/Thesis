% !TEX root = thesis.tex

\chapter{Dijets and Gap Jets at \texttt{ATLAS}}
\label{chap:ATLAS}

	Here we present the results of a complex experimental study of the effects of jet vetoes and
	azimuthal decorrelations in dijet events~\cite{Aad:2014pua}.  High Energy Jets is compared
	to both data and state-of-the-art fixed-order perturbative QCD predictions supplemented with
	radiation by merging with a parton shower from \texttt{POWHEG+PYTHIA8} and \texttt{POWHEG+HERWIG}
	(both implemented through the \texttt{POWHEG BOX} package~\cite{1126-6708-2004-11-040}).

	The data are taken from 7~TeV proton-proton collisions as observed by the ATLAS
	experiment in two distinct data sets referred to as 2010 data and 2011 data.  The
	experimental cuts applied to these data sets differs and both are outlined in tab.~\eqref{tab:atlasPcuts}.

	\begin{table}[bth]
	  \centering
	  \begin{tabular}{|l|c|}
	    \hline
	    2010 Jets (anti-$k_T$, 0.6) & $p_{Tj}>20$~GeV, \; $|y_j|<4.4$ \\
	    & $p_{T1}> 60.0$~GeV, \; $p_{T2}> 50.0$~GeV, \; \\
	    & $Q_0=20$~GeV \\
	    \hline
	    2011 Jets (anti-$k_T$, 0.6) & $p_{Tj}>30$~GeV, \; $|y_j|<2.4$ \\
	    & $p_{T1}> 60.0$~GeV, \; $p_{T2}> 50.0$~GeV \\
	    & \; $Q_0=30$~GeV, \; $\Delta y_{jj}>1.0$\\
	\hline
	  \end{tabular}
	  \caption{Cuts applied to theory simulations in the ATLAS dijets analyses.
	  $Q_0$ is the gap jet veto scale. The results are shown in
	  figs.~\eqref{fig:atlasPJ1}--\eqref{fig:atlasPJ6}.}
	  \label{tab:atlasPcuts}
	\end{table}

	This study focused on additional jet activity in dijet events where the dijet system
	is constructed using the two leading jets in $p_T$ - this, is in stark contrast to
	defining dijets by the most forward and most backward jets: naturally the $p_\perp$
	choice favours high transverse momentum dijets in the central region (and hence with
	relatively small rapidity gaps) while the forward-backward dijet system definition
	typically lead to softer systems with bigger rapidity spans~\cite{Abazov:2013gpa}.
	These two jets are required
	to be significantly harder than any additional jets with extra cuts on the leading
	and sub-leading jets of $60$~GeV and $50$~GeV respectively.  After tagging the two
	leading jets in the event and additional jet radiation is only considered in the rapidity
	interval bounded by the dijets as a QCD correction.  Within the two data sets
	defined in tab.~\eqref{tab:atlasPcuts} a further subdivision was made.  For both the
	2010 and 2011 data a subset of events was defined by vetoing events with extra
	QCD radiation in the rapidity interval bounded by the dijet system above some veto
	scale $Q_0$.  It should also be noted that the 2011 data set required a minimum
	rapidity gap of 1.0 between the leading jets (for both cases with and without the gap jet
	veto) in order to encourage a large rapidity span into which extra radiation may arise.
	In summary, the full breakdown of this analysis then is into four event categories;
	2010 data with and without a veto applied to gap jets and the 2011 data with and
	without a veto applied to gap jets.

	I generated the predictions for these analyses on behalf of the High Energy Jets
	collaboration at the request of the ATLAS collaboration. Predictions for both
	the partonic \HEJ calculation (shown in green in this chapter) and the \HEJA
	calculation (shown in orange) were calculated. As discussed in chapter~\ref{chap:HEQCD} \ARIADNE is a parton shower package
	based on the Lund colour cascade dipole model. As per the algorithm outlined in section~\ref{sec:HEJ}, the steps necessary to remove the
	double counting in this interfaced package (which arises from \HEJ and \ARIADNE both
	generating soft radiation) make generating large data samples which can be used to give
	statistically significant predictions for the challenging regions of phase space considered
	here is \emph{extremely} computationally demanding.  The generation of these
	predictions was so demanding several changes were made to the \HEJ codebase; the
	size of event files generated meant that the analysis framework for \hej required
	rewriting so that binning to histograms could occur on-the-fly.  As discussed in section~\ref{sec:comp}, this analysis was
	initially run using local computing resources however it quickly became apparent
	that this would not be sufficient, at which point it was necessary to switch to
	running \HEJ on the CERN grid computing cluster - this was the first time this was done.

	% As discussed in section~\ref{sec:HEJ} a standard partonic \HEJ run
	% evaluates the matrix element for each event at 76 different scale choices (since
	% we allow for four choices of `central' scale and 19 combinations of the renormalisation
	% scale, $\mu_r$, and factorisation scale, $\mu_f$ based on each central choice).
	A central scale was chosen based on previously agreement with data seen in a previous
	ATLAS comparison to gap jets~\cite{Aad:2011jz}; this set of 19 scale combination
	form an `envelope' of predictions for each bin in each plot - this spread is then
	represented by the green bands shown in the figs.~\eqref{fig:atlasPJ1}--\eqref{fig:atlasPJ6}.
	Due to the structure of the matrix element evaluations within \HEJA
	we can only afford one scale choice per event and, as such, we cannot provide
	scale variation uncertainty bands with our showered numbers.  To be clear this
	is not a limitation of the physics since it is entirely possible to evaluate
	each matrix element multiple times - it is only a computational consideration.
	As seen in chapter~\ref{chap:Zs}, the renormalisation scale appears in a non-trivial way
	in the High Energy Jets matrix element.  It is contained implicitly in the strong coupling
	constant which is contained within the virtual corrections exponential.  As such
	we cannot simply generate predictions at a single scale and then re-weight events
	at the post-analysis level.  This is further complicated in \HEJA because of the
	additional scale dependencies in the parton shower description.

	The orange bands shown with the \HEJA predictions throughout this chapter are the
	statistical bands shown at the 68\% confidence level.  The calculation
	of these statistical bands had to be completed manually - that is to say, not using
	the out-of-the-box result provided by \texttt{ROOT} since many of the individual
	distributions required calculating ratios of distributions for which the statistical
	errors in the numerator and denominator are highly correlated.  It was seen that the
	na\"ive statistical uncertainty bands were not representative of the actual uncertainty.
	The bands presented were calculated using a statistical bootstrapping approach: given
	a Monte Carlo sample comprising $N$ events we repeatedly form `bootstrap samples'
	each of which also with $N$ events by randomly selecting subsets of the full sample (allowing for
	repetition).  Once a sufficient number of these have been generated we have a distribution
	of results for each bin in each histogram and we can easily form a statistical
	uncertainty band which contains 68\% of the bootstrapped samples in each bin.

	Here observables were studied
	as a function of two properties of a dijet system.  The rapidity gap between the dijets,
	$\Delta y$, and the mean traverse momenta of the dijet system, $\overline{p_T}$.  For small
	$\Delta y$ we expect to see fewer gap jets since there is
	a limited region in which extra jets may be clustered before they are clustered in
	to the dijet system itself.   Conversely, as we pull the dijets apart in rapidity
	we expect to see an increase in additional QCD radiation (since there is a larger
	phase space in which to radiate).  Similarly for $\overline{p_T}$, we expect that
	events with harder dijet systems will have higher gap activity simply because they
	can cheaply add extra radiation.

	Throughout the remainder of this chapter the  left hand figures, (a), show data and
	predictions for the 2010 data set with respect to the rapidity span of the dijet
	system, $\Delta y$, while the right hand figures, (b), show data and predictions from the
	2011 data set with respect to the mean transverse momenta of the dijet system,
	$\overline{p_T}$, with a rapidity gap enforced.

	We begin by discussing the gap fraction, $f(Q_0)$, defined as:

	\begin{equation}
		f(Q_0) = \frac{\sigma_{jj}(Q_0)}{\sigma_{jj}},
	\end{equation}

	where $\sigma_{jj}$ is the total dijet cross section passing the cuts in
	tab.~\eqref{tab:atlasPcuts} and $\sigma_{jj}(Q_0)$ is the dijet cross section passing
	the cuts plus the extra veto on additional gap scales for a scale choice $Q_0$.

	Fig.~\eqref{fig:atlasPJ1} shows the gap fraction with a veto scale of $20$~GeV in
	fig.~\eqref{fig:atlasPJ1a} and a veto scale of $30$~GeV in fig.~\eqref{fig:atlasPJ1b}.
	The data are shown in black with the inner bars representing the statistical uncertainty
	while the outer lines are the total uncertainty arising from statistical and systematic
	effects.  The behaviour observed is in line with our expectation discussed previously since the
	gap fraction decreases at both large $\Delta y$ and large $\overline{p_T}$.  We can see the
	best description of both data sets is given by \HEJA (excluding two very high $\overline{p_T}$
	bins where the predictions from \HEJA are visibly statistically limited). The partonic \HEJ
	prediction overshoots both data sets meaning that we underestimate the jet activity in the
	gap region while the predictions from \texttt{POWHEG} plus parton showers overestimate the
	QCD radiation here.  The big difference between the partonic and showered \HEJ lines in
	fig.~\eqref{fig:atlasPJ1} indicates clearly that in this region of phase space it
	is not sufficient to only describe the wide-angle logarithmically enhanced terms resummed
	within the High Energy Jets framework and that the soft and collinear logarithms added by
	interfacing to \ARIADNE play a very important r\^ole - equally it is clear from the
	differences between \HEJ and the other theory descriptions that we cannot
	completely describe data without resumming the High Energy logarithms.  It is visible from
	fig.~\eqref{fig:atlasPJ1} that the scale variations shown in \emph{partonic} \HEJ (since the
	\HEJA bands are statistical only) are quite significantly bigger than that shown by the other
	predictions.  This is expected since we only include matching up to leading-order whereas
	\texttt{POWHEG} is formally NLO accurate.  In pure fixed-order calculations it is well
	observed that as higher order terms are included a better control of the scale uncertainties
	is achieved.

	\begin{figure}[bth]
		\centering
		\begin{subfigure}[b]{0.48\textwidth}
			\includegraphics[width=\textwidth, height=1.0\textwidth]{pureJets3a}
			\caption{}
			\label{fig:atlasPJ1a}
		\end{subfigure}
		~
		\begin{subfigure}[b]{0.48\textwidth}
			\includegraphics[width=\textwidth, height=1.0\textwidth]{pureJets3b}
			\caption{}
			\label{fig:atlasPJ1b}
		\end{subfigure}
		\caption{The gap fraction, $f(Q_0)$, as a function of (a) the rapidity gap,
		$\Delta y$ in the 2010 data, and (b) the average $p_T$, $\overline{p_T}$, of
		the dijet system in the 2011 data.}
		\label{fig:atlasPJ1}
	\end{figure}

	Similarly, when we study the mean number of jets in the rapidity interval shown for in
	fig.~\eqref{fig:atlasPJ2} we see that \HEJA and \texttt{POWHEG+PYTHIA8} give the best
	description of the data.  Once again the partonic \HEJ prediction undershoots the data
	significantly when describing both of the dijet characteristics while \texttt{POWHEG+HERWIG}
	overestimates the jet activity.

	\begin{figure}[bth]
		\begin{subfigure}[b]{0.48\textwidth}
			\includegraphics[width=\textwidth, height=1.0\textwidth]{pureJets4a}
			\caption{}
			\label{fig:}
		\end{subfigure}
		~
		\begin{subfigure}[b]{0.48\textwidth}
			\includegraphics[width=\textwidth, height=1.0\textwidth]{pureJets4b}
			\caption{}
			\label{fig:}
		\end{subfigure}
		\caption{The average number of jets, $\langle N_{\text{jets}}
		\text{ in the rapidity interval}\rangle$, in the rapidity gap
		         bounded by the dijet system, as a function of (a) the
		         rapidity gap, $\Delta y$, and (b) the average $p_T$,
		         $\overline{p_T}$, of the dijet system.}
		\label{fig:atlasPJ2}
	\end{figure}

	We now turn to look at the azimuthal decorrelations of dijet systems.  These are defined
	as $\langle\cos(n(\pi-\Delta\phi))\rangle$ with $n=1, 2, \ldots$ and $\Delta\phi$ is the azimuthal separation between the leading dijets.  Here we only consider
	the first and second moments.  We follow the notation of~\cite{Aad:2014pua}
	and rewrite the second moment as $\langle\cos(2\Delta\phi)\rangle$.  Clearly for a final state
	with two partons momentum conservation will enforce that the jets be in a back-to-back configuration
	i.e. they will have $\Delta \phi=\pi$ and both $\langle\cos(\pi-\Delta\phi)\rangle$ and
	$\langle\cos(2\Delta\phi)\rangle$ will simply be constant at $1.0$.  As additional
	partons are emitted the moments will depart from the straight line as the constraint softens
	and the extra radiation allows for $\Delta\phi<\pi$.  These moments have long been seen
	as an excellent test of the difference between DGLAP QCD parton showers and BFKL-like
	resummations~\cite{Ducloue:2012bm}.  Indeed it is in these figures where we see the biggest
	difference between \HEJ and the \texttt{POWHEG} plus parton shower results.
	% The large
	% difference between \HEJ and \HEJA approach clearly shows that the parton shower effects
	% are again being probed in these regions of phase space.

	Fig.~\eqref{fig:atlasPJ3} shows the first azimuthal moment for the inclusive selection.
	We see that in the 2010 study \HEJA and both \texttt{POWHEG} descriptions slightly
	underestimate $\langle\cos(\pi-\Delta\phi)\rangle$ while the partonic \HEJ result slightly
	overestimates.  However, the evolution of the first azimuthal moment with respect to
	$\overline{p_T}$ shows a clear difference between the two formalisms.  Partonic \HEJ does not
	radiate sufficiently to predict the correct decorrelation at low mean transverse momentum
	(which is understood since it does not include the parton shower effects) while both
	\texttt{POWHEG} descriptions cause too much decorrelation at low $\overline{p_T}$.
	The best description of the data is given by \HEJA since it adds extra emissions to High
	Energy Jets which improves our description of the decorrelation.  Once again it is clear that the low
	$p_\perp$ region of fig.~\eqref{fig:atlasPJ3} is extremely sensitive to shower effects - not
	only because there is a large difference between \HEJ and \HEJA but also because the two
	\texttt{POWHEG} plus shower results differ significantly.

	\begin{figure}[bth]
		\centering
		\begin{subfigure}[b]{0.48\textwidth}
			\includegraphics[width=\textwidth, height=1.0\textwidth]{pureJets5a}
			\caption{}
			\label{fig:}
		\end{subfigure}
		~
		\begin{subfigure}[b]{0.48\textwidth}
			\includegraphics[width=\textwidth, height=1.0\textwidth]{pureJets5b}
			\caption{}
			\label{fig:}
		\end{subfigure}
		\caption{The first azimuthal angular moment, $\langle \cos(\pi-\Delta\phi)\rangle$,
		as a function of (a) the rapidity gap, $\Delta y$ and (b) the average $p_T$,
		$\overline{p_T}$, of the dijet system.}
		\label{fig:atlasPJ3}
	\end{figure}

	Another variable thought to be a good test of DGLAP vs. BFKL physics is the ratio of
	the second moment to the first moment, this is shown in fig.~\eqref{fig:atlasPJ4}.  Once
	again we do see a sizeable difference in the predictions given by the two resummations.
	Similarly to fig.~\eqref{fig:atlasPJ3} we see that interfacing to the \ARIADNE package
	brings the partonic \HEJ prediction into much better agreement with the data and that the
	parton showers alone does not give a good description of data.  We remark on the similarity
	of the \texttt{POWHEG} prediction that that of \HEJA here, given the stark contrast of the
	underlying physics in these two descriptions it is not at all expected.  This similarity
	has also been observed in previous studies and has been discussed in~\cite{Alioli:2012tp}.

	\begin{figure}[bth]
		\begin{subfigure}[b]{0.48\textwidth}
			\includegraphics[width=\textwidth, height=1.0\textwidth]{pureJets5c}
			\caption{}
			\label{fig:}
		\end{subfigure}
		~
		\begin{subfigure}[b]{0.48\textwidth}
			\includegraphics[width=\textwidth, height=1.0\textwidth]{pureJets5d}
			\caption{}
			\label{fig:}
		\end{subfigure}
		\caption{The ratio of the second azimuthal angular moment, $\langle \cos(2\Delta\phi)\rangle$,
		to the first azimuthal angular moment, $\langle \cos(\pi-\Delta\phi)\rangle$, as a function of
		(a) the rapidity gap, $\Delta y$, and (b) the average $p_T$, $\overline{p_T}$, of the dijet system.}
		\label{fig:atlasPJ4}
	\end{figure}

	The two remaining figures are similar to figs. \eqref{fig:atlasPJ3} and \eqref{fig:atlasPJ4} but
	with the addition of the jet veto applied to events.  Similarly to the inclusive case we see that the partonic \HEJ
	predictions overestimates the correlation for the 2010 and the 2011 data sets while the NLO plus
	shower predictions, once again, undershoot the decorrelation.  Given the statistical limited
	data available (especially for the 2010 gap jet vetoed data set) it is more difficult to draw
	clear conclusions here but certainly the inclusion of the \ARIADNE shower improves the High
	Energy Jets results.

	Lastly we have the ratio of the second azimuthal moment to the first azimuthal moment for the
	events which pass the additional jet veto.  \HEJA and \texttt{POWHEG+PYTHIA8} come closest
	to describing the data however there is some disagreement; in particular no one gives a good
	description of the evolution of this ratio at low $\overline{p_T}$.

	In summary, the best description of the data overall is given by \HEJA and \texttt{POWHEG+PYTHIA8}
	while parton level \HEJ overestimates (underestimate) the gap fraction (the mean number of jets
	in the rapidity gap) and overshoots both the first azimuthal
	moment and the ratio of the second to the first azimuthal moment.  \texttt{POWHEG+HERWIG} describes
	the data poorly for the gap fraction, the mean number of gap jets and the azimuthal decorrelations.
	From this it is clear that while the logarithmically enhanced resummed in the High Energy Jets
	framework are important in regions of phase space where we have large rapidity gaps (such as the
	analyses described here) there are equally important contributions arising from the logarithms
	given to us by the interface with a parton shower.  It is clear that no one package describes all
	of the data presented and therefore the LHC is probing challenging reasons of phase space and thus
	more attention is needed.  This study also strongly motivates further development of \hej with
	other parton shower descriptions.

	\begin{figure}[bth]
		\centering
		\begin{subfigure}[b]{0.48\textwidth}
			\includegraphics[width=\textwidth, height=1.0\textwidth]{pureJets6a}
			\caption{}
			\label{fig:}
		\end{subfigure}
		~
		\begin{subfigure}[b]{0.48\textwidth}
			\includegraphics[width=\textwidth, height=1.0\textwidth]{pureJets6b}
			\caption{}
			\label{fig:}
		\end{subfigure}
		\caption{The first azimuthal angular moment, $\langle \cos(\pi-\Delta\phi)\rangle$,
		for events passing the veto on gap activity above $Q_0=20\text{GeV}$ as a function
		of (a) the rapidity gap, $\Delta y$, and (b) the average transverse momentum, $\overline{p_T}$,
		of the dijet system.}
		\label{fig:atlasPJ5}
	\end{figure}

	\begin{figure}[bth]
		\begin{subfigure}[b]{0.48\textwidth}
			\includegraphics[width=\textwidth, height=1.0\textwidth]{pureJets6c}
			\caption{}
			\label{fig:}
		\end{subfigure}
		~
		\begin{subfigure}[b]{0.48\textwidth}
			\includegraphics[width=\textwidth, height=1.0\textwidth]{pureJets6d}
			\caption{}
			\label{fig:}
		\end{subfigure}
		\caption{The ratio of the second azimuthal angular moment,
		$\langle \cos(2\Delta\phi)\rangle$, to the first azimuthal angular moment,
		$\langle \cos(\pi-\Delta\phi)\rangle$, as a function of (a) the rapidity gap,
		$\Delta y$, and (b) the average $p_T$, $\overline{p_T}$, of the dijet system.  A
		veto of $Q_0=20\text{GeV}$, for (a), and $Q_0=30\text{GeV}$, for (b), is applied
		on activity in the rapidity gap is applied.}
		\label{fig:atlasPJ6}
	\end{figure}

