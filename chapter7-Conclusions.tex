% !TEX root = thesis.tex

\chapter{Conclusions and Outlook}
\label{chap:conclusion}

In this thesis we have studied the perturbative description of Quantum Chromodynamics at colliders
including the current Large Hadron Collider and a possible hadronic Future Circular Collider.  In
chapter~\ref{chap:theory} we gave an overview of a few key components required to perform the
calculations seen in later sections.  In particular, the question of the validity of truncating
this expansion at fixed order in $\alpha_s$ was raised since the presence of large logarithm
corrections could mean that the convergence of any such truncated series would be in question.

In chapter~\ref{chap:HEQCD} we specifically discussed the High Energy limit of QCD processes and
how, in this limit, simple $2\to2$ scattering processes can be expressed as a contraction of currents
over a pole in the momentum transfer variable, $t$.  We saw that the next-to-leading order corrections
to these simple processes contained an unexpected enhancement in exactly the form discussed earlier;
a logarithm of $s/t$ which would be large in the limit in question.  The idea of $t$-channel dominance
was generalised to more complicated processes using ideas from Regge theory.  We were able to include
extra real and virtual corrections by way of an effective vertex term and the Lipatov ansatz respectively.
We concluded with a discussion of the \hej framework for describing all-order corrections to dijet events
at hadronic colliders built upon these results; in particular we saw that in the High Energy limit we can describe processes in terms
of contractions of `current' and discussed the software set-up of \hej in the form of the publicly
available \HEJ package.

In chapter~\ref{chap:Zs} we focussed on dijet events in the presence of a di-lepton pair produced by
the decay of a $Z^0$ electroweak boson and an off-shell photon, $\gamma^*$.  We were able to derive a
current describing the $Z^0$ emission and saw that, for the two jet case, this agreed exactly with the
leading order result.  This was extended to include emission of a photon as well as the resulting $\zg$ interference.
We saw that the interference term arising from the multiple possible boson emission sites neglected in previous work by the \hej collaboration was no long small
and, therefore, developed the tools to be able to include all possible emission sites for a boson \emph{and}
the interference terms.  This result was then extended to arbitrarily high multiplicity final states through
the aforementioned effective vertex.  Due to the inclusion of the interference term it was necessary
to develop a new regularisation for the $\zg$ plus jets matrix elements - this was done and the resulting
scheme was explicitly shown to be finite upon performing the phase space integration.  The resulting
regularised matrix element was shown to be independent of our phase-space regularising
parameter, $\lambda_{cut}$.  The procedure by which the \HEJ matrix elements are matched to the leading-order
exact matrix elements (provided by \texttt{MadGraph\_aMC@NLO}) was then detailed and a summary of the computational
challenges encountered throughout this work was presented.  We concluded by comparing the \HEJ $\zg$ plus jets
package to several recent experimental comparisons from both the ATLAS and CMS experiments.  \HEJ was seen to
be in excellent agreement with the data in the regions where the logarithms we capture are significant.
In regions where the effect of the high energy logarithms are not the dominant force, such as the double-differential
Drell-Yan study presented, we see that \HEJ agrees well with the leading order prediction since in these
regions the fixed-order matching we include becomes significant.  Any deviations are well understood.

In chapter~\ref{chap:ATLAS} a study of dijet and gap events was presented.  Four separate final states
were considered in which the \HEJ package interfaced with the parton shower \ARIADNE, \HEJA, gave an
excellent description of data throughout.  We saw that partonic \HEJ alone was not able to correctly
describe the data highlighting the importance of the collinear logarithms added by the parton shower
resummation.  The fixed-order plus parton shower schemes also struggled to give good agreement with
data indicating that, already at centre-of-mass energies of 7~TeV, it is necessary to resum the higher
order logarithmic corrections included within the \hej framework.

In chapter~\ref{chap:100TeV} we presented an analysis of inclusive dijets in association with a $\zg$
decaying to a di-lepton pair at 100~TeV.  Three possible final state cuts were shown with possible
minimum transverse momentum jet cuts of 30, 60 and 100~GeV.  We saw that at 100~TeV the
convergence of the fixed-order QCD perturbative series could be seriously undermined by the logarithmic
corrections discussed in this thesis; this was somewhat neutralised by enforcing a jet cut of
100~GeV and even here we are only able to reduce the problems at 100~TeV to something comparable
to those observed at 7~TeV.  The regions of phase space where there were dijets spanning
large rapidity intervals or with high invariant mass were seen to be most effected by the increased
centre-of-mass energy.

To conclude, the effect of large logarithmic corrections on inclusive dijet events in association
with a $\zg$ boson is seen to be large.  Certainly a good description of data throughout the entire
phase-space would not be possible without an attempt to capture at least the leading logarithmic terms in $s/t$ at all orders in $\alpha_s$.
More generally correctly describing QCD events at hadronic colliders is not possible without resumming
the logarithms significant in the High Energy limit but this, alone, is not the full picture.  We have
seen throughout this work that, although the leading logarithmic terms are undoubtedly important, it is
also necessary to include as many of the terms described by current fixed-order approaches as possible.
In the experimental studies of chapters~\ref{chap:Zs} and~\ref{chap:ATLAS} \HEJ's predictions have
scale variation bands which are significantly larger than the fixed-order descriptions.  This is well
understood to be linked to our leading order matching - therefore the highest priority for \hej is to
modify the matching scheme within the \hej framework must be improved to allow matching the next-to-leading
order in $\alpha_s$ matrix elements.  Unfortunately this is not as simple as just replacing our current
\texttt{MadGraph\_aMC@NLO} interface to instead call an NLO matrix element since this would cause us to double
count contributions.  Furthermore the collinear terms neglected by \hej's wide angle approximation must
also be considered.  We saw in chapter~\ref{chap:ATLAS} that there are regions of phase space where the
logarithms contributed by \ARIADNE are necessary for us to describe data and that, once included, we gave
the best description of data.  To that end it would be excellent to interface \HEJ with more of the current
state-of-the-art parton showers such as \texttt{HERWIG} and \texttt{PYTHIA}.

Armed with next-to-leading
order accuracy and a range of possible parton showers the description of data at 7, 8, 13 or even 100~TeV
would be, I believe, excellent in many final states and troublesome regions of phase space currently of
interest to experimental and theoretical physicists alike.

