\chapter{Next-to-leading order corrections for $\gamma^*\to q\overline{q}$}
\label{chap:NLOAppendix}

	Here we present the full calculation for the next-to-leading order corrections (in $\alpha_s$) to
	$\gamma^*\to q\overline{q}$ which was summarised in chapter~\ref{chap:theory}.  The diagrams
	contributing at NLO were depicted previously in fig~\eqref{fig:NLOContributions}.

	\subsubsection{The Leading Order Process}

		If we let the pair-produced quarks have charge $\pm Qe$ then we have:

		\begin{equation}
			\mathcal{A}_0 = -ieQ\overline{u}^{\lambda_2}(k_2)\gamma^\mu v^{\lambda_1}(k_1)\epsilon^r_\mu(p),
		\end{equation}

		where we have used the QED Feynman rule for a quark-antiquark-photon vertex: $iQe\gamma^\mu$, the $\lambda_i$'s are the
		spins of the quarks, $r$ is the polarisation of the incoming photon and $p = k_1 + k_2$ is the momentum carried by the
		incoming photon.  To proceed we can square and since we are typically interested in unpolarised calculations we perform
		a sum over all polarisations, spins and colours:

		\begin{equation}
			|\overline{\mathcal{A}_0}|^2 = 3\sum_{\forall\lambda, r}e^2Q^2[\overline{u}^{\lambda_2}(k_2)\gamma^\mu
			v^{\lambda_1}(k_1)][\overline{v}^{\lambda_1}(k_1)\gamma^\nu v^{\lambda_2}(k_1)]\epsilon^r_\mu(p)\epsilon^r_{*\mu}(p).
		\end{equation}

		We now use Casimir's trick~\cite{griff} to convert this spinor string into a trace, using the replacements
		$\sum_r\epsilon^r_\mu\epsilon^r_{*\nu}=-g_{\mu\nu}$ and the completeness conditions for spinors:

		\begin{equation}
			|\overline{\mathcal{A}_0}|^2 = - e^2Q^2\Tr[\slashed k_2\gamma^\mu\slashed k_1\gamma_\mu],
		\end{equation}

		where we have used the high energy limit to discard the quark mass terms.  This trace can be evaluated in arbitrary
		dimensions to give, in the high energy limit:

		\begin{equation}
			|\overline{\mathcal{A}_0}|^2 = 6e_d^2Q^2s(d-2),
		\end{equation}

		where we have defined the usual Mandelstam variable $s=(k_1+k_2)^2=2k_1\cdot k_2$ and define $e_d^2=e^2\mu^{4-d}$ where
		$\mu$ has units of mass in order to make the coupling $e$ dimensionless.  To find the leading order cross-section we divide by the
		particle flux and multiply by the two particle phase space which is given by:

		\begin{equation}
			\int d^{2d-2}R_2 = 2^{1-d}\pi^{\frac{d}{2}-1}\frac{\Gamma(\frac{d}{2}-1)}{\Gamma(d-2)}s^\frac{d-4}{2},
		\end{equation}

		where $R_2$ is the two particle phase space in $d$ dimensions.  Combining these factors and defining $\alpha_e=\frac{e^2}{4\pi}$:

		\begin{equation}
		\begin{split}
			\sigma_0 &= 3\cdot2^{2-d}\pi^{1-\frac{d}{2}}\frac{\Gamma(\frac{d}{2}-1)}{\Gamma(d-2)}s^\frac{d-4}{2}4\pi\alpha\mu^{d-4}Q^2s(d-2)\frac{1}{2s} \\
			&= 3\alpha Q^2\left(\frac{s}{4\pi\mu^2}\right)^{\frac{d}{2}-2}\left(\frac{d}{2}-1\right)\frac{\Gamma(\frac{d}{2}-1)}{\Gamma(d-2)}.
		\end{split}
		\end{equation}

		and finally using $x\Gamma(x)=\Gamma(x+1)$ we get:

		\begin{equation}
			\sigma_0 = 3\alpha Q^2 \frac{\Gamma(\frac{d}{2})}{\Gamma(d-2)}\left(\frac{s}{4\pi\mu^2}\right)^{\frac{d}{2}-2}.
			\label{eqn:bornCrossSection}
		\end{equation}

		It is important to note that in the limit $\epsilon\to0$ (i.e. $d\to4$) the Born cross-section remains finite.

	\subsubsection{The Virtual $\mathcal{O}(\alpha_s)$ Corrections}

		The virtual correction graphs are shown in figs.~\eqref{fig:NLOfig_2},~\eqref{fig:NLOfig_3} and~\eqref{fig:NLOfig_4}.  We will begin by calculating
		the second term in eqn.~\eqref{eqn:MEBreakdown}.  Using the Feynman rules we have:

		\small
			\begin{subequations}
			\begin{align*}
				\mathcal{A}_v = \int\frac{d^{d}k}{(2\pi)^{d}} \overline{u}^{\lambda_2}(k_2)
				(-ig_s\mu^\epsilon\gamma^\alpha T^a_{ij})\frac{i(\slashed k_1 + \slashed k)}{(k_1+k)^2}
				(-ieQ\gamma^\mu)\frac{i(\slashed k_2 - \slashed k)}{(k_2 - k)^2}\\
				(-g_s\mu^\epsilon\gamma^\beta T^a_{ij})
				\epsilon^r_\mu(p)\frac{-i}{k^2}\left(g_{\alpha\beta} +
				(1-\xi)\frac{k^\alpha k^\beta}{k^2}\right)v^{\lambda_1}(k_1).
			\end{align*}
			\begin{align*}
				\mathcal{A}_v = -ig_s^2eQ\mu^{2\epsilon}\Tr(T^aT^a)\overline{u}^{\lambda_2}
				(k_2)\int\frac{d^{d}k}{(2\pi)^{d}}\frac{\mathcal{N}_1(k_1, k_2, k)}{k^2(k_1+k)^2(k_2-k)^2}v^{\lambda_2}(k_2),
			\end{align*}
			\end{subequations}
		\normalsize

		where the numerator of the fraction is given by:

		\begin{equation}
			\mathcal{N}_1(k_1, k_2, k) = \gamma^\alpha(\slashed k_1 + \slashed k)\gamma^\mu(\slashed k_2 -
			\slashed k)\gamma_\beta\Big(g^{\alpha\beta} + (1-\xi)\frac{k^\alpha k^\beta}{k^2}\Big).
		\end{equation}

		From eqn.~\eqref{eqn:MEBreakdown} we see we need $\mathcal{A}_0^*\mathcal{A}_v$:

		\begin{align}
			\mathcal{A}_0^*\mathcal{A}_v = &g_s^2e^2Q^2\Tr(T^aT^a)[\overline{v}^{\lambda_1}(k_1)\gamma^\nu u(k_2)]\\
			&\left[\overline{u}^{\lambda_2}(k_2)\int\frac{d^{d}k}{(2\pi)^{d}}\frac{\mathcal{N}_1(k_1, k_2, k)}
			{k^2(k_1+k)^2(k_2-k)^2}v^{\lambda_1}(k_1)\right]\epsilon^r_{\mu}(p)\epsilon^r_{*\nu}(p).
		\end{align}

		Now performing the spin/polarisation/colour sum and average gives:

		\begin{equation}
			\overline{\mathcal{A}_0^*\mathcal{A}_v} = -\frac{g_s^2e^2Q^2}{2}\int\frac{d^{d}k}
			{(2\pi)^{d}}\frac{\mathcal{N}_2(k_1, k_2,k)}{k^2(k_1+k)^2(k_2-k)^2},
		\end{equation}

		where:

		\begin{equation}
			\mathcal{N}_2(k_1, k_2, k) = \Tr[\slashed k_1 \gamma_\alpha(\slashed k_1 + \slashed k)\gamma_\mu
			(\slashed k_2 - \slashed k)\gamma_\beta\slashed k_2\gamma^\mu]\Big(g^{\alpha\beta} + (1-\xi)\frac{k^\alpha k^\beta}{k^2}\Big).
			\label{eqn:258}
		\end{equation}

		Before we can proceed any further we must evaluate the trace term in eqn.~\eqref{eqn:258}.  As mentioned briefly in
		section~\ref{sub:regularising} this is not as easy as it seems because, although the Dirac matrices still satisfy the Clifford
		algebra, the various identities for their contractions and traces change when we are in $d$ dimensions.  Two useful
		examples are shown below:

		\begin{subequations}
			\begin{equation}
			g_{\mu\nu}g^{\mu\nu} = d
			\end{equation}
			\begin{equation}
			\gamma^\mu\gamma_\nu\gamma_\mu = (d-2)\gamma_nu
			\end{equation}
		\end{subequations}

		Using the \texttt{FORM} package ~\cite{form} to perform the two trace terms present gives:

		\begin{align}
		\begin{split}
			\Tr[\slashed k_1 \gamma_\alpha(\slashed k_1 + \slashed k)\gamma_\mu(\slashed k_2 -
			\slashed k)\gamma^\alpha\slashed k_2\gamma^\mu] &= s[s(8-4d) + \frac{(k_1\cdot k)(k_2\cdot k)}{s}(32-16d) \\
			&- (16-8d)(k_1\cdot k - k_2\cdot k) + k^2(16-12d+2d^2)],
		\end{split}
		\end{align}

		and,

		\begin{align}
		\begin{split}
			\Tr[\slashed k_1 \gamma_\alpha(\slashed k_1 + \slashed k)\gamma_\mu(\slashed k_2 - \slashed k)\gamma_\beta
			\slashed k_2\gamma^\mu]k^\alpha k^\beta &= s[(k_1\cdot k)(k_2\cdot k)(16 - 8d) \\
			&+ k^2(8 - 4d)(k_2\cdot k - k_1\cdot k) - k^4(4 - 2d)],
		\end{split}
		\end{align}

		where $s = 2k_1\cdot k_2$ and we have used the on-shell relations.  After factorising the terms
		quadratic in $d$ and combining the two trace terms we arrive at:

		\begin{equation}
			\overline{\mathcal{A}_0^*\mathcal{A}_v} = -4s\left(\frac{d}{2}-1\right)\frac{g_s^2e^2Q^2}{2}
			\int\frac{d^{d}k}{(2\pi)^{d}}\frac{\mathcal{N}_3(k_1, k_2, k)}{k^2(k_1+k)^2(k_2-k)^2},
		\end{equation}

		where:

		\begin{align}
		\begin{split}
			\mathcal{N}_3(k_1, k_2, k) &= -2s + \frac{8k\cdot k_1k\cdot k_2}{s} + (6+2\xi)(k\cdot k_1 -
			k\cdot k_2) + k^2(d-4) \\&- 4(1-\xi)\frac{k\cdot k_1 k\cdot k_2}{k^2} - (1-\xi)k^2.
		\end{split}
		\end{align}

		Combining this with the particle flux and the two particle phase space we can write an expression
		for the vertex corrected cross-section.  Once again we scale the couplings such that they remain
		dimensionless by defining $g_d^2=g_s^2\mu^{2-\frac{d}{2}}$:

		\begin{subequations}
			\begin{equation*}
			\sigma_v = -4s\left(\frac{d}{2}-1\right)\frac{g_d^2\mu^{2-\frac{d}{2}}e^2Q^2}{4s}2^{1-d}\pi^{\frac{d}{2}-1}
			\frac{\Gamma(\frac{d}{2}-1)}{\Gamma(d-2)}s^\frac{d-4}{2}\int\frac{d^{d}k}{(2\pi)^{d}}\frac{\mathcal{N}_3(k_1, k_2, k)}{k^2(k_1+k)^2(k_2-k)^2},
			\end{equation*}
			\begin{equation*}
			\Rightarrow\sigma_v = -g_d^2\mu^{2-\frac{d}{2}}Q^2 4\pi\alpha\mu^{4-d}2^{1-d}\pi^{\frac{d}{2}-1}\frac{\Gamma(
			\frac{d}{2})}{\Gamma(d-2)}s^\frac{d-4}{2}\int\frac{d^{d}k}{(2\pi)^{d}}\frac{\mathcal{N}_3(k_1, k_2, k)}{k^2(k_1+k)^2(k_2-k)^2},
			\end{equation*}
			\begin{equation*}
			\Rightarrow\sigma_v = -\frac{4\sigma_0}{3}g_d^2\mu^{2-\frac{d}{2}}\int\frac{d^{d}k}{(2\pi)^{d}}
			\frac{\mathcal{N}_3(k_1, k_2, k)}{k^2(k_1+k)^2(k_2-k)^2},
			\end{equation*}
		\end{subequations}

		where we have expressed the virtual rate as a multiplicative correction to the Born level rate.
		We must now use the Feynman parametrisation to re-express the product of propagators as
		a sum by introducing new integration variables.  Using:

		\begin{equation}
			\frac{1}{ab} = \int_0^1dy\frac{1}{(ay+b(1-y))^2},
			\label{eqn:usefulParam}
		\end{equation}

		we have:

		\begin{equation}
			\sigma_v = -\frac{4\sigma_0}{3}g_d^2\mu^{2-\frac{d}{2}}\int\frac{d^{d}k}{(2\pi)^d}
			\int_0^1dy\frac{\mathcal{N}_3(k_1, k_2, k)}{(k^2-2k\cdot k_y)^2k^2},
		\end{equation}

		where $k_y = yk_1 -(1-y)k_2$.  Examining now the integrand we see there are two
		different $k$ dependences and so we partition the terms as follows:

		\begin{equation}
			\sigma_v = -\frac{4\sigma_0}{3}g_d^2\mu^{2-\frac{d}{2}}\int\frac{d^{d}k}{(2\pi)^d}\int_0^1dy
			\left(\frac{\mathcal{N}^{(1)}_3(k_1, k_2, k)}{(k^2-2k\cdot k_y)^2k^2} +
			\frac{\mathcal{N}^{(2)}_3(k_1, k_2, k)}{(k^2-2k\cdot k_y)^2k^4}\right),
		\end{equation}

		where,

		\begin{equation}
			\mathcal{N}^{(1)}_3(k_1, k_2, k) = -2s + \frac{8k\cdot k_1k\cdot k_2}{s} +
			(6+2\xi)(k\cdot k_1 - k\cdot k_2) + k^2(d-4) - (1-\xi)k^2,
		\end{equation}
		and
		\begin{equation}
			\mathcal{N}^{(2)}_3(k_1, k_2, k) = - 4(1-\xi)k\cdot k_1 k\cdot k_2.
		\end{equation}

		Differentiating eqn.~\eqref{eqn:usefulParam} with respect to $a$ and $b$ we get the following useful parametrisations:

		\begin{align}
		\begin{split}
			\frac{1}{a^2b} &= \int_0^1dx\frac{2x}{(ax+b(1-x))^3},\\
			\frac{1}{a^2b^2} &= \int_0^1dx\frac{6x(1-x)}{(ax+b(1-x))^4}.
		\end{split}
		\end{align}

		and taking $a = k^2-2k\cdot k_y$ and $b = k^2$, simplifying the denominators and performing a change of variables $K=k-xp_y$ yields:

		\begin{align}
		\begin{split}
			\sigma_v = -\frac{4\sigma_0}{3}g_d^2\mu^{2-\frac{d}{2}}\int\frac{d^{d}K}{(2\pi)^d}\int_0^1dy\int_0^1dx
			\Bigg(&\frac{2x\mathcal{N}^{(1)}_3(k_1, k_2, K+xk_y)}{(K^2-C)^3} + \\&\frac{6x(1-x)
			\mathcal{N}^{(2)}_3(k_1, k_2, K+xk_y)}{(K^2-C)^4}\Bigg),
		\end{split}
		\end{align}

		where $C = x^2p_y^2$.  The change of variables modifies the numerator terms to:

		\begin{subequations}
			\begin{align}
			\begin{split}
				\mathcal{N}^{(1)}_3(k_1, k_2, K+xk_y) = &-2s + K^2\Big(\frac{4}{d} + d - 5 + \xi\Big) \\ &- (3 + \xi)xs + x^2ys(1-y)(3-d-\xi),
				\label{eqn:numeratorTerms1}
			\end{split}
			\end{align}
			\begin{equation}
			\mathcal{N}^{(2)}_3(k_1, k_2, K+xk_y) = (1-\xi)\left(x^2ys^2(1-y)-\frac{2s}{d}K^2\right).
			\label{eqn:numeratorTerms2}
			\end{equation}
			\label{eqn:numeratorTerms}
		\end{subequations}

		We can now perform the integrations over $K$ with the aid of the following result:

		\begin{equation}
			\int\frac{d^{d}K}{(2\pi)^d}\frac{(K^2)^m}{(K^2-C)^n} = \frac{i(-1)^{m-n}}{(4\pi)^\frac{d}{2}}
			C^{m-n+\frac{d}{2}}\frac{\Gamma(m+\frac{d}{2})\Gamma(n-m-\frac{d}{2})}{\Gamma(\frac{d}{2})\Gamma(n)}.
			\label{eqn:eqn54}
		\end{equation}

		Looking at the $K$ structure of eqs.~\eqref{eqn:numeratorTerms} we can see that there are going to be four forms
		of eqn.~\eqref{eqn:eqn54} needed in this calculation.  I will not show the calculation for every integral
		but will show one as an example of how the calculations can proceed.  Consider the contribution of the first
		term of eqn.~\eqref{eqn:numeratorTerms1}:

		\begin{align}
		\begin{split}
			 I &= -4s\int_0^1dy\int_0^1dxx\int\frac{d^{d}K}{(2\pi)^d}\frac{1}{(K^2-C)^3} \\
			 &= 4si\int_0^1dy\int_0^1dxx(4\pi)^{-\frac{d}{2}}
			C^{-3+\frac{d}{2}}\frac{\Gamma(\frac{d}{2})\Gamma(3-\frac{d}{2})}{\Gamma(\frac{d}{2})\Gamma(3)}.
		\end{split}
		\end{align}

		From above we see that $C=x^2k_y=-x^2y(1-y)s$ and so:

		\begin{align}
		\begin{split}
			I = 4si(4\pi)^{-\frac{d}{2}}\Gamma(3-\frac{d}{2})(-s)^{-3+\frac{d}{2}}\int_0^1dy\int_0^1dxx^{-5+d}
			y^{\left(-2+\frac{d}{2}\right)-1}(1-y)^{\left(-2+\frac{d}{2}\right)-1},
		\end{split}
		\end{align}

		Therefore:

		\begin{equation}
			I = 4si(4\pi)^{-\frac{d}{2}}\Gamma\left(3-\frac{d}{2}\right)(-s)^{-3+\frac{d}{2}}
			\frac{1}{d-4}\frac{\Gamma^2(\frac{d}{2}-2)}{\Gamma(d-4)}.
		\end{equation}

		Choosing $d=4+\epsilon$ (with the intention of taking the limit $\epsilon\rightarrow0$
		once it is safe to do so), and manipulating the gamma functions to expose the pole structure gives:

		\begin{equation}
			-4\int_0^1dy\int_0^1dxx\int\frac{d^{d}K}{(2\pi)^d}\frac{1}{(K^2-C)^3} = 4(-s)^{\frac{\epsilon}{2}}i(4\pi)^{-2-\frac{\epsilon}{2}}
			\frac{4}{\epsilon^2}\frac{\Gamma\left(1-\frac{\epsilon}{2}\right)\Gamma^2\left(1+\frac{\epsilon}{2}\right)}{\Gamma(1+\epsilon)},
		\end{equation}

		which is clearly divergent in the limit $\epsilon\to0$.  The other integrals follow similarly and
		the combined result can be expressed as:

		\begin{equation}
			\sigma_v = \frac{2\alpha_s}{3\pi}\sigma_0\Big(\frac{s}{4\pi\mu^2}\Big)^{\frac{\epsilon}{2}}\frac{\Gamma
			\Big(1-\frac{\epsilon}{2}\Big)\Gamma^2\Big(1+\frac{\epsilon}{2}\Big)}{\Gamma(1+\epsilon)}\Big(-\frac{8}{\epsilon^2} +
			\frac{6}{\epsilon} - \frac{8+4\epsilon}{1+\epsilon}\Big),
		\end{equation}

		where we have used $\alpha_s=\frac{g_d^2}{4\pi}$. Expanding the product of gamma matrices for $\epsilon\rightarrow0$
		gives:

		\begin{subequations}
			\begin{equation}
			\frac{\Gamma\left(1-\frac{\epsilon}{2}\right)\Gamma^2\left(1+\frac{\epsilon}{2}\right)}{\Gamma(1+\epsilon)} =
			\frac{\gamma_E}{2}\epsilon + \left(\frac{\gamma_E^2}{8} - \frac{\pi^2}{48}\right)\epsilon^2 + \mathcal{O}(\epsilon^3),
			\end{equation}
			\begin{equation}
			\left(\frac{s}{4\pi\mu^2}\right)^{\frac{\epsilon}{2}} = e^{\ln{\left(\frac{s}{4\pi\mu^2}\right)^{\frac{\epsilon}{2}}}} =
			e^{\frac{\epsilon}{2}\ln\left(\frac{s}{4\pi\mu^2}\right)} = 1 + \frac{\epsilon}{2}\ln\left(\frac{s}{4\pi\mu^2}\right) + \mathcal{O}(\epsilon^2),
			\end{equation}
		\end{subequations}

		where $\gamma_E$ is Euler's constant.  Finally then we have:

		\begin{align}
		\begin{split}
			\sigma_v = \frac{2\alpha_s}{3\pi}\sigma_0\Big[&-\frac{8}{\epsilon^2} + \frac{1}{\epsilon}\left(6-4\gamma_E-4L\right) +
			\gamma_E(3-\gamma_E)\\&-8+\frac{\pi^2}{6}+\pi^2-L^2-(2\gamma_E-3)L\Big],
		\end{split}
		\end{align}

		where $L = \ln{\left(\frac{s}{4\pi\mu^2}\right)}$.  We can now see that
		the result for the vertex correction is gauge independent as the $\xi$ dependence has completely cancelled.  We also see that
		the parameter introduced to fix the coupling to be dimensionless appears in the final result;  this
		is often the case when using dimensional regularisation and the modified minimal subtraction renormalisation scheme.

	\subsubsection{The Real $\mathcal{O}(\alpha_s)$ Corrections}

		The real gluon emission diagrams which contribute to the $\mathcal{O}(\alpha_s)$ corrections are
		figs.~\eqref{fig:NLOfig_5} and~\eqref{fig:NLOfig_6}.  These diagrams have an indistinguishable final
		state and so the real contribution, $\mathcal{A}_r$, will be of the form:

		\begin{equation}
			|\mathcal{A}_r|^2 = |\mathcal{A}_{left} + \mathcal{A}_{right}|^2 =
			|\mathcal{A}_{left}|^2 + |\mathcal{A}_{right}|^2 + 2\mathcal{A}_{left}\mathcal{A}_{right}^*,
		\end{equation}

		where $\mathcal{A}_{left}$ and $\mathcal{A}_{right}$ refer to figs.~\eqref{fig:NLOfig_5} and~\eqref{fig:NLOfig_6} respectively and are given by:

		\begin{subequations}
			\begin{equation}
			\mathcal{A}_{left} = -Qeig_sT^a_{ij}\overline{u}(k_2)\gamma^\mu\frac{\slashed k_1 + \slashed k}{(k_1 + k)^2}\gamma^\nu v(k_1)\epsilon_\nu\eta_\mu,
			\end{equation}
			\begin{equation}
			\mathcal{A}_{right} = -Qeig_sT^a_{ij}\overline{u}(k_2)\gamma^\nu\frac{\slashed k_2 + \slashed k}{(k_2 + k)^2}\gamma^\mu v(k_1)\epsilon_\nu\eta_\mu.
			\end{equation}
		\end{subequations}

		In the calculation of the terms of eqn.~\eqref{eqn:MEBreakdown} it will be useful to write the energy fractions for each
		particle as $x_i = \frac{2E_i}{\sqrt{s}}$ (where $i=1$ is the external antiquark, $i=2$ is the antiquark
		and $i=3$ is the external gluon).  In terms of these invariants the contraction of any two external
		particles simplifies to $p_i\cdot p_j = \frac{1}{2}s(1-x_k)$ which (since we are still assuming our
		quarks can be taken to be massless) gives a simple expression for the Mandelstam variables.
		Evaluating the modulus squared terms gives:

		\begin{subequations}
			\begin{equation}
			|\mathcal{A}_{left}|^2  = \frac{Q^2e^2g_s^2}{(k_1+k)^4}\Tr(T^aT^a)\Tr(\slashed k_2 \gamma^\mu (\slashed k_1 + \slashed k)
			\gamma^\nu \slashed k_1 \gamma_\nu (\slashed k_1 + \slashed k) \gamma_\mu),
			\end{equation}
			\begin{equation}
			|\mathcal{A}_{right}|^2 = \frac{Q^2e^2g_s^2}{(k_2+k)^4}\Tr(T^aT^a)\Tr(\slashed k_2 \gamma^\nu (\slashed k_2 + \slashed k)
			\gamma^\mu \slashed k_2 \gamma_\mu (\slashed k_2 + \slashed k) \gamma_\nu),
			\end{equation}
			\begin{equation}
			\mathcal{A}_{left}\mathcal{A}_{right}^* = \frac{Q^2e^2g_s^2}{(k_2+k)^2(k_1+k)^2}\Tr(T^aT^a)\Tr(\slashed k_2\gamma^\mu
			(\slashed k_1 + \slashed k) \gamma^\nu \slashed k_1 \gamma_\mu (\slashed k_2 + \slashed k) \gamma_\nu).
			\end{equation}
		\end{subequations}

		Evaluating the trace terms and rearranging in terms of the energy fractions gives:

		\begin{subequations}
			\begin{equation}
			|\mathcal{A}_{left}|^2  = 32Q^2e^2g_s^2\left(1+\frac{\epsilon}{2}\right)^2\frac{1-x_1}{1-x_2},
			\end{equation}
			\begin{equation}
			|\mathcal{A}_{right}|^2 = 32Q^2e^2g_s^2\left(1+\frac{\epsilon}{2}\right)^2\frac{1-x_2}{1-x_1},
			\end{equation}
			\begin{equation}
			2\mathcal{A}_{left}\mathcal{A}_{right}^* = 64Q^2e^2g_s^2\left(1+\frac{\epsilon}{2}\right)\left(-\frac{\epsilon}{2}-2\frac{1-x_3}{(1-x_1)(1-x_2)}\right).
			\end{equation}
		\end{subequations}

		Summing these expressions gives:

		\begin{equation}
			|\mathcal{A}_r|^2 = 32Q^2e^2g_s^2\left[\left(1+\frac{\epsilon}{2}\right)^2\frac{x_1^2+x_2^2}{(1-x_2)(1-x_1)} +
			\epsilon\left(1+\frac{\epsilon}{2}\right)\frac{2-2x_1-2x_2+x_1x_2}{(1-x_2)(1-x_1)}\right].
			\label{eqn:eqn67}
		\end{equation}

		As with the virtual contributions, we are interested in the observable cross-section and so we must
		include the phase space factor for a three particle final state.  Unlike the two particle phase space
		calculation the three particle phase space, $\int d^{3d-3}R_3$, cannot be integrated completely and we are left with a
		differential in terms of the energy fractions defined above:

		\begin{equation}
			\frac{d^2R_3}{dx_1dx_2} = \frac{s}{16(2\pi)^3}\left(\frac{s}{4\pi}\right)^\epsilon\frac{1}{\Gamma(2+\epsilon)}
			\left(\frac{1-z^2}{4}\right)^{\frac{\epsilon}{2}}x_1^\epsilon x_2^\epsilon,
			\label{eqn:eqn68}
		\end{equation}

		where $z = 1 - 2\frac{1-x_1-x_2}{x_1x_2}$.  Combining eqs.~\eqref{eqn:eqn67} and~\eqref{eqn:eqn68} with a flux factor gives:

		\begin{equation}
			\frac{d^2\sigma_r}{dx_1dx_2} = \frac{2Q^2e^2g_s^2F(x_1, x_2; \epsilon)}{\pi}\left(\frac{s}{4\pi}\right)^\epsilon
			\frac{1}{\Gamma(2+\epsilon)}\left(\frac{1-z^2}{4}\right)^{\frac{\epsilon}{2}}x_1^\epsilon x_2^\epsilon,
		\end{equation}

		where we define $F(x_1, x_2; \epsilon)$ as the algebraic factor in square brackets from eqn.~\eqref{eqn:eqn67}.  Switching to
		a dimensionless coupling and introducing $\alpha_s$ as above comparing with the Born cross-section in
		eqn.~\eqref{eqn:bornCrossSection} this can be written as:

		\begin{equation}
			\frac{d^2\sigma_r}{dx_1dx_2} = \frac{2\alpha_s\sigma_0}{3\pi}F(x_1, x_2; \epsilon)\left(\frac{s}{4\pi\mu^2}\right)^
			{\frac{\epsilon}{2}}\frac{1}{\Gamma(2+\frac{\epsilon}{2})}\left(\frac{1-z^2}{4}\right)^{\frac{\epsilon}{2}}x_1^\epsilon x_2^\epsilon.
		\end{equation}

		Integrating over the allowed region of $x_1$ and $x_2$ gives:

		\begin{equation}
			\sigma_r = \frac{2\alpha_s\sigma_0}{3\pi}\left(\frac{s}{4\pi\mu^2}\right)^{\frac{\epsilon}{2}}\frac{1}{\Gamma(2+\frac{\epsilon}{2})}
			\int_0^1dx_1x_1^\epsilon\int^1_{1-x_1}x_2^\epsilon\left(\frac{1-z^2}{4}\right)^{\frac{\epsilon}{2}}F(x_1, x_2;\epsilon).
			\label{eqn:eqn72}
		\end{equation}

		We can define a change of variables $x_2=1-vx_1$ to decouple these integrals and therefore:

		\begin{equation}
			\sigma_r = \frac{2\alpha_s\sigma_0}{3\pi}\left(\frac{s}{4\pi\mu^2}\right)^{\frac{\epsilon}{2}}\frac{\Gamma^2
			\left(1+\frac{\epsilon}{2}\right)}{\Gamma\left(1+\frac{3\epsilon}{2}\right)}\left[\frac{8}{\epsilon^2} - \frac{6}{\epsilon} + \frac{19}{2}\right].
		\end{equation}

		Further expanding the Gamma functions gives:

		\begin{equation*}
			\sigma_r = \frac{2\alpha_s}{3\pi}\sigma_0\left[\frac{8}{\epsilon^2} + \frac{1}{\epsilon}\left(-6+4\gamma_E+4L\right)-
			\gamma_E(3-\gamma_E)-\frac{57}{6}+\frac{7\pi^2}{6}+L^2+(2\gamma_E-3)L\right].
		\end{equation*}

		As in the case of the virtual corrections this is divergent in the limit $\epsilon\rightarrow0$ and exhibits a residual
		dependence on $\mu$.

	\subsubsection{Cancellation of divergences}

		Having now found the vertex corrections and the real corrections up to $\mathcal{O}(\epsilon^2)$
		we can write the next-to-leading order cross-section by simply summing the two:

		\begin{align}
			\sigma_{NLO} &= \sigma_r + \sigma_v = \frac{\alpha_s}{\pi}\sigma_0.
		\end{align}

		So the total cross-section to next-to-leading order accuracy is:

		\begin{align}
			\sigma &= \sigma_0\left(1 + \frac{\alpha_s}{\pi}\right) + \mathcal{O}(\alpha_s^2).
		\end{align}

		The fact that the infrared divergences in both the real and virtual emission NLO diagrams cancel is an example of the
		KLN theorem~\cite{KLN} which states that the Standard Model is completely free of infrared divergences at all orders.
