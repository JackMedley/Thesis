% \chapter{The Faddeev-Popov Trick}
% \label{chap:fpTrick}

% All that remains to be done is to evaluate the gluon propagator.  As in QED when trying
% to compute the propagator of a massless gauge boson we can use the work of Faddeev and Popov.
% The functional integral we want to evaluate is in the form:

% \begin{equation}
% 	\int DAe^{-\frac{i}{4}\int d^4xF^a_{\mu\nu}F^{a\mu\nu}}.
% \end{equation}

% Where $DA=\prod_x\prod_{a, \mu}dA^a_\mu$.  As briefly outlined above we would like to perform
% a functional integration over all possible gauge choices and then pick out the subset of gauges
% we are interested in by enforcing the gauge condition $G(A)=0$ to eliminate over-counting.
% This constraint may be written as ~\cite{p&s}:

% \begin{equation}
% \int D\alpha(x)\delta(G(A^\alpha))Det\left(\frac{\delta G(A^\alpha)}{\delta\alpha(x)}\right) = 1.
% \end{equation}

% Where $A^\alpha_\mu = A_\mu - \frac{1}{g_s}\partial_\mu\alpha(x)$.  Making a gauge
% transformation ($A_\mu\rightarrow A^\alpha_\mu$) and inserting equation (18):

% \begin{subequations}
% \begin{equation}
% \int DAe^{-\frac{i}{4}\int d^4F^a_{\mu\nu}F^{a\mu\nu}} = \int DA\int D\alpha(x)\delta(G(A^\alpha))
% Det\left(\frac{\delta G(A^\alpha)}{\delta\alpha(x)}\right)e^{-\frac{i}{4}\int d^4F^a_{\mu\nu}F^{a\mu\nu}},
% \end{equation}
% \begin{equation}
% = \int D\alpha(x)\int DA\delta(G(A^\alpha))Det\left(\frac{\delta G(A^\alpha)}{\delta\alpha(x)}\right)
% e^{-\frac{i}{4}\int d^4F^a_{\mu\nu}F^{a\mu\nu}}.
% \end{equation}
% \end{subequations}

% We are free to change the functional integration variable to $A_\mu^\alpha$ since everything
% is gauge invariant leading to an integrand which \emph{only} depends on $A^\alpha_\mu$.
% We can therefore simply relabel back to $A_\mu$:

% \begin{equation}
% = \left(\int D\alpha(x)\right)\int DA\delta(G(A))Det\left(\frac{\delta G(A)}{\delta\alpha(x)}\right)
% e^{-\frac{i}{4}\int d^4F^a_{\mu\nu}F^{a\mu\nu}}.
% \end{equation}

% The functional integration can now just be factored out as a constant and we can choose the
% function $G(A)$ as a generalisation of the Lorentz gauge: $G(A)=\partial^\mu A^a_\mu-\omega^a$.
% This choice leads us to the correct gluon propagator - along with our free parameter, $\xi$:

% \begin{equation}
% \langle0|A_a(x)A_b(y)|0\rangle = G_F^{\mu\nu}(x-y) = \int \frac{d^4x}{(2\pi)^4}e^{-ik\cdot(x-y)}
% \delta_{ab}\frac{-i}{k^2+i\epsilon}\left(g^{\mu\nu}-(1-\xi)\frac{k^\mu k^\nu}{k^2}\right).
% \end{equation}

% but because the QCD gauge transformation is more involved than the
% QED equivalent the determinant term still depends on $A_\mu$:

% \begin{equation}
% Det\left(\frac{\delta G(A)}{\delta\alpha(x)}\right) = Det\left(\frac{\partial_\mu D^\mu}{g_s}\right).
% \end{equation}

% We can however simply invent another type of field and choose to write out determinant as

% \begin{equation}
% 	Det\left(\frac{\delta G(A)}{\delta\alpha(x)}\right) = \int D\chi D\overline{\chi}
% 	e^{i\int d^4x\overline{\chi}(-\partial_\mu D_\mu)\chi}.
% \end{equation}

% These non-physical modes are called the Faddeev-Popov ghosts/anti-ghosts and are a consequence
% of enforcing gauge invariance - they are represented by the final term in equation (12a).